\documentclass[10pt, a4paper]{article}
\usepackage[slovene]{babel}
\usepackage[T1]{fontenc}
\usepackage[utf8]{inputenc}
\usepackage{lmodern}
\usepackage{amsmath}
\usepackage{amsthm}
\usepackage{amssymb}
\usepackage{parskip}
\usepackage{pgfplots}
\pgfplotsset{compat=1.16}
\usepgfplotslibrary{colormaps,fillbetween}
\usepackage{comment}
\usepackage{graphicx}
\usepackage{booktabs}
\usepackage{array}
\usepackage{adjustbox}
\usepackage{physics}

%%%%%%%%%%%%%%%%%%%%%%%%%%%%%%%%%%%%%%%%%%%%%%%%%%%%%%%%%%%%%%%%%%%%%%

\usepackage[top=105pt, bottom=75pt, left=75pt, right=75pt]{geometry}
\setlength{\headsep}{15pt}
\setlength{\footskip}{45pt}

\usepackage{xcolor}
\usepackage{lipsum}

\usepackage{ifthen}
\usepackage{tikz}
\usetikzlibrary{calc}
\usetikzlibrary{cd}
\usetikzlibrary{babel}
\usetikzlibrary{lindenmayersystems}
\pgfdeclarelindenmayersystem{cantor set}{
  \rule{F -> FfF}
  \rule{f -> fff}
}
%%%%%%%%%%%%%%%%%%%%%%%%%%%%%%%%%%%%%%%%%%%%%%%%%%%%%%%%%%%%%
\usepackage{tcolorbox}
\tcbuselibrary{skins, breakable}


%%%%%%%%%%%%%%%%%%%%%%%%%%%%%%
%%% with separate title
\xdefinecolor{thmTopColor}{RGB}{102, 102, 238}
\xdefinecolor{thmBackColor}{RGB}{245, 245, 255}

%%%%%%%%%%%%%%%%%%%%%%%%%%%%%%%%%%%%%%%%%%%%%%%%%%%%%%%%%%%%%%%%%%%%%%%%

\graphicspath{ {./images/} }

\newtheorem{izr}{Izrek}[section]

\newenvironment{thmbox}[1]{%
  \tcolorbox[%
  empty,
  parbox=false,
  noparskip,
  enhanced,
  breakable,
  sharp corners,
  boxrule=-1pt,
  left=2ex,
  right=0ex,
  top=0ex,
  boxsep=1ex,
  before skip=2.5ex plus 2pt,
  after skip=2.5ex plus 2pt,
  colback=thmBackColor,
  colframe=white,
  coltitle=black,
  colbacktitle=thmBackColor,
  fonttitle=\bfseries,
  title=#1,
  titlerule=1pt,
  titlerule style=thmTopColor,
  overlay unbroken and last={%
    \draw[color=thmTopColor, line width=1.25pt]
    ($(frame.north west)+(.5em, -4.1ex)$)
    -- ($(frame.south west)+(.5em, 1ex)$) -- ++(2em, 0);
  }]
}{\endtcolorbox}

\newenvironment{izrek}[1][]{% before
  \refstepcounter{izr}%
  \ifthenelse{\equal{#1}{}}{%
    \begin{thmbox}{Izrek \theizr.}\itshape\hspace{-.75ex}%
  }{%
    \begin{thmbox}{Izrek \theizr%
        \hspace{.75ex}(\textnormal{#1}).}\itshape\hspace{-.75ex}
    }}
  {\end{thmbox}
}

{\theoremstyle{plain}
\newtheorem{posledica}[izr]{Posledica}
\newtheorem{trditev}[izr]{Trditev}

}

{\theoremstyle{definition}
\newtheorem{defi}[izr]{Definicija}
\newtheorem{aksiom}{Aksiom}[section]
}

\newenvironment{noticeB}{%
  \tcolorbox[%
  notitle,
  empty,
  enhanced,  % delete the edge of the bottom page for a broken box
  breakable,
  coltext=black,
  colback=white, 
  fontupper=\rmfamily,
  %parbox=false,
  noparskip,
  sharp corners,
  boxrule=-1pt,  % width of the box' edges
  frame hidden,
  left=7pt,  % inner space from text to the left edge
  right=7pt,
  top=5pt,
  bottom=5pt,
  % boxsep=0pt,
  before skip=2.5ex plus 2pt,
  after skip=2.5ex plus 2pt,
  borderline west = {1.5pt}{-0.1pt}{blue!30!black}, % second argument = offset
  overlay unbroken and last={%
    \draw[color=black, line width=1.25pt]
    ($(frame.south west)+(1.pt, -0.1pt)$) -- ++(2em, 0);
  }
  ]}
{\endtcolorbox}

\newenvironment{definicija}{\begin{noticeB}\begin{defi}}{%
\end{defi}\end{noticeB}}

{\theoremstyle{remark}
\newtheorem*{opomba}{Opomba}
}

\newtheorem{primer}[izr]{Primer}
\tcolorboxenvironment{primer}{%
  enhanced jigsaw,
  boxrule=-1pt,
  colframe=gray!15,
  %borderline west={2pt}{0pt}{black},  % second argument is the offset
  interior hidden,
  sharp corners,
  breakable,
  before skip=2.5ex plus 2pt,
  after skip=2.5ex plus 2pt
}

%%%%%%%%%%%%%%%%%%%%%%%%%%%%%%%%%%%%%%%%%%%%%%%%%%%%%%%%%%%%%%%%%%%%%%%%
\newtheorem{lema}[izr]{Lema}
\tcolorboxenvironment{lema}{%
  enhanced jigsaw,
  boxrule=-1pt,
  sharp corners,
  colframe=white,
  borderline west={2pt}{0pt}{orange},  % second argument is the offset
  interior hidden,
  breakable,
  before skip=2.5ex plus 2pt,
  after skip=2.5ex plus 2pt
}

%%%%%%%%%%%%%%%%%%%%%%%%%%%%%%%%%%%%%%%%%%%%%%%%%%%%%%%%%%%%%%%%%%
\newenvironment{noticeC}{%
  \tcolorbox[%
  notitle,
  empty,
  enhanced,  % delete the edge of the bottom page for a broken box
  breakable,
  coltext=black, 
  fontupper=\rmfamily,
  %parbox=false,
  noparskip,
  sharp corners,
  boxrule=-1pt,  % width of the box' edges
  frame hidden,
  left=7pt,  % inner space from text to the left edge
  right=7pt,
  top=5pt,
  bottom=5pt,
  % boxsep=0pt,
  before skip=2.5ex plus 2pt,
  after skip=2.5ex plus 2pt,
  %borderline west = {1.5pt}{-0.1pt}{gray}, % second argument = offset
  overlay unbroken and last={%
    %\draw[color=gray, line width=1.25pt]
    %($(frame.west)$);
    %\draw[color=gray, line width=1.25pt]
    %($(frame.east)$);
  },
  ]}
{\endtcolorbox}

\newenvironment{dokaz}%
  {\begin{noticeC}\begin{proof}}%
  {\end{proof}\end{noticeC}}

%%%%%%%%%%%%%%%%%%%%%%%%%%%%%%%%%%%%%%%%%%%%%%%%%%%%%%%%%%%%%%%%%%%%

\makeatletter
\newlength\xvec@height%
\newlength\xvec@depth%
\newlength\xvec@width%
\newcommand{\xvec}[2][]{%
  \ifmmode%
    \settoheight{\xvec@height}{$#2$}%
    \settodepth{\xvec@depth}{$#2$}%
    \settowidth{\xvec@width}{$#2$}%
  \else%
    \settoheight{\xvec@height}{#2}%
    \settodepth{\xvec@depth}{#2}%
    \settowidth{\xvec@width}{#2}%
  \fi%
  \def\xvec@arg{#1}%
  \def\xvec@dd{:}%
  \def\xvec@d{.}%
  \raisebox{.2ex}{\raisebox{\xvec@height}{\rlap{%
    \kern.05em%  (Because left edge of drawing is at .05em)
    \begin{tikzpicture}[scale=1]
    \pgfsetroundcap
    \draw (.05em,0)--(\xvec@width-.05em,0);
    \draw (\xvec@width-.05em,0)--(\xvec@width-.15em, .075em);
    \draw (\xvec@width-.05em,0)--(\xvec@width-.15em,-.075em);
    \ifx\xvec@arg\xvec@d%
      \fill(\xvec@width*.45,.5ex) circle (.5pt);%
    \else\ifx\xvec@arg\xvec@dd%
      \fill(\xvec@width*.30,.5ex) circle (.5pt);%
      \fill(\xvec@width*.65,.5ex) circle (.5pt);%
    \fi\fi%
    \end{tikzpicture}%
  }}}%
  #2%
}
\makeatother

% --- Override \vec with an invocation of \xvec.
\let\stdvec\vec
\renewcommand{\vec}[1]{\xvec[]{#1}}
% --- Define \dvec and \ddvec for dotted and double-dotted vectors.
\newcommand{\dvec}[1]{\xvec[.]{#1}}
\newcommand{\ddvec}[1]{\xvec[:]{#1}}
\newcommand{\stcomp}[1]{{#1}^{\mathsf{c}}}

%%%%%%%%%%%%%%%%%%%%%%%%%%%%%%%%%%%%%%%%%%%%%%%%%%%%%%%%%%%%%%%%%%%%
\newcommand{\bigslant}[2]{{\raisebox{.2em}{$#1$}\left/\raisebox{-.2em}{$#2$}\right.}}


\setlength{\parskip}{1em}


\begin{document}

\title{Uvod v diferencialno geometrijo}
\author{Jaša Knap}
\date{}
\maketitle

\section{Uvod}%
\label{sec:Uvod}

\begin{definicija}
\label{def_mnt}
Topološki prostor $M$ je $n$-dimenzionalna mnogoterost, če za vsak $m \in M$ obstaja okolica $m \in  U \subseteq M$ in homeomorfizem $\varphi : U \to V^{\text{odp}} \subseteq \mathbb{R}^n$ (pri tem je $V \approx B^{n}$).      
\end{definicija}


\begin{primer}
Naslednje množice so primeri mnogoterosti.

	\begin{enumerate}
	\item $M = \mathbb{R}^n$ je $n$-dimenzionalna mnogoterost,
	\item $S^1$ je $1$-dimenzionalna mnogoterost,
	\item $S^{n} = \left\{\left( x_1, x_2, \ldots, x_n, x_{n+1} \right)  \middle| \sum_{j=1}^{n+1} x_{i}^2 = 1\right\} \subseteq \mathbb{R}^{n+1}$ je $n$-dimenzionalna mnogoterost,
	\item Projektivni prostori $\mathbb{R}P^{n} =  \bigslant{B^{n}}{\sim} $, kjer je $\vec{x}  \sim \vec{y}  \iff \vec{y}  = -\vec{x}$ so $n$-dimenzionalne mnogoterosti.
	\item Grupa \[
			\operatorname{SU}\left( 2 \right) = \left\{ g =  \begin{pmatrix}
	\alpha & \beta \\
	- \overline{\beta}  & \overline{\alpha}  \\
\end{pmatrix}  \middle|\, \alpha, \beta \in  \mathbb{C}, \det g = 1
\right\} 	
\]je $3$-dimenzionalna mnogoterost. Topološko in geometrijsko je namreč $\operatorname{SU}\left( 2 \right) = S^3.$ To je primer Lijeve grupe.
\item Grupa \[
		\operatorname{SO}\left( 3 \right) = \left\{ g = 
\begin{pmatrix}
	a_{11} & a_{12} & a_{13} \\
	a_{21} & a_{22} & a_{23} \\
	a_{31} & a_{32} & a_{33} \\
\end{pmatrix}
 \middle|\, g^{T} = g^{-1}, \det  g = 1
		\right\}. 
\] Izkaže se, da je $\operatorname{SO}\left( 3\right ) =  \bigslant{B^3}{\sim } = \mathbb{R}P^3.$ To velja, ker vsaka preslikava iz $\operatorname{SO}\left( 3\right )$ predstavlja rotacijo prostora, vsako rotacijo pa lahko predstavimo z osjo in velikostjo kota vrtenja. Pri tem kota $\pi$ in $-\pi$ predstavljata vrtenje za isti kot. Če točki v krogli $B\left( 0, \pi \right)^{3} \approx B^3$ priredimo os in njeno razdaljo od izhodišča proglasimo za velikost kota vrtenja ter enačimo iste rotacije, dobimo natanko projektivni prostor $\mathbb{R}P^3$.
\end{enumerate}
\end{primer}

\subsection{Gladke mnogoterosti}%
\label{sub:Gladke_mnogoterosti}

Na topoloških mnogoterostih bi radi znali odvajati različne objekte, kot so na primer funkcije, krivulje, tenzorji itd. Zato moramo mnogoterosti opremiti z dodatno strukturo. Za začetek se spomnimo definicije odvedljivosti preslikav v evklidskih prostorih.

\begin{definicija}
Preslikava $F: W^{\text{odp}}  \subseteq  \mathbb{R}^n \to  \mathbb{R}^n$ je odvedljiva v točki $w \in  W$, če obstaja linearna preslikava $A : \mathbb{R}^n \to  \mathbb{R}^n$ in preslikava $\mathcal{O}: W \to \mathbb{R}^n$, da za vse ustrezne argumente velja \[
	F\left( w + h \right) = F\left( w \right) + Ah + \mathcal{O}\left( h \right)\] in $\lim_{h \to 0} \frac{\lvert\lvert \mathcal{O}\left( h \right) \rvert\rvert }{\lvert\lvert h \rvert\rvert } = 0.$ Odvod preslikave $F$ v točki $w$ je preslikava $A = D_wF = \left( DF \right)_w.$ 
\end{definicija}


\begin{definicija}
\label{def_odvedljivost}
Preslikava $F: W^{\text{odp}}  \subseteq  \mathbb{R}^n \to  \mathbb{R}^n$ je odvedljiva na množici $W$, če je odvedljiva v vsaki točki $w \in  W$.\end{definicija}

\begin{definicija}
\label{def_difeomorfizem}
Difeomorfizem je bijektivna odvedljiva preslikava, ki ima odvedljiv inverz. 

\end{definicija}

\begin{definicija}
\label{def_atlas}
Naj bo $M$ $n$-dimenzionalna mnogoterost. Gladek atlas $\mathcal{U}$ na $M$ je družina parov $\mathcal{U} = \left\{  \left( U_\alpha, \varphi_\alpha\right)  \mid  \alpha \in  A \right\},$ če za vsak $\alpha \in A$ velja: \begin{enumerate}
	\item $U_\alpha^{\text{odp}} \subseteq M$
	\item $\varphi_\alpha: U_\alpha \to  V_\alpha \subseteq  \mathbb{R}^n$ je homeomorfizem za nek $V_\alpha \subseteq  \mathbb{R}^n$
	\item $\left\{ U_\alpha  \middle|\, \alpha\in  A \right\}$ je pokritje $M$
	\item za vsaka $\alpha, \beta \in A$ je preslikava $g_{\alpha \beta}  = \varphi_\beta \circ \varphi_\alpha^{-1}: (\varphi_\alpha)_{*}(U_\alpha \cap  U_\beta) \to(\varphi_\beta)_{*}(U_\alpha \cap  U_\beta)$ difeomorfizem     
\end{enumerate}
Dodatek: Če so vse prehodne preslikave $g_{\alpha \beta}$  $k$-difeomorfizmi z zveznim $k$-tim odvodom, imamo $\mathcal{C}^{k}$-atlas. Če so vse preslikave gladke, imamo $\mathcal{C}^{\infty}$-atlas, če so vse analitične, pa $\mathcal{C}^{\omega}$-atlas.     
 
\end{definicija}

\begin{opomba}
	Preslikava $g_{\alpha \beta}$  iz prejšnje definicije je preslikava iz $U_\alpha \subseteq  \mathbb{R}^n \to  \mathbb{R}^n$. Torej jo znamo odvajati in vemo, da je v izbranih koordinatah na $\mathbb{R}^n$ matrika odvoda enaka Jacobijevi matriki: \[
F\left( x_1, \ldots, x_{n} \right) = \begin{pmatrix}F_1(x_1, \dots, x_n)\\ \vdots\\ F_n(x_1, \dots, x_n) \end{pmatrix}	\implies D_wF = \begin{pmatrix}
\frac{\partial F_1}{\partial x_1} & \dots & \frac{\partial F_1}{\partial x_n} \\
\vdots & \ddots & \vdots \\
\frac{\partial F_n}{\partial x_1} & \dots & \frac{\partial F_n}{\partial x_n}
\end{pmatrix}_w	. 
\]     
\end{opomba}

\begin{definicija}
\label{def_gladka_mnt}
Topološka mnogoterost $M$, ki premore kakšen gladek atlas, je gladka mnogoterost.
\end{definicija}

Za motivacijo naslednje definicije se spomnimo dejstva, da vemo, kakšne so gladke preslikave iz $\mathbb{R}^n \to  \mathbb{R}^n$. Nismo pa še definirali gladkih preslikav iz mnogoterosti $M \to \mathbb{R}$. 

\begin{definicija}
\label{def_gladke_preslikave}
 Naj bo $M$ $n$-dimenzionalna mnogoterost. Funkcija $f : M \to  \mathbb{R}$ je gladka, če je gladka vsaka preslikava $f \circ \varphi_\alpha^{-1} : V_\alpha \subseteq  \mathbb{R}^n \to  \mathbb{R}$.
\end{definicija}


\begin{definicija}
\label{def_gladka_krivulja}
 Naj bo $\left( M, \mathcal{U} \right)$  gladka mnogoterost. Krivulja $\gamma : \left( a, b \right) \to  M$  je gladka krivulja, v $M$, če za $\forall  \alpha \in A$ velja, da je $\varphi_\alpha \circ \gamma : \left( a, b \right) \to  V_\alpha \subseteq  \mathbb{R}^n$ gladka krivulja v $V_\alpha \subseteq  \mathbb{R}^n$.      
\end{definicija}

\begin{definicija}
\label{def_ekvivalnentnost_atlasov}
 Atlasa $\mathcal{U} = \left\{ \left( U_\alpha, \varphi_\alpha \right)  \middle|\, \alpha \in  A\right\}$ in $\mathcal{V} = \left\{ \left( W_\beta, \varphi_\beta \right)  \middle|\,  \beta \in  B\right\}$ na mnogoterosti $M$ sta ekvivalentna, če za vsak par $\left( \alpha, \beta \right) \in  A \times  B$ iz $U_\alpha \cap  W_\beta \neq \emptyset$ sledi, da je \[
 \psi_\beta \circ  \varphi_\alpha^{-1} : \left( \varphi_\alpha \right)_{*}\left( U_\alpha \cap  W_\beta \right) \subseteq  \mathbb{R}^n \to \left( \psi_\beta \right)_{*}\left( U_\alpha \cap  W_\beta \right)\subseteq  \mathbb{R}^n
 \] difeomorfizem.     
\end{definicija}

\begin{opomba}
 Ekvivalentnost atlasov je ekvivalenčna relacija, ekvivalenčni razred atlasa $\mathcal{U}$  označimo z $\left[ \mathcal{U} \right].$ 
\end{opomba}

\begin{definicija}
\label{def_gladka_struktura}
 Naj bo $M$ topološka mnogoterost in $\mathcal{U}$ gladek atlas na $M$. Potem je $\left[ \mathcal{U} \right]$  gladka struktura na $M$.    
\end{definicija}

\begin{opomba}
 Dejstvo, da lahko obstajajo kakšne netrivialne (eksotične strukture) na mnogoterostih, je zelo netrivialno. Iz Donaldsonovega in Freedmanovega izreka sledi, da ima $\mathbb{R}^{4}$ neštevno neskončno eksotičnih gladkih struktur. Vsi ostali $\mathbb{R}^n$ imajo zgolj svojo trivialno in nobene eksotične.  
\end{opomba}

\section{Gladke vložene ploskve}%
\label{sec:Gladke_vložene_ploskve}

V splošnem bi lahko mnogoterosti obravnavali kot abstraktne matematične strukture, ki ne prebivajo nujno v evklidskih prostorih. Pri uvodu v diferencialno geometrijo pa se bomo v glavnem ukvarjali z eno in dvodimenzionalnimi mnogoterostmi, vloženimi v prostor $\mathbb{R}^3$. 

\begin{definicija}
\label{def_gladka_vložena_ploskev}
Množica $X \subseteq  \mathbb{R}^3$ je gladka vložena ploskev, če za vsak $m \in  X$ obstaja krogla za $m$ $W \subseteq  \mathbb{R}^n$ in gladka funkcija $f : W \to  \mathbb{R}$, za katero velja \begin{enumerate}
	\item $X \cap W = f^{*}\left( \left\{ 0\right\}  \right)$
	\item $\left( Df \right)_w \neq 0 $ za vsak $w \in  X \cap  W$  

\end{enumerate}    
\end{definicija}

Vložena ploskev $X \subseteq  \mathbb{R}^n$ je tudi abstraktna mnogoterost. Poglejmo si, kako bi konstruirali atlas na $X$. Vzemimo točko $m \in  X$. Po definiciji vložene ploskve obstaja nivojnica $f: W \ni m \to \mathbb{R}$ in vemo, da $D_mf = \left( \frac{ \partial f }{ \partial x } , \frac{ \partial f }{ \partial y }  , \frac{ \partial f }{ \partial z }  \right)\left( m \right) \neq 0$. Zdaj se spomnimo izreka o implicitni funkciji. Naj bo $m = \left( x_0, y_0 , z_0 \right)$ in BŠS naj bo $\frac{ \partial f }{ \partial z }\left( m \right) \neq 0$. Torej obstaja gladka okolica $V \ni \left( x_0, y_0 \right) \subseteq  \mathbb{R}^2$  in gladka funkcija $g: V \to \mathbb{R}$, da velja $f\left( x, y, g\left( x,y \right) \right) = 0$  za vsak $\left( x,y \right) \in  V$. Po potrebi lahko množico $W$ zmanjšamo na $W_0 \subseteq  W$, da dobimo difeomorfizem \begin{align*}
	r: V &\longrightarrow W_0 \cap  X \\
	\left( x,y \right) &\longmapsto \left( x,y,g\left( x,y \right) \right)
\end{align*} z inverzom \begin{align*}
	\varphi: W_0 \cap  X &\longrightarrow V \\
	\left( x,y,z \right) &\longmapsto \left( x,y \right).
\end{align*}Ta inverz je v bistvu projekcija na prvi dve koordinati. Če definiramo $U = W_0 \cap  X$, postane par $\left( U, \varphi \right)$  karta na $X$.  

\subsection{Metrika na ploskvi}%

Če hočemo meriti razdalje med pari točk na gladki mnogoterosti, potrebujemo še dodatno strukturo -- metriko. Ta nam omogoča merjenje dolžin krivulj. Če si predstavljamo krivuljo $\gamma : \left( a,b \right) \to M$, je najbolj naravna definicija njene dolžine \[
\mathcal{L}\left( \gamma \right) = \int_{a}^{b} \lvert\lvert \dot{\gamma } \left( t \right) \rvert\rvert  \, dt. 
\]Znati moramo torej izračunati dolžino oziroma normo tangentnega vektorja. Najbolje je, če je ta norma porojena s skalarnim produktom, torej $\lvert\lvert x \rvert\rvert = \sqrt{\langle x,x \rangle } $.

Naj bo $\langle \cdot , \cdot  \rangle $  neki skalarni produkt na $\mathcal{V} = \mathbb{R}^n$ in naj bo $\left\{ v_1, \ldots , v_{n} \right\}$  baza za $\mathcal{V}$, ki ni nujno ortonormirana. Vzemimo vektorja $\vec{x}  = \sum_{i=1}^{n} a_{i}v_{i}$ in $\vec{y}  = \sum_{i=1}^{n} b_{i}v_{i}.$ Potem velja, da je skalarni produkt enak \[
\langle \vec{x} , \vec{y}  \rangle = \sum_{i,j = 1}^{n} a_{i}b_{j}\langle v_{i}, v_{j} \rangle = \begin{pmatrix}
	a_1 & \ldots & a_n
\end{pmatrix}
 \begin{pmatrix}
\langle a_1, a_1 \rangle & \dots & \langle a_1, a_n \rangle \\
\vdots & \ddots & \vdots \\
\langle a_n, a_1 \rangle & \dots & \langle a_1,a_n \rangle 
\end{pmatrix} \begin{pmatrix} b_1 \\ \vdots \\ b_n \end{pmatrix}
\] 
Iz simetričnosti skalarnega produkta ($\langle v_{i}, v_{j} \rangle  = \langle v_{j},  v_{i} \rangle $) sledi, da je zgornja matrika simetrična. Iz pozitivne definitnosti skalarnega produkta ($\langle v_{i}, v_{i} \rangle > 0$) pa sledi še pozitivna definitnost te matrike. 

\begin{opomba}
 Kvadratne matrike so lahko koordinatni zapisi linearnih preslikav iz $\mathbb{R}^n \to  \mathbb{R}^n$, lahko pa so tudi koordinatni zapisi skalarnih produktov. To je odvisno od tega, kako se matrike transformirajo pri prehodu v različno bazo.

 Naj bo $P$ poljubna preslikava med bazama,  $L_e$ linearna preslikava glede na bazo $\left\{ e_1, \ldots , e_{n}\right\}$, $L_f$ pa glede na bazo $\left\{ f_1, \ldots, f_{n}\right\}.$ Potem iz algebre 1 vemo, da je \[
 L_f = PL_eP^{-1}.
 \]Zdaj pa izpeljimo, kako se transformira matrika skalarnega produkta. Naj bosta $a_f = Pa_e$ in $b_f = Pb_e$. Potem dobimo iz enakosti \begin{align*}
     \langle a_f, b_f \rangle  &= \langle a_e, b_e \rangle  \\
     a_f^{T} A_f b_f &= a_e^{T} A_e b_e \\
	 a_e^{T} P^{T} A_f P b_e &= a_e^{T} A_e b_e, \forall a_e, b_e.
 \end{align*} Od tod sledi, da je $P^{T}A_fP = A_e$ oziroma zaradi ortogonalnosti $P$ ekvivalentno \[A_f = PA_eP^{T}.\]

 Torej transformacijska pravila določajo vrsto preslikave, podobno kot pri fiziki.
\end{opomba}

\end{document}

