\section{Gladke vložene ploskve}%
\label{sec:Gladke_vložene_ploskve}

V splošnem bi lahko mnogoterosti obravnavali kot abstraktne
matematične strukture, ki ne prebivajo nujno v evklidskih
prostorih. Pri uvodu v diferencialno geometrijo pa se bomo v
glavnem ukvarjali z eno in dvodimenzionalnimi mnogoterostmi,
vloženimi v prostor $\mathbb{R}^3$. 

\begin{definicija}
\label{def_gladka_vložena_ploskev} Množica
$X \subseteq \mathbb{R}^3$ je gladka vložena ploskev, če
za vsako točko $m \in  X$ obstaja krogla $W \ni m \subseteq
\mathbb{R}^3$ ter gladka funkcija $f : W \to \mathbb{R}$,
za katero velja 
\begin{enumerate} \item $X \cap W =
f^{*}\left( \left\{ 0\right\}  \right)$,
\item $D_wf \neq 0 $ za vsak $w \in X \cap  W$. 
\end{enumerate}    
\end{definicija}

Vložena ploskev $X \subseteq  \mathbb{R}^3$ je tudi abstraktna
mnogoterost. Poglejmo si, kako bi konstruirali atlas na $X$.
Vzemimo točko $m \in  X$. Po definiciji vložene ploskve
obstaja nivojnica $f: W \ni m \to \mathbb{R}$, za katero po drugi točki definicije vemo, da
$D_mf = \left( \frac{ \partial f }{ \partial x } , \frac{
\partial f }{ \partial y }  , \frac{ \partial f }{ \partial z
} \right)\left( m \right) \neq 0$. Zdaj se spomnimo izreka o
implicitni funkciji. Naj bo $m = \left( x_0, y_0 , z_0 \right)$ in
BŠS naj bo $\frac{ \partial f }{ \partial z }\left( m \right) \neq
0$. Torej obstaja gladka okolica $V \ni \left( x_0, y_0 \right)
\subseteq \mathbb{R}^2$  in gladka funkcija $g: V \to \mathbb{R}$,
da velja $f\left( x, y, g\left( x,y \right) \right) = 0$  za vsak
$\left( x,y \right) \in  V$. Po potrebi lahko množico $W$
zmanjšamo na $W_0 \subseteq  W$, da dobimo difeomorfizem
\begin{align*} r: V &\longrightarrow W_0 \cap  X \\ \left( x,y
\right) &\longmapsto \left( x,y,g\left( x,y \right) \right)
\end{align*} z inverzom \begin{align*} \varphi: W_0 \cap X
&\longrightarrow V \\ \left( x,y,z \right)
&\longmapsto \left( x,y \right). \end{align*}Ta inverz
je v bistvu projekcija na prvi dve koordinati. Če
definiramo $U = W_0 \cap  X$, postane par $\left( U,
\varphi \right)$ karta na $X$.  

\subsection{Metrika na ploskvi}%

Če hočemo meriti razdalje med pari točk na gladki
mnogoterosti, potrebujemo še dodatno strukturo --
metriko. Ta nam omogoča merjenje dolžin krivulj. Če si
predstavljamo krivuljo $\gamma : \left( a,b \right)
\to M$, je najbolj naravna definicija njene dolžine \begin{equation*}
\mathcal{L}\left( \gamma \right) = \int_{a}^{b}
\lvert\lvert \dot{\gamma } \left( t \right) \rvert\rvert
\, dt. \end{equation*}Znati moramo torej izračunati dolžino oziroma normo
tangentnega vektorja. Najbolje je, če je ta norma porojena s
skalarnim produktom, torej $\lvert\lvert x \rvert\rvert =
\sqrt{\langle x,x \rangle } $.

Naj bo $\langle \cdot , \cdot  \rangle $  neki -- ne nujno običajni -- skalarni
produkt na $\mathcal{V} = \mathbb{R}^n$ in naj bo $\left\{
v_1, \ldots , v_{n} \right\}$  baza za $\mathcal{V}$, ki ni
nujno ortonormirana. Vzemimo vektorja $\vec{x}  =
\sum_{i=1}^{n} a_{i}v_{i}$ in $\vec{y}  = \sum_{i=1}^{n}
b_{i}v_{i}.$ Potem velja, da je skalarni produkt enak \begin{equation*}
\langle \vec{x} , \vec{y} \rangle = \sum_{i,j = 1}^{n}
a_{i}b_{j}\langle v_{i}, v_{j} \rangle =
\begin{pmatrix}
a_1 & \ldots & a_n
\end{pmatrix}
\begin{pmatrix}
\langle v_1, v_1 \rangle & \dots & \langle v_1, v_n
\rangle \\ \vdots & \ddots & \vdots \\ \langle v_n,
v_1 \rangle & \dots & \langle v_n,v_n \rangle
\end{pmatrix}
\begin{pmatrix} b_1 \\ \vdots \\ b_n
\end{pmatrix}. \end{equation*}
Iz simetričnosti skalarnega produkta
($\langle v_{i}, v_{j} \rangle  = \langle v_{j},  v_{i}
\rangle $) sledi, da je zgornja matrika simetrična. Iz
pozitivne definitnosti skalarnega produkta ($\langle
v_{i}, v_{i} \rangle > 0$) pa sledi še pozitivna 
definitnost te matrike. 

\begin{opomba}
Kvadratne matrike so lahko koordinatni
zapisi linearnih preslikav iz $\mathbb{R}^n \to
\mathbb{R}^n$, lahko pa so tudi koordinatni zapisi
skalarnih produktov. To je odvisno od tega, kako se
matrike transformirajo pri prehodu v različno bazo.

Naj bo $\mathcal{L}: \mathbb{R}^n \to  \mathbb{R}^n$ linearna, $L_e$ matrika
linearne preslikave $\mathcal{L}$ glede na bazo $\left\{ e_1, \ldots
, e_{n}\right\}$, $L_f$ pa glede na bazo $\left\{ f_1,
\ldots, f_{n}\right\}.$ Naj bo $P$ prehodna preslikava
iz baze $\left\{ e_1, \ldots
, e_{n}\right\}$ v bazo $\left\{ f_1, \ldots
, f_{n}\right\}$. Potem iz algebre 1 vemo, da je \begin{equation*}
L_f = PL_eP^{-1}. \end{equation*}Zdaj pa izpeljimo, kako se
transformira matrika skalarnega produkta. Naj bosta
$a_f = Pa_e$ in $b_f = Pb_e$. Potem dobimo iz enakosti
\begin{align*} \langle a_f, b_f \rangle &= \langle
a_e, b_e \rangle \\ a_f^{T} A_f b_f &= a_e^{T} A_e
b_e \\ a_e^{T} P^{T} A_f P b_e &= a_e^{T} A_e b_e,
\forall a_e, b_e. \end{align*} Od tod sledi, da je
$P^{T}A_fP = A_e$ oziroma zaradi ortogonalnosti
$P$ ekvivalentno \begin{equation*}A_f = PA_eP^{T}.\end{equation*}

Torej transformacijska pravila določajo vrsto
preslikave, podobno kot pri fiziki. 
\end{opomba}

Preden se lotimo definicije tangentne ravnine, se spomnimo naslednje
definicije. 

\begin{definicija}
\label{def_rang_preslikave}
 Naj bo preslikava $F: W \subseteq  \mathbb{R}^n \to  \mathbb{R}^m$ odvedljiva.
 Rang preslikave $F$ v točki $w \in  W$ je enak rangu matrike $D_wF$. Pravimo,
 da ima $F$ v točki $w \in  W$ maksimalen rang, če ima matrika $D_wF$
 maksimalen rang.
\end{definicija}


\begin{definicija} 
	\label{def_tangentna_ravnina}
	Naj bo $X \subseteq \mathbb{R}^3$ vložena ploskev in točka $m
\in X$. Tangentna ravnina $T_mX$ je množica tangent
vseh krivulj v $X$, ki v času $t = 0$ gredo skozi $m$.
\begin{equation*} T_mX = \left\{ \dot{\gamma}\left( 0 \right) \mid
\gamma : \left(  - \varepsilon, \varepsilon \right)
\to  X \subseteq  \mathbb{R}^3 \text{ krivulja, } \gamma\left(
0 \right) = m\right\} \end{equation*}
\end{definicija}


\begin{trditev}
\label{trd_tangentna_ravnina_je_dvodimenzionalen_vektorski_prostor}
$T_mX$ je dvodimenzionalen realen vektorski podprostor v
$\mathbb{R}^3$. 
\end{trditev} 
{\em Dokaz:\/} Naj bo $r : V
\subseteq  \mathbb{R}^2 \to X \subseteq \mathbb{R}^3$ neka regularna
parametrizacija ploskve $X$ (to pomeni, da mora biti rang preslikave
$r$ maksimalen, torej konstantno enak 2) v okolici točke  $m \in  X$.  Označimo za vsak $\left( u, v \right) \in  V \subseteq \mathbb{R}^2$ \begin{equation*} r\left( u,v \right) = 
\begin{pmatrix} x\left( u,v
\right) \\ y\left( u,v \right) \\ z\left( u,v \right)
\end{pmatrix} . \end{equation*}
Naj bo $m = r\left( u_0, v_0 \right).$ Trdimo, da
je $ T_mX = \operatorname{im} \left( D_{(u_0, v_0)}r \right) $.
Najprej dokažimo inkluzijo $T_mX \subseteq \operatorname{im} \left(
D_{(u_0, v_0)}r \right)$. 
Naj bo $\gamma: \left( -\varepsilon,
\varepsilon \right) \to X \subseteq \mathbb{R}^2,$ $\gamma \left( 0
\right) = m$ poljubna krivulja. Direktno
po definiciji tangentne krivulje sledi $\dot{\gamma}\left( 0 \right)
\in T_mX$. 
Dokazati moramo $\dot{\gamma}(0) \in \operatorname{im} \left(D_{\left( u_0, v_0 \right)} r \right)$. Naj bo
$\beta \left( t \right) : \left( -\varepsilon, \varepsilon \right)\to
V$ podana z $\beta\left( t \right) = r^{-1}\left( \gamma\left( t
\right) \right)$. Ker je praslika preslikave $\beta$ vsebovana v
$\mathbb{R}^2,$ obstajata funkciji $u\left( t \right), v\left( t
\right)$, da je $\beta\left( t \right) = \left( u(t), v(t) \right).$
Pri tem velja, da je $\beta(0) = (u(0), v(0)) = (u_0, v_0).$ Vidimo,
da je $\gamma(t) = r(u(t), v(t)).$ Po verižnem pravilu za odvajanje
imamo \begin{equation*} \dot{\gamma }(0)= \frac{d}{dt} \big|_{t = 0} \gamma(t) =
\left( D_{ (u_0, v_0)}r \right) \dot{\beta} (0).\end{equation*}Torej je
$\dot{\gamma} (0) \in \operatorname{im} \left( D_{(u_0,v_0)}r
\right).$ 

Nato dokažimo še obratno inkluzijo $T_mX \supseteq
\operatorname{im} \left( D_{(u_0, v_0)}r \right)$. Vzemimo poljuben
vektor $w \in \mathbb{R}^2$ in naj bo $v = \left( D_{(u_0, v_0)} r \right) w$. Potrebujemo krivuljo $\gamma(t) : (-\varepsilon, \varepsilon) \to  X,
\gamma(0) = m,$ za katero bo veljalo $\dot{\gamma} (0) = v.$
Oglejmo si \begin{equation*} \left( D_{(u_0, v_0)}r \right)\dot{\beta} (0) =
\frac{d}{dt} \big|_{t = 0} r(\beta(t)). \end{equation*}  Trdimo, da za $\gamma(t) =
r(\beta(t))$  velja $\dot{\gamma}(0) \in  T_mX.$ To
je res, saj je $\gamma(t): (-\varepsilon,
\varepsilon)\to  X \subseteq \mathbb{R}^3$, hkrati pa tudi $\gamma(0) = r(\beta(0) )= r(u_0, v_0) = m$. Torej velja,
da je $T_mX = \operatorname{im}(D_{(u_0,v_0)}r)$. Ker smo zahtevali,
da je parametrizacija regularna, je matrika $D_{(u_0,v_0)}r$ ranga 2,
torej je  $T_mX$ dvodimenzionalen vektorski prostor.
\qed

\begin{opomba}
TODO
Če se ne motim, je v drugi polovici zgornjega dokaza napaka. Morali bi nastaviti funkcijo $\beta(t): (-\varepsilon, \varepsilon) \to V$,
s predpisom $\beta(t) = wt + (u_0, v_0)$. Potem definiramo krivuljo $\gamma(t) = r(\beta(t))$, za katero res veljata zahtevani lastnosti.
\end{opomba}



\begin{opomba}
 Tangentna ravnina je pravi vektorski prostor in ne afin kot
 recimo pri analizi 2a.
\end{opomba}

Do nadaljnjega nas bodo zanimale lokalne lastnosti ploskev, zato bomo
delali v glavnem s ploskvami, ki jih lahko pokrijemo z eno samo karto
oziroma z eno samo parametrizacijo.

\begin{definicija}
\label{def_metrika_na_ploskvi}
 Metrika na ploskvi $X \subseteq  \mathbb{R}^3$, opremljeni s
 parametrizacijo $r: V \subseteq  \mathbb{R}^2 \to  X \subseteq
 \mathbb{R}^3$, je preslikava \begin{align*}
 	g: X &\longrightarrow M_2(\mathbb{R}), \\
 	m &\longmapsto 
	\begin{pmatrix}
		g_{11}(m) & g_{12}(m) \\\
		g_{21}(m) & g_{22}(m) \\
	\end{pmatrix} = \begin{pmatrix}
		\langle r_u, r_u \rangle_g (m) & \langle r_u, r_v \rangle_g(m) \\\
		\langle r_u, r_v \rangle_g(m) & \langle r_v, r_v \rangle_g(m) \\
	\end{pmatrix},
 \end{align*}
 kjer je za vsak $m \in  X$ matrika $g(m)$ simetrična in pozitivno definitna. To lahko povemo s pogojema
 $\det g(m) > 0$ in $g_{11}(m) >0.$

\end{definicija}
\begin{opomba}
 Za kakšno drugo parametrizacijo ploskve $X$ bi dobili druge koeficiente
 matrike.
\end{opomba}
 Naj bo $\gamma: [a,b] \to  X$ krivulja. Njeno parametrizacijo $r$
lahko napišemo v obliki  $\gamma(t) = r(u(t), v(t))$ za primerne
funkcije $u,v : [a,b] \to \mathbb{R}$, $\beta(t) = (u(t), v(t))$,
$\gamma(t) = r(\beta(t)).$ V koordinatah lahko zapišemo 

\begin{equation*} \gamma(t) =  \begin{pmatrix}x(u(t), v(t))\\ y(u(t), v(t)) \\
z(u(t), v(t)) \end{pmatrix}, \end{equation*}
\begin{equation*} \dot{\gamma}(t_0) = \frac{d}{dt} \big|_{t = t_0} r(\beta(t)) =
(D_{(u_0, v_0)}r)(\dot{\beta}(t_0)) = 
\begin{pmatrix}
	\frac{ \partial x }{ \partial u }  & \frac{ \partial x }{ \partial v }  \\
	\frac{ \partial y }{ \partial u }  & \frac{ \partial y }{ \partial
	v }  \\
	\frac{ \partial z }{ \partial u }  & \frac{ \partial z }{ \partial v }  \\

\end{pmatrix}_{(u_0, v_0)}
\begin{pmatrix}
	\dot{u} \\ \dot{v}
\end{pmatrix}
 = r_u(u_0 , v_0)\dot{u}(t_0) + r_v(u_0 , v_0)\dot{v}(t_0). \end{equation*}
 To je razvoj vektorja $\dot{\gamma}(t_0)$ po bazi $\left\{ r_u(u_0 ,
v_0), r_v(u_0 , v_0) \right\}$ prostora $T_{\gamma(t_0)}X$, ki pa ni
nujno ortogonalna. Pravzaprav je ortogonalna le v precej posebnih
primerih. 

\begin{definicija}
\label{def_dolzina_krivulje}
Naj bo $X \subseteq \mathbb{R}^3$ ploskev ter $r: V \subseteq \mathbb{R}^2 \to X$ njena parametrizacija. Naj bo $\gamma: [a,b] \to  X$ krivulja, ki ji pripadata funkciji $u, v: [a,b] \to  \mathbb{R}$, tako da velja \begin{equation*}
\gamma(t) = r(u(t), v(t)).
\end{equation*}  
   Dolžina krivulje $\gamma$ glede na metriko $g$ je v parametrizaciji $r$ podana s formulo 
\begin{equation*} \mathcal{L}_g(\gamma) = \int_{a}^{b} \sqrt{ 
\begin{pmatrix}
    \dot{u}(t) & \dot{v}(t) \\
\end{pmatrix}
\begin{pmatrix}
	g_{11} & g_{12}  \\
	g_{21} & g_{22} \\
\end{pmatrix}
\begin{pmatrix}
	\dot{u}(t) \\
	\dot{v}(t) \\
\end{pmatrix}
}  \, dt  \end{equation*}
\end{definicija}


Ustrezni skalarni produkti na ravnini $T_mX$ so glede na
parametrizacijo $r$ podani s predpisi $\left<r_u, r_u \right>_g  =
g_{11}$, $\left<r_u, r_v \right>_g  =
g_{12}$, $\left<r_v, r_v \right>_g  =
g_{22}$. Naj bo sedaj ambientni prostor $\mathbb{R}^3$ opremljen s
fiksnim evklidskim skalarnim produktom, in koeficiente $g_{ij}$
poračunamo z njim (na enak način kot prej). Pri tem uporabimo
naslednje standardne oznake: 
\begin{equation*} E(u,v) = \left<r_u(u,v), r_u(u,v) \right>,   \,\,\,   F(u,v) =  \left<r_u(u,v),
r_v(u,v) \right>, \,\,\,     G(u,v) =  \left<r_v(u,v), r_v(u,v)
\right>.  \end{equation*}
Včasih tudi zlorabimo notacijo 
\begin{equation*} E(m) = E(r(u,v)) = E(u,v). \end{equation*}

\begin{definicija}
\label{def_prva_fundamentalna_forma}
 Metrika na $X \subseteq  \mathbb{R}^2$, ki je glede na parametrizacijo $r: V \to  X
\subseteq  \mathbb{R}^3$ podana z matrično funkcijo \begin{align*}
 	g_f: V &\longrightarrow M_2(\mathbb{R}) \\
	(u,v) &\longmapsto 
\begin{pmatrix}
	E(u,v) & F(u,v) \\
	F(u,v) & G(u,v) \\
\end{pmatrix},
\end{align*}
se imenuje prva fundamentalna forma ploskve.
\end{definicija}

\begin{opomba}
 Dolžina krivulje glede na prvo fundamentalno
 formo $\mathcal{L}_{g_f}(\gamma)$ sovpada z običajno dolžino krivulje $\mathcal{L}(\gamma)$: \begin{align*}
    \mathcal{L}(\gamma)  = \int_{a}^{b} \lvert\lvert \dot{\gamma}(t) \rvert\rvert  \, dt  &=
	 \int_{a}^{b} \sqrt{\left<\dot{u}r_u + \dot{v}r_v, \dot{u}r_u + \dot{v}r_v \right> }  \, dt \\
        &= \int_{a}^{b} \sqrt{ 
\begin{pmatrix}
	\dot{u}(t) & \dot{v}(t) \\
\end{pmatrix}
\begin{pmatrix}
	\left<r_u, r_u \right> & \left<r_u, r_v \right>  \\
	\left< r_u, r_v \right>	  & \left<r_v, r_v \right> \\
\end{pmatrix}_{(u(t), v(t))}
\begin{pmatrix}
	\dot{u}(t) \\
	\dot{v}(t) \\
\end{pmatrix}
}  \, dt  \\
 &=  \int_{a}^{b} \sqrt{ 
\begin{pmatrix}
	\dot{u}(t) & \dot{v}(t) \\
\end{pmatrix}
\begin{pmatrix}
	E & F \\
	F & G \\
\end{pmatrix}_{(u(t), v(t))}
\begin{pmatrix}
	\dot{u}(t) \\
	\dot{v}(t) \\
\end{pmatrix}
}  \, dt  \\
&= \mathcal{L}_{g_f}(\gamma).
 \end{align*}
\end{opomba}

Izomorfizmi v diferencialni geometriji so izometrije, katerih
definicija pa je nekoliko drugačna, kot bi morda pričakovali.

\begin{definicija}
\label{def_izometrija}
 Preslikava $f: X \to \tilde{X}$ nad dvema ploskvama $X, \tilde{X} \subseteq
 \mathbb{R}^3$ je izometrija, če za vsako krivuljo $\gamma: \left[ a,b
 \right] \to  X$ velja enakost med dolžinama 
 \begin{equation*} \mathcal{L}_X(\gamma) = \mathcal{L}_{\tilde{X}}(f(\gamma)). \end{equation*}
\end{definicija}

\begin{opomba}
 Izometrije med ploskvama porodijo izometrije v običajnem metričnem
 smislu.
\end{opomba}

\begin{primer}
 Prva fundamentalna forma na sferi glede na sferične koordinate. Če
 odvzamemo iz $S^2$ en poldnevnik, jo lahko parametriziramo s
 sferičnimi koordinatami: 
 \begin{equation*} r(u,v): V = \left( -\pi, \pi \right) \times \left(
 0, \pi  \right) \subseteq  \mathbb{R}^2 \to  \mathbb{R}^3
  \end{equation*} s predpisom \begin{equation*} r(u,v) = 
\begin{pmatrix}\cos u \cos v \\ \cos u \sin v \\ \sin u
\end{pmatrix}.\end{equation*}
Potem dobimo parcialna odvoda 
\begin{equation*}   r_u(u,v) = \begin{pmatrix}-\sin u \cos v \\ -\sin u \sin v \\ \cos u
\end{pmatrix}, \,\,  r_v(u,v) = 
\begin{pmatrix}-\cos u \sin  v \\ \cos u \cos v \\ 0
\end{pmatrix}.\end{equation*}
Od tod sledi 
\begin{align*}
    E &= \langle r_u, r_u \rangle = 1, \\
    F &= \langle r_u, r_v \rangle = 0, \\
	G &= \langle r_v, r_v \rangle = \cos^2 u,
\end{align*} kar lahko zapišemo v obliki 
\begin{equation*} 
\begin{pmatrix}
	E & F \\
	F & G \\
\end{pmatrix}
=
\begin{pmatrix}
	1 & 0 \\
	0 & \cos^2 u \\
\end{pmatrix}
.
\end{equation*}
\end{primer}

\begin{primer}
 Naj bo $X$ rotacijska ploskev, ki jo dobimo, če krivuljo $x =  f(z)$
 zavrtimo okoli osi $z$: 
 \begin{equation*} r: \left( u,v \right) \mapsto  
\begin{pmatrix}
	f(u) \cos v \\
	f(u) \sin v \\
	u \\
\end{pmatrix}
.\end{equation*}Potem imamo odvoda 
\begin{equation*}   r_u(u,v) = \begin{pmatrix} f'(u) \cos v \\ f'(u) \sin v \\ 1
\end{pmatrix}, \,\,  r_v(u,v) = 
\begin{pmatrix}-f(u) \sin  v \\ f(u) \cos v \\ 0
\end{pmatrix}.\end{equation*}
Od tod sledi 
\begin{align*}
    E &= \langle r_u, r_u \rangle = 1 + f'(u)^2, \\
    F &= \langle r_u, r_v \rangle = 0, \\
	G &= \langle r_v, r_v \rangle = f(u)^2,
\end{align*}
kar lahko zapišemo v obliki 
\begin{equation*} 
\begin{pmatrix}
	E & F \\
	F & G \\
\end{pmatrix}
=
\begin{pmatrix}
	1 + f'(u)^2 & 0 \\
	0 & f(u)^2 \\
\end{pmatrix}
.
\end{equation*}
\end{primer}

Naslednji izrek nam pove povezavo med prvo fundamentalno formo in izometričnostjo ploskev.

\begin{izrek}
\label{izr_izometricnost_in_prva_forma}
Ploskvi $X$ in $\tilde{X}$ sta izometrični natanko tedaj, ko obstaja par parametrizacij
$r : V \to X$ in $\tilde{r}: V \to  \tilde{X}$, da za pripadajoči
fundamentalni formi velja 
\begin{equation*} 
\begin{pmatrix}
	E & F \\
	F & G \\
\end{pmatrix}_{(u,v)}
=
\begin{pmatrix}
	\tilde{E}  & \tilde{F}  \\
	\tilde{F}  & \tilde{G}  \\
\end{pmatrix}_{(u,v)}
.\end{equation*}
\end{izrek}
{\em Dokaz:\/}
Pokažimo najprej obrat $(\impliedby)$. Recimo, da obstajata parametrizaciji $r: V
\to  X$ in $\tilde{r}: V \to  \tilde{X} $, da velja 
\begin{equation*} 
  \begin{pmatrix}
    E & F \\
    F & G \\
  \end{pmatrix}_{(u,v)}
  =
\begin{pmatrix}
	\tilde{E}  & \tilde{F}  \\
	\tilde{F}  & \tilde{G}  \\
\end{pmatrix}_{(u,v)}
.\end{equation*} 
% TODO ali je res smisleno, da je indeks _r? Verjetno je bolj smiselno, da je _X, ker je dolžina itak neodvisna od parametrizacije  
Naj bo $f: X \to  \tilde{X}$ podana s predpisom $f = \tilde{r} \circ  r^{-1}$. S primerjavo dolžini $\mathcal{L}_r(\gamma)$ ter
$\mathcal{L}_{\tilde{r}}(f(\gamma))$ hočemo preveriti, da je $f$
izometrija . Naj bo $\gamma: [a,b] \to  X$ poljubna krivulja. Potem obstaja krivulja $\beta:
[\alpha,\beta] \to  V$ za katero velja $\gamma(t) = r(\beta(t))$.
Potem za $\tilde{\gamma}(t) := f(\gamma(t))$ velja 
\begin{equation*}\tilde{\gamma}(t) =  f(\gamma(t)) = f(r(\beta(t))) = \tilde{r}(\beta(t)).\end{equation*}
Ker je $\beta(t)$ ravninska krivulja, velja 
\begin{align*}
    \gamma(t) &= r(\beta(t)) = r(u(t), v(t)), \\
    \tilde{\gamma}(t) &= \tilde{r} (\beta(t)) = \tilde{r} (u(t), v(t)).   
\end{align*}
Torej imamo enačbi 
\begin{equation*} \mathcal{L}_r(\gamma) = \int_{a}^{b} \sqrt{E(u,v)\dot{u}^2 +
2F(u,v)\dot{u}\dot{v} + G(u,v)\dot{v}^2} \, dt, \end{equation*}
\begin{equation*} \mathcal{L}_{\tilde{r}}(\tilde{\gamma}) = \int_{a}^{b} \sqrt{\tilde{E} (u,v)\dot{u}^2 +
2\tilde{F}(u,v)\dot{u}\dot{v} + \tilde{G}(u,v)\dot{v}^2} \, dt.  \end{equation*}
Ker so posamezni sumandi po predpostavki enaki, sta dolžini enaki, torej je $f =
\tilde{r} \circ r^{-1}$ res izometrija.

Zdaj dokažimo še $(\implies)$. Denimo, da imamo med ploskvama $X$ in $\tilde{X}$
izomerijo $f: X \to  \tilde{X}.$ Naj bo $\gamma(t): (0, \varepsilon) \to X$ krivulja s predpisom $\gamma(t) = r(\beta_1(t)) =
r(u_0 + t, v_0)$. Pri tem je točka $(u_0, v_0) \in V$ poljubna in določa točko $p = r(u_0, v_0) \in  X$. Zdaj
izračunamo 
\begin{equation*} 
\mathcal{L}_r(\gamma(t)) = \int_{0}^{\varepsilon} \lvert\lvert
\dot{\gamma}(t) \rvert\rvert   \, dt = \int_{0}^{\varepsilon} \sqrt{\langle
\dot{\gamma}(t), \dot{\gamma}(t)  \rangle} \, dt =
\int_{0}^{\varepsilon} \sqrt{\langle r_u(u_0 + t, v_0), r_u(u_0 + t, v_0) \rangle}   \, dt =
\int_{0}^{\varepsilon}\sqrt{  E(u_0 + t, v_0) }  \, dt. 
\end{equation*}
Sedaj si oglejmo $f(\gamma (t)) = \tilde{r}(\beta_1(t)) = \tilde{r}(u_0
+ t, v_0)$. Potem imamo:

\begin{align*}
  \mathcal{L}_{\tilde{r}}(f(\gamma(t))) &= \int_{0}^{\varepsilon} \lvert\lvert
  \dot{\tilde{\gamma}}(t) \rvert\rvert   \, dt  \\ &= 
  \int_{0}^{\varepsilon} \sqrt{\langle\dot{\tilde{\gamma}}(t), \dot{\tilde{\gamma}}(t)  \rangle} \, dt  \\ &=
  \int_{0}^{\varepsilon} \sqrt{\langle \tilde{r}_u(u_0 + t, v_0), \tilde{r}_u(u_0 + t, v_0) \rangle}   \, dt \\ &= 
  \int_{0}^{\varepsilon} \sqrt{\tilde{E}(u_0 + t, v_0)}   \, dt. 
\end{align*}
Po predpostavki o izometričnosti velja 
\begin{align*}
    \mathcal{L}_r(\gamma) &= \mathcal{L}_{\tilde{r}}(f(\gamma))  \\
    \implies \int_{0}^{\varepsilon} \sqrt{E(u_0 + t, v_0)}   \, dt  &= \int_{0}^{\varepsilon} \sqrt{\tilde{E}(u_0 + t, v_0)} \, dt.
\end{align*}
Po izreku o povprečni vrednosti obstajata vrednosti $\hat{t}, \hat{\hat{t}} \in  (0, \varepsilon)$, da velja
\begin{equation*} E(u_0 + \hat{t} , v_0) \varepsilon = \tilde{E}(u_0 + \hat{\hat{t}} , v_0) \varepsilon\end{equation*}in če pošljemo $\varepsilon \to  0$,
zaradi zveznosti funkcij dobimo 
\begin{equation*} E(u_0, v_0) = \tilde{E}(u_0, v_0). \end{equation*}
Še lažje ta rezultat dobimo tako, da na obeh straneh odvajamo po $\varepsilon$ in vstavimo $t = 0$.

Če zdaj vzamemo $\beta_2(t) = (u_0, v_0 + t)$, dobimo po enakem postopku kot prej 
\begin{equation*} G(u_0, v_0) = \tilde{G}(u_0, v_0). \end{equation*}
Nato vzamemo $\beta_3(t) = (u_0 + t, v_0 + t)$ in imamo
\begin{align*}
  \mathcal{L}_r(\gamma) &= \int_{0}^{\varepsilon} \sqrt{E(u_0 + t, v_0 + t) + 2F(u_0 + t, v_0 + t) + G(u_0 + t, v_0 + t)}   \, dt  \\
  = \mathcal{L}_{\tilde{r}}(\tilde{\gamma}) &= \int_{0}^{\varepsilon} \sqrt{\tilde{E}(u_0 + t, v_0 + t) + 2\tilde{F}(u_0 + t, v_0 + t) + \tilde{G}(u_0 + t, v_0 + t)}   \, dt. 
\end{align*}

Če zdaj zopet odvajamo po $\varepsilon$ in vstavimo $t = 0$, dobimo enakost integrandov v točki $(u_0, v_0)$, od koder sledi še zadnja zahteva
\begin{equation*} F(u_0, v_0) = \tilde{F}(u_0, v_0). \end{equation*}
\qed

\begin{primer}
 Ali je stožec brez ene tvorilke izometričen kosu ravnine? Naj bo podan 
 \begin{equation*} S = \left\{ (x,y,z)  \middle|\, x^2 + y^2 + z^2 = 1 \right\}.\end{equation*}
 Parametriziramo ga z
 \begin{equation*} r(u,v) = (u \cos v, u \sin v, u) \end{equation*}in po znanem postopku dobimo 
 \begin{equation*} \begin{pmatrix}
  E & F\\
  F & G\\
 \end{pmatrix}_{(u,v)} 
 = \begin{pmatrix}
  2 & 0\\
  0 & u^2\\
 \end{pmatrix}.\end{equation*}
Pričakujemo, da bo $S$ izometričen nekemu krožnemu izseku, ki ga parametriziramo z 
\begin{equation*} \tilde{r}(u,v) = (\alpha u \cos(\beta v), \alpha u \sin(\beta v), 0).\end{equation*}Tako dobimo sistem enačb 
\begin{equation*} \begin{pmatrix}
  \tilde{E} & \tilde{F}\\
  \tilde{F} & \tilde{G}\\
 \end{pmatrix}_{(u,v)} 
 = \begin{pmatrix}
  \alpha^2 & 0\\
  0 & \alpha^2 \beta^2 u^2\\
 \end{pmatrix}.\end{equation*}
Pogoj  \begin{equation*} \begin{pmatrix}
  E & F\\
  F & G\\
 \end{pmatrix}_{(u,v)}
 = \begin{pmatrix}
  \tilde{E} & \tilde{F}\\
  \tilde{F} & \tilde{G}\\
 \end{pmatrix}_{(u,v)}\end{equation*}je izpolnjen pri $\alpha = \sqrt{2},\, \beta = \frac{1}{\sqrt{2}}$ (če bi zamenjali predznake $\alpha$ in $\beta$, bi dobili drugačne parametrizacije).
 Torej je stožec $S$ izometričen krožnemu izseku s parametrizaijo  
 \begin{equation*} \tilde{r}(u,v) = \left(\sqrt{2}  u \cos(\frac{1}{\sqrt{2}} v), \sqrt{2}  u \sin(\frac{1}{\sqrt{2}} v), 0 \right).\end{equation*}
\end{primer}

Na tej točki se pojavi naravno vprašanje: ali znamo poiskati vse ploskve v $\mathbb{R}^3$, ki so izomertične ravnini?
Izkaže se, da znamo, pred izrekom pa navedimo še definicijo tangentno premonosne ploskve.
\begin{definicija}
  \label{def_tangentno_premonsna_ploskev}
   Naj bo $\gamma: [a,b] \to  \mathbb{R}^3$ prostorska krivulja, parametriziarana z naravnim parametrom. Tangentna premonosna ploskev,
   podana s krivuljo $\gamma$, je del prostora $\mathbb{R}^3$, ki ga opiše tangenta na $\gamma(t)$ na intervalu $t \in  [a,b]$. Parametriziramo jo lahko s predpisom \begin{equation*}
   r(t, u) = \gamma(t) + u\dot{\gamma}(t).
   \end{equation*}  
     
  \end{definicija}
\begin{izrek}
\label{izr_izometricnost_ploskev_ravnini}
Naj bo ploskev $X$ izometrična kakšnemu kosu ravnine. Potem je $X$ bodisi stožec, valj, ali kakšna tangentna premonosna ploskev.
\end{izrek}



Če je $\gamma = \gamma(u)$ naravna parametrizacija, potem je smiselna parametrizacija tangentno premonosne ploskve $X$
podana z 
\begin{equation*} r(u,v) = \gamma(u) + v \dot{\gamma}(u).\end{equation*}Ker je $\gamma$ naravna parametrizacija, velja \begin{equation*} \lvert\lvert \dot{\gamma}(u) \rvert\rvert = \langle \dot{\gamma}(u), \dot{\gamma}(u) \rangle  = 1, \end{equation*}in z odvajanjem zveze dobimo 
\begin{equation*} \langle \ddot{\gamma}(u) , \dot{\gamma}(u) \rangle + \langle \dot{\gamma}(u) , \ddot{\gamma}(u) \rangle = 0.  \end{equation*}
Iz simetričnosti skalarnega produkta sledi 
\begin{equation*} 2 \langle \ddot{\gamma}(u) , \dot{\gamma}(u) \rangle = 0 \implies \langle \ddot{\gamma}(u) , \dot{\gamma}(u) \rangle = 0. \end{equation*}
Torej je pospešek pri naravni parametrizaciji vedno pravokoten na hitrost. To lahko opazimo, če se v avtu peljemo s konstantno hitrostjo. Pospešek bomo čutili
samo v ovinkih in to pravokotno glede na smer vožnje.

\begin{definicija}
\label{def_fleksijska_ukrivljenost}
 Fleksijska ukrivljenost naravno parametrizirane krivulje $\gamma(u)$ je podana s 
 \begin{equation*} \kappa(u) = \sqrt{\langle \ddot{\gamma}(u), \ddot{\gamma}(u) \rangle } = \lvert\lvert \ddot{\gamma}(u) \rvert\rvert. \end{equation*}
\end{definicija}

Izračunajmo prvo fundamentalno formo tangentno premonosne ploskve. Ker velja zveza
\begin{equation*} r(u,v) = \gamma(u) + v \dot{\gamma}(u),\end{equation*}takoj dobimo 
\begin{align*}
    r_u(u,v) &= \dot{\gamma}(u) + v \ddot{\gamma}(u), \\
    r_v(u,v) &= \dot{\gamma}(u). 
\end{align*}
Če od tod po znanem postopku poračunamo koeficiente prve fundamentalne forme (pri čemer upoštevamo, da je pospešek pravokoten na hitrost, torej $ \langle \ddot{\gamma}(u) , \dot{\gamma}(u) \rangle = 0$, ter naravnost parametrizacije, torej $ \langle \dot{\gamma}(u) , \dot{\gamma}(u) \rangle = 1$), dobimo 
\begin{equation*} \begin{pmatrix}
E & F \\
F & G
\end{pmatrix}_{(u,v)} = \begin{pmatrix}
  \langle \dot{\gamma}(u) + v \ddot{\gamma}(u), \dot{\gamma}(u) + v \ddot{\gamma}(u) \rangle  & \langle \dot{\gamma}(u) + v \ddot{\gamma}(u), \dot{\gamma}(u) \rangle \\
  \langle \dot{\gamma}(u) + v \ddot{\gamma}(u), \dot{\gamma}(u) \rangle & \langle \dot{\gamma}(u), \dot{\gamma}(u) \rangle
  \end{pmatrix}= \begin{pmatrix}
  1 + v^2 \kappa^2(u) & 1\\
  1 & 1\\
\end{pmatrix}.\end{equation*}

Torej vidimo, da je matrika prve fundamentalne forme tangentno premonosne ploskve odvisna samo od fleksijske ukrivljenosti.

\begin{trditev}
\label{trd_obstoj_krivulje_glede_na_funkcijo_fleksijske_ukrivljenosti}
 Naj bo podana poljubna (zvezna?) funkcija $\tilde{\kappa}: [\alpha, \beta] \to  \mathbb{R}$. Potem obstaja naravno parametrizirana ravninska krivulja $\gamma: [\alpha, \beta] \to  \mathbb{R}^2$, katere fleksijska ukrivljenost v točki
  $\gamma(u)$ je enaka $\kappa(u) = \tilde{\kappa}(u)$.
\end{trditev}

{\em Dokaz:\/}
 Ker je krivulja $\gamma$ ravninska, lahko zapišemo 
 \begin{align*}
     \gamma(u) &= (x(u), y(u)), \\
     \dot{\gamma}(u) &= (\dot{x}(u), \dot{y}(u)), \\ 
     \ddot{\gamma}(u) &= (\ddot{x}(u), \ddot{y}(u)).
 \end{align*}
 Ker je parametrizacija naravna, imamo še sistem enačb
 \begin{align*}
  \lvert\lvert \dot{\gamma}(u) \rvert\rvert &= 1,   \\
  \lvert\lvert \ddot{\gamma}(u) \rvert\rvert &= \kappa(u) ,   \\
    \langle \ddot{\gamma}(u), \dot{\gamma}(u) \rangle   &= 0.
 \end{align*}
 Enotski vektor, pravokoten na $\dot{\gamma}(u) = (\dot{x}(u), \dot{y}(u))$, je vektor $(\dot{y}(u), -\dot{x}(u))$. Ta vektor je vzporeden vektorju pospeška,
 torej bo za neko funkcijo $k : [\alpha, \beta]  \to \mathbb{R}$ veljalo 
 \begin{equation*} (\ddot{x}(u), \ddot{y}(u)) = k(u) (\dot{y}(u), -\dot{x}(u)). \end{equation*}Če obe strani enačbe normiramo, iz prejšnega sistema enačb vidimo, da mora priti natanko 
 \begin{equation*} (\ddot{x}(u), \ddot{y}(u)) = \kappa(u) (\dot{y}(u), -\dot{x}(u)).\end{equation*}Zanima nas, ali obstaja krivulja, za katero je $\kappa = \tilde{\kappa}$, torej z drugimi besedami, ali je rešljiv sistem navadnih diferencialnih enačb 
 \begin{align*}
     \ddot{x}(u) &= \tilde{\kappa}(u) \dot{y}(u), \\
     \ddot{y}(u) &= -\tilde{\kappa}(u) \dot{x}(u).
 \end{align*}
Pri analizi 3 smo dokazali eksistenčni izrek za obstoj rešitev tega sistema. Torej obstaja
naravno parametrizirana krivulja $\gamma(u) = (x(u), y(u))$, za katero za vsak $u \in [\alpha, \beta]$ velja $\lvert\lvert \ddot{\gamma}(u) \rvert\rvert = \tilde{\kappa}(u)$.

\qed

\begin{opomba}
 Pri določenih $(u,v)$ nam krivulja $\gamma(u)$ podaja tangentno premonosno ploskev. Ta ploskev je del ravnine, v kateri
 leži krivulja krivulja $\gamma(u)$. Po izreku \href{izr_izometricnost_ploskev_ravnini} je ta ploskev izometrična nekemu kosu ravnine.
\end{opomba}

S tem razmislekom smo dokazali izrek.

\begin{izrek}
\label{izr_tangentno_premonosna_ploskev_izometricna_kosu_ravnine}
  Vsaka tangentno premonosna ploskev je izometrična kosu ravnine.
\end{izrek}

\begin{definicija}
\label{def_ploscina_plosvke}
  Ploščina ploskve $X$, regularno parametrizirane z $r: V \subseteq  \mathbb{R}^2 \to  X \subseteq  \mathbb{R}^3$ je podana z 
  \begin{equation*} A(X) = \int_{V} \lvert\lvert r_u \times  r_v \rvert\rvert   \, du \, dv.\end{equation*} 
\end{definicija}

\begin{opomba}
 Da je ta definicija dobra, moramo še preveriti.
\end{opomba}

\begin{opomba}
 Ker velja zveza
 \begin{equation*} \lvert\lvert r_u \times  r_v \rvert\rvert^2  = \langle r_u, r_u \rangle \langle r_v , r_v \rangle  - \langle r_u, r_v \rangle^2 = EG - F^2, \end{equation*}
lahko ploščino izrazimo tudi kot 
 \begin{equation*} A(X) = \int_{V} \sqrt{EG - F^2}  \, du \, dv = \int_{V} \sqrt{\det \begin{pmatrix}
 E & F \\
 F & G
 \end{pmatrix}}  \, du \, dv. \end{equation*}
\end{opomba}

\begin{definicija}
\label{def_prehodna_preslikava_med_parametrizacijama}
 Naj bosta $r: V_1 \subseteq  \mathbb{R}^2 \to  X \subseteq  \mathbb{R}^3$ in $\tilde{r}: V_2 \subseteq  \mathbb{R}^2 \to  X \subseteq  \mathbb{R}^3$ različni regularni parametrizaciji ploskve $X$.
 Potem je preslikava \begin{align*}
  g = \tilde{r}^{-1} \circ r: V_1 &\longrightarrow V_2 \\
  (u,v) &\longmapsto (\tilde{u}(u,v), \tilde{v}(u,v)),
 \end{align*}
 prehodna preslikava med parametrizacijama $r$ in $\tilde{r}$.
\end{definicija}

Velja $r(u,v) = \tilde{r}(g(u,v))$. Poglejmo si, kaj se zgodi z matriko prve fundamentalne forme transformaciji med parametrizacijama.

% TODO tukaj ta razmislek premakni v izrek

\subsection{Transformacijska pravila za prvo fundamentalno formo}

Naj bosta $r: V_1 \to  X$ in $\tilde{r}: V_2 \to  X$ dve parametrizaciji iste ploskve. Med njima velja $r(u,v) = \tilde{r}(\tilde{u}(u,v), \tilde{v}(u,v))$.
To nam da prehodno preslikavo $g: \tilde{r}^{-1} \circ  r : V_1 \to  V_2$. Vektor, razvit po bazi $\left\{ r_u, r_v \right\}$ bomo skušali
razviti bo bazi $\left\{ \tilde{r}_{\tilde{u}}, \tilde{r}_{\tilde{v}} \right\}$. Imamo 
\begin{align*}
  r(u,v) &= \tilde{r}(\tilde{u}(u,v), \tilde{v}(u,v)), \\
  r_u   &= \tilde{r}_{\tilde{u}} \tilde{u}_u + \tilde{r}_{\tilde{v}} \tilde{v}_u, \\ 
  r_v &= \tilde{r}_{\tilde{u}} \tilde{u}_v + \tilde{r}_{\tilde{v}} \tilde{v}_v.
\end{align*}
Zdaj hočemo vektor $a r_u + b r_v$ zapisati v obliki $\alpha \tilde{r}_\tilde{u} + \beta \tilde{r}_\tilde{v}$. Dobimo 
\begin{equation*} a r_u + b r_v  = a(\tilde{r}_{\tilde{u}} \tilde{u}_u + \tilde{r}_{\tilde{v}} \tilde{v}_u) + b (\tilde{r}_{\tilde{u}} \tilde{u}_v + \tilde{r}_{\tilde{v}} \tilde{v}_v) 
= \underbrace{(a \tilde{u}_u + b \tilde{u}_v)}_{\alpha} \tilde{r}_{\tilde{u}} + \underbrace{(a \tilde{v}_u + b \tilde{v}_v)}_{\beta} \tilde{r}_{\tilde{v}}. \end{equation*}
Torej za vsak par vektorjev $(a, b)^{T}$, $(\alpha, \beta)^{T}$ velja zveza 
\begin{equation*} \begin{pmatrix}
  \alpha\\
  \beta\\
\end{pmatrix} = 
\underbrace{\begin{pmatrix}
  \tilde{u}_u & \tilde{u}_v\\
  \tilde{v}_u & \tilde{v}_v \\
\end{pmatrix}}_{\text{Jac}(g)}  
\begin{pmatrix}
  a\\
  b\\
\end{pmatrix}. \end{equation*}

Torej za vse pare vektorjev velja 
\begin{align*} 
    \left\langle \begin{pmatrix}
      a\\
      b\\
    \end{pmatrix}, \begin{pmatrix}
      c\\
      d\\
    \end{pmatrix} \right\rangle  &= \left\langle \begin{pmatrix}
      \alpha\\
      \beta  \\
    \end{pmatrix}, \begin{pmatrix}
      \gamma\\
      \delta  \\
    \end{pmatrix} \right\rangle  \\
    \begin{pmatrix}
      a & b\\
    \end{pmatrix} \begin{pmatrix}
    E & F \\
    F & G
    \end{pmatrix} \begin{pmatrix}
      c\\
      d  \\
    \end{pmatrix}  &= \begin{pmatrix}
      \alpha & \beta\\
    \end{pmatrix} \begin{pmatrix}
    \tilde{E} & \tilde{F} \\
    \tilde{F} & \tilde{G}
    \end{pmatrix} \begin{pmatrix}
      \gamma \\
      \delta  \\
    \end{pmatrix} \\ 
    &= \begin{pmatrix}
      a & b\\
    \end{pmatrix} 
    \operatorname{Jac}(g)^{T}
    \begin{pmatrix}
      \tilde{E} & \tilde{F} \\
      \tilde{F} & \tilde{G}
      \end{pmatrix}
    \operatorname{Jac}(g)
    \begin{pmatrix}
      c\\
      d  \\
    \end{pmatrix}.
\end{align*} 
Od tod sledi naslednji izrek.

\begin{izrek}
\label{izr_transformacija_1_forme}
  Naj za parametrizaciji ploskve $r: V_1 \to  X$ in $\tilde{r}: V_2 \to X$ velja 
  \begin{equation*} r(u,v) = \tilde{r}(\tilde{u}(u,v), \tilde{v}(u,v)) = \tilde{r}(g(u,v)).\end{equation*}
  Potem za prvi fundamentalni formi glede na ti parametrizaciji velja 
  \begin{equation*} \begin{pmatrix}
  E & F \\
  F & G
  \end{pmatrix} = \operatorname{Jac}(g)^{T}
  \begin{pmatrix}
    \tilde{E} & \tilde{F} \\
    \tilde{F} & \tilde{G}
    \end{pmatrix}
  \operatorname{Jac}(g). \end{equation*}
\end{izrek}

\begin{posledica}
\label{psl_dobra_definiranost_ploscine}
 Definicija ploščine \href{def_ploscina_plosvke} je dobra.
\end{posledica}
\noindent
{\em Dokaz:\/}
 Dokazujemo 
 \begin{equation*} A(X) = \int_{V_1} \sqrt{EG - F^2}   \, du \, dv = \int_{V_2}  \sqrt{\tilde{E}\tilde{G} -  \tilde{F}^2}  \, d \tilde{u} \, d \tilde{v}.\end{equation*}
 Po izreku o transformaciji prve fundamentalne forme \href{izr_transformacija_1_forme} velja 
 \begin{align*}
     A(X) &= \int_{V_1} \sqrt{\det \begin{pmatrix}
     E & F \\
     F & G
     \end{pmatrix}}   \, du \, dv \\
      &= \int_{V_1}  \sqrt{\det \left( \text{Jac}(g)^{T}
      \begin{pmatrix}
        \tilde{E} & \tilde{F} \\
        \tilde{F} & \tilde{G}
        \end{pmatrix}
      \text{Jac}(g) \right) }   \, du \, dv \\ 
      &= \int_{V_1} \sqrt{\begin{pmatrix}
        \tilde{E} & \tilde{F} \\
        \tilde{F} & \tilde{G}
      \end{pmatrix}} \left| \det \left( \operatorname{Jac}(g) \right) \right|  \, du \, dv \\ 
     \text{(uvedba novih spremenljivk)} &= \int_{V_2} \sqrt{\begin{pmatrix}
      \tilde{E} & \tilde{F} \\
      \tilde{F} & \tilde{G}
      \end{pmatrix}}   \, d \tilde{u} \, d \tilde{v}.  
 \end{align*}
\qed

Recimo, da imamo podano $V \subseteq  \mathbb{R}^2$ in matrično funkcijo \begin{align*}
  M: V &\longrightarrow M_2(\mathbb{R})_\text{simetrične, pozitivno definitne}  \\
  (u,v) &\longmapsto \begin{pmatrix}
  E & F \\
  F & G
  \end{pmatrix}_{(u,v)}.
\end{align*} 
Zanimivo vprašanje se glasi: Ali lahko to funkcijo realiziramo v vloženi mnogoterosti? Odgovor je da (ampak lokalno, ker bi lahko prišlo do problemov s samopresečišči).

\subsection{Ukrivljenost}

\subsubsection{Druga fundamentalna forma}

Naj bo $X \subseteq \mathbb{R}^3$ ploskev s parametrizacijo $r: V \subseteq \mathbb{R}^2 \to  X$.
Intuitivno je ukrivljenost ploskve $X$ v točki $r(u,v) = m \in X$ `hitrost` oddaljevanja $X$ od tangentne ravnine $T_{m}X$.
Naj bo $n$ normala na ravnino $T_{m}X$ v točki $m$. Naj bo točka $r(u', v') = r(u + \Delta u, v + \Delta v)$ blizu točke $m = r(u,v)$. Izmeriti hočemo razdaljo od točke $r(u', v')$ do $T_mX$.
Dobimo 
\begin{equation*} d = \langle n, r(u', v') - r(u,v) \rangle.\end{equation*}
Ker sta spremembi $\Delta u$ in $\Delta v$ majhni, naredimo Taylorjev razvoj, ter izrazimo 
\begin{equation*} r(u', v') - r(u,v) = r_u \Delta u + r_v \Delta v + \frac{1}{2} \left(r_{uu} (\Delta u)^2 + 2r_{uv} \Delta u \Delta v + r_{vv} (\Delta v)^2 \right) + \ldots \end{equation*}

Ker vemo tudi, da je normala pravokotna na vektorja $r_u$ in $r_v$, lahko zapišemo 
\begin{equation*} d = \langle n, r(u', v') - r(u,v) \rangle \approx \underbrace{\langle r_u, n \rangle}_0   \Delta u + \underbrace{\langle r_v, n \rangle}_0  \Delta v  
+ \frac{1}{2} \left( \langle r_{uu}, n \rangle (\Delta u)^2 +  2\langle r_{uv}, n \rangle \Delta u \Delta v + \langle r_{vv}, n \rangle (\Delta v)^2 \right).\end{equation*}
oziroma 
\begin{equation*} d \approx \frac{1}{2} \left( \langle r_{uu}, n \rangle (\Delta u)^2 +  2\langle r_{uv}, n \rangle \Delta u \Delta v + \langle r_{vv}, n \rangle (\Delta v)^2 \right). \end{equation*}

Zdaj lahko smiselno definiramo drugo fundamentalno formo ploskve.

\begin{definicija}
\label{def_druga_fundamentalna_forma_ploskve}
 Naj bo $X \subseteq \mathbb{R}^3$ ploskev s parametrizacijo $r: V \subseteq \mathbb{R}^2 \to  X$. Druga fundamentalna forma $X$ v točki $m = r(u,v)$ je podana z matriko 
 \begin{equation*} \begin{pmatrix}
 L & M \\
 M & N
 \end{pmatrix}_{(u,v)} = \begin{pmatrix}
 L(u,v) & M(u,v) \\
 M(u,v) & N(u,v)
 \end{pmatrix}, \end{equation*}
 kjer so $L, M, N: V \to  \mathbb{R}$ funkcije s predpisi 
 \begin{align*}
    L(u,v)  &= \langle r_{uu}(u,v), n(u,v) \rangle  \\
    M(u,v)  &= \langle r_{uv}(u,v), n(u,v) \rangle  \\
    N(u,v)  &= \langle r_{vv}(u,v), n(u,v) \rangle  
 \end{align*}

\end{definicija}

\begin{opomba}
    Za drugo fundamentalno formo je res nujen skalarni produkt v $\mathbb{R}^3$ (za razliko od prve fundamentalne forme, ki v vsaki točki določa drug skalarni produkt). 
\end{opomba}

Razmislili smo, da velja ocena razdalje
\begin{equation*} d(u, v, \Delta u, \Delta v) \approx \frac{1}{2} \begin{pmatrix}
  \Delta u  &  \Delta v 
\end{pmatrix} \begin{pmatrix}
  L & M \\
  M & N
\end{pmatrix}_{(u,v)}  
\begin{pmatrix}
  \Delta u \\
  \Delta v 
\end{pmatrix}.\end{equation*}

Od te točke naprej predpostavljamo, da so vse krivulje naravno parametrizirane. Naj bo $\gamma: [a,b] \to X$ gladka krivulja. 
Spomnimo se definicije fleksijske ukrivljenosti krivulje \href{def_fleksijska_ukrivljenost}, velja $\kappa(t) = \lvert\lvert \ddot{\gamma}(t) \rvert\rvert$.
V nadaljevanju bomo razdelili vektor pospeška na geodetsko in normalno komponentno, torej 
\begin{equation*} \ddot{\gamma}(t) = \ddot{\gamma}_g(t) + \ddot{\gamma}_n(t).\end{equation*}

Zato najprej definirajmo geodetsko in normalno ukrivljenost krivulje.

\begin{definicija}
\label{def_geodetska_in_normalna_ukrivljenost}
Naj bo $X \subseteq \mathbb{R}^3$ ploskev s parametrizacijo $r: V \subseteq \mathbb{R}^2 \to  X$ in naj bo $\gamma: [a,b] \to X$ gladka krivulja. 
Definirajmo normalo na krivuljo $\gamma$ kot preslikavo $n_{\gamma}: [a,b] \to  TX$ s predpisom
\begin{equation*} n_{\gamma}(t) = \frac{r_u(u(t), v(t)) \times  r_v(u(t), v(t))}{\lvert\lvert r_u(u(t), v(t)) \times  r_v(u(t), v(t)) \rvert\rvert }.\end{equation*}
Potem definiramo normalno in geodetsko ukrivljenost krivulje $\gamma$ kot 
\begin{align*}
    \kappa_n(t) &= \langle \ddot{\gamma}(t), n_{\gamma}(t) \rangle,  \\
    \kappa_g(t) &= \langle \ddot{\gamma}(t),  n_{\gamma}(t)  \dot{\gamma}(t) \rangle.
\end{align*}

\end{definicija}

Zdaj si poglejmo, kako se izraža normalna ukrivljenost krivulje s pomočjo druge fundamentalne forme.

\begin{izrek}
\label{izr_izrazava_normalne_ukrivljenosti_z_drugo_fundamentalno_formo}
  Naj bo $\gamma: [a,b] \to X$ gladka krivulja, podana s parametrizacijo $\gamma(t) = r(u(t), v(t))$. Tedaj lahko njeno normalno ukrivljenost
  izračunamo po formuli 
  \begin{equation*} \kappa_n(t) = \begin{pmatrix}
    \dot{u}(t) & \dot{v}(t) 
  \end{pmatrix}  
  \begin{pmatrix}
    L & M \\
    M & N
  \end{pmatrix}_{(u(t), v(t))}  
  \begin{pmatrix}
    \dot{u}(t) \\
    \dot{v}(t) 
  \end{pmatrix}.  \end{equation*}
\end{izrek}

\noindent
{\em Dokaz:\/}
 Najprej zapišemo prva dva odvoda 
 \begin{align*}
     \dot{\gamma}(t) &= r_u \dot{u} + r_v \dot{v}, \\
     \ddot{\gamma}(t) &=  r_u \ddot{u}+  r_v \ddot{v}+ r_{uu} \dot{u}^2 + 2r_{uv} \dot{u} \dot{v} + r_{vv} \dot{v}^2   .
 \end{align*}
Od tod dobimo 
\begin{align*}
    \kappa_n = \langle \ddot{\gamma}, n \rangle  &= \langle r_{uu}, n \rangle \dot{u}^2 + 2  \langle r_{uv}, n \rangle \dot{u} \dot{v} + \langle r_{vv}, n \rangle  \dot{v}^2 \\
     &= L \dot{u}^2 + 2M \dot{u} \dot{v} + F \dot{v}^2 \\
     &= \begin{pmatrix}
        \dot{u} & \dot{v} 
      \end{pmatrix}
      \begin{pmatrix}
        L & M \\
        M & N
      \end{pmatrix}  
      \begin{pmatrix}
        \dot{u} \\
        \dot{v} 
      \end{pmatrix}.
\end{align*}
\qed

Vsako funkcijo normalne ukrivljenosti lahko razširimo na tangentno ravnino na naslednji način.

\begin{definicija}
\label{def_razsiritev_normalne_ukrivljenosti}
    Naj bo $X \subseteq \mathbb{R}^3$ ploskev s parametrizacijo $r: V \subseteq \mathbb{R}^2 \to  X$
    in naj bo $\begin{pmatrix}
    L & M \\
    M & N
    \end{pmatrix}$ druga fundamentalna forma ploskve $X$ glede na $r$. Naj bo $r(u_0, v_0) = m \in  X$. Vemo, da je $\left\{ r_u(u_0, v_0), r_v(u_0, v_0) \right\}$ baza za
    tangetno ravnino $T_mX$, zato lahko vsak vektor v njej enolično zapišemo v obliki $\xi r_u(u_0, v_0) + \eta r_v(u_0, v_0).$
    Potem obstaja funkcija \begin{align*}
       \kappa_n : T_mX &\longrightarrow \mathbb{R} \\
        \kappa_n(\xi r_u(u_0, v_0) + \eta r_v(u_0, v_0)) &=
        \begin{pmatrix}
          \xi  & \eta
        \end{pmatrix}
        \begin{pmatrix}
          L & M \\
          M & N
        \end{pmatrix}_{(r(u_0, v_0))}  
        \begin{pmatrix}
          \xi \\
          \eta
        \end{pmatrix}.
    \end{align*} 
Včasih zlorabimo notacijo in pišemo \begin{equation*}
  \kappa_n(\xi, \eta) := \kappa_n(\xi r_u(u_0, v_0) + \eta r_v(u_0, v_0)).
\end{equation*}  
\end{definicija}

\begin{opomba}
 Če je $(\xi, \eta) = (\dot{u}, \dot{v})$ in je $\gamma$ naravno parametrizirana krivulja, potem par $(\xi, \eta)$ leži
 na enotski krožnici v ravnini $T_mX$. V $\mathbb{R}^3$ je to res običajna krožnica glede na
 evklidkski skalarni produkt, v koordinatah $(\xi, \eta)$ pa jo določa enačba 
 \begin{equation*} E \xi^2 + 2F \xi \eta + G \eta^2  = 1.\end{equation*}
\end{opomba}

\begin{definicija}
\label{def_glavni_ukrivljenosti_ploskve}
 Ekstrema funkcije $\kappa_n: T_mX \to  \mathbb{R}$, skrčene na enotsko sfero $E \xi^2 + 2F \xi \eta + G \eta^2  = 1$, imenujemo glavni ukrivljenosti ploskve $X$ v točki $m$. 
 Označimo ju s $\kappa_1$ in $\kappa_2$. 
\end{definicija}

\begin{opomba}
 Ali minimum označimo s $\kappa_1$ ali $\kappa_2$, je stvar dogovora. Ekstrema obstajata, ker je krožnica kompakt, torej ima zvezna funkcija $\kappa_n$ na njej minimum in maksimum.
\end{opomba}

\begin{definicija}
\label{def_glavni_smeri}
 Tangetna vektorja $\xi_1 r_u  + \eta_1 r_v$, $\xi_2 r_u  + \eta_2 r_v \in T_mX$, ki sta (različna) ekstrema funkcije $\kappa_n$, se imenujeta glavni smeri.
\end{definicija}

\begin{opomba}
 Kmalu bomo dokazali, da sta glavni smeri res kvečjem dve, zato je takšna definicija upravičena (če bi bila $\kappa_n$ poljubna zvezna funkcija, bi lahko imela več ekstremov).
\end{opomba}

\begin{definicija}
\label{def_Gaussova_ukrivljenost}
 Gaussova ukrivljenost ploskve $X$ v točki $m$ je podana s produktom 
 \begin{equation*} \kappa(m) = \kappa_1(m) \kappa_2(m).\end{equation*}
\end{definicija}

\begin{opomba}
 Izkazalo se bo, da je Gaussova ukrivljenost izometrična varianta in da je tesno povezana z Eulerjevo karakteristiko.
\end{opomba}


\begin{definicija}
\label{def_povprecna_ukrivljenost}
    Povprečna ukrivljenost ploskve $X$ v točki $m$ je podana s formulo 
    \begin{equation*} H(m) = \frac{1}{2} (\kappa_1(m) + \kappa_2(m) ).\end{equation*}
\end{definicija}

Z naslednjim izrekom bomo utemeljili, da sta glavni ukrivljenosti res kvečjem dve. 

\begin{izrek}
\label{izr_glavni_ukrivljenosti_sta_nicli_kvadratne_emacbe}
  Glavni ukrivljenosti ploskve $X$ v točki $m \in X$ sta ničli kvadratne enačbe 
  \begin{equation*} \det \left( \begin{pmatrix}
  L & M \\
  M & N
  \end{pmatrix}_{(m)} - \lambda \begin{pmatrix}
  E & F \\
  F & G
  \end{pmatrix}_{(m)}  \right) = 0. \end{equation*}
\end{izrek}
\noindent
{\em Dokaz:\/}
 Ekstrema $\lambda_1$, $\lambda_2$ sta vezana ekstrema za funkcijo $\kappa_n: T_mX \to \mathbb{R}$.
 Zaradi homeomorfizma $T_mX \approx \mathbb{R}^2$ lahko identificiramo elemente $T_mX$ s pari $(\xi, \eta)$. Naša krožnica je
 določena z vezjo $\lvert\lvert (\xi, \eta) \rvert\rvert = 1$ oziroma ekvivalentno $\lvert\lvert (\xi, \eta) \rvert\rvert^2 = 1$. Torej iščemo ekstreme funkcije \begin{equation*}
   \kappa_n(\xi, \eta) = \begin{pmatrix}
     \xi & \eta 
   \end{pmatrix}
   \begin{pmatrix}
     L & M \\
     M & N
   \end{pmatrix}  
   \begin{pmatrix}
     \xi \\
     \eta 
   \end{pmatrix} = L \xi^2 + 2M \xi \eta + N \eta^2
 \end{equation*}  
   pri vezi \begin{equation*}
     g(\xi, \eta)  = \lvert\lvert (\xi, \eta) \rvert\rvert =  \lvert\lvert (\xi, \eta) \rvert\rvert^2 = \begin{pmatrix}
       \xi & \eta 
     \end{pmatrix}
     \begin{pmatrix}
       E & F \\
       F & G
     \end{pmatrix}  
     \begin{pmatrix}
       \xi \\
       \eta 
     \end{pmatrix} = E \xi^2 + 2F \xi \eta + G \eta^2 = 1.
   \end{equation*}  
    Zdaj se s pomočjo analize 2 spomnimo, da se vezane ekstreme išče s pomočjo Lagrangeeve funkcije \begin{equation*}
      \kappa_n(\xi, \eta) - \lambda g(\xi, \eta).
    \end{equation*}  
    Da bo skalar $\lambda$ določal ekstrem, mora veljati zveza \begin{equation*}
      \operatorname{grad}(\kappa_n(\xi, \eta) - \lambda g(\xi, \eta)) = 0.
    \end{equation*}  
    Zaradi linearnosti gradienta je to ekvivalentno zvezi \begin{equation*}
        \operatorname{grad}(\kappa_n(\xi, \eta)) - \lambda \operatorname{grad}(g(\xi, \eta)) = 0.
    \end{equation*}
    Zdaj poračunamo gradienta, da dobimo \begin{align*}
        \operatorname{grad}(\kappa_n(\xi, \eta)) = \begin{pmatrix}
            \frac{ \partial \kappa_n }{ \partial \xi } \\
            \frac{ \partial \kappa_n }{ \partial \eta }
        \end{pmatrix}  &= 2 \begin{pmatrix}
        L & M \\
        M & N
        \end{pmatrix}  \begin{pmatrix}
            \xi \\
            \eta
        \end{pmatrix},  \\
        \operatorname{grad}(g(\xi, \eta)) = \begin{pmatrix}
            \frac{ \partial g }{ \partial \xi } \\
            \frac{ \partial g }{ \partial \eta }
        \end{pmatrix}  &= 2 \begin{pmatrix}
        E & F \\
        F & G
        \end{pmatrix}  \begin{pmatrix}
            \xi \\
            \eta
        \end{pmatrix}.
    \end{align*}

    Torej rešujemo enačbo \begin{equation*}
      \left(
        \begin{pmatrix}
      L & M \\
      M & N
      \end{pmatrix}  - \lambda \begin{pmatrix}
    E & F \\
    F & G
    \end{pmatrix} \right) \begin{pmatrix}
        \xi \\
        \eta
    \end{pmatrix} =  \begin{pmatrix}
        0 \\
        0
    \end{pmatrix}.
    \end{equation*}  
     Ta enačba ima netrivialne rešitve natanko tedaj, ko je \begin{equation*}
       \det \left( 
        \begin{pmatrix}
      L & M \\
      M & N
      \end{pmatrix}  - \lambda \begin{pmatrix}
    E & F \\
    F & G
    \end{pmatrix} \right) = 0.
     \end{equation*}  
     Denimo, da imamo neki rešitvi $\lambda_1$, $\lambda_2$. Potem obstajata vektorja $(\xi_1, \eta_1)^{T}$, $(\xi_2, \eta_2)^{T}$, da je za $i = 1,2$
     \begin{equation*}
        \left(
            \begin{pmatrix}
          L & M \\
          M & N
          \end{pmatrix}  - \lambda_i \begin{pmatrix}
        E & F \\
        F & G
        \end{pmatrix} \right) \begin{pmatrix}
            \xi_i \\
            \eta_i
        \end{pmatrix} =  \begin{pmatrix}
            0 \\
            0
        \end{pmatrix}.
     \end{equation*}  
    Če vzamemo enotska vektorja $(\xi_1, \eta_1)^{T}$, $(\xi_2, \eta_2)^{T}$, in množimo $i$-to zgornjo enačbo z leve z $(\xi_i, \eta_i)$, 
    dobimo 
    \begin{equation*}
        \underbrace{\begin{pmatrix}
            \xi_i & \eta_i
        \end{pmatrix}
            \begin{pmatrix}
          L & M \\
          M & N
          \end{pmatrix}\begin{pmatrix}
            \xi_i \\
            \eta_i
        \end{pmatrix}}_{\kappa_n(\xi, \eta)}    - \lambda_i 
        \underbrace{\begin{pmatrix}
            \xi_i & \eta_i
        \end{pmatrix}
        \begin{pmatrix}
        E & F \\
        F & G
        \end{pmatrix} \begin{pmatrix}
            \xi_i \\
            \eta_i
        \end{pmatrix}}_1  =  \begin{pmatrix}
            0 \\
            0
        \end{pmatrix}.
     \end{equation*}
     Desni člen je enak $1$, ker smo izbrali enotska vektorja glede na skalarni produkt, porojen s to matriko. Preostaneta nam torej enačbi \begin{equation*}
       \kappa_n(\xi_i, \eta_i) = \lambda_i, \,\,\, i = 1,2.
     \end{equation*}  
     Točki $(\xi_i, \eta_i)$ sta vezana ekstrema funkcije $\kappa_n$ pri vezi $g = 1$. Torej sta to ničli kvadratnega polinoma
     \begin{equation*}
        \det \left( 
         \begin{pmatrix}
       L & M \\
       M & N
       \end{pmatrix}  - \lambda \begin{pmatrix}
     E & F \\
     F & G
     \end{pmatrix} \right) = 0.
      \end{equation*} 
\qed

\begin{definicija}
\label{def_posploseni_lastni_problem}
 Posplošeni lastni problem ali relativni lastni problem je lastni problem oblike \begin{equation*}
   A(\vec{v}) = \lambda B(\vec{v}).
 \end{equation*}  
  Če je $B = Id$, potem dobimo običajni lastni problem \begin{equation*}
    A(\vec{v}) = \lambda \vec{v}.
  \end{equation*}  
\end{definicija}

Iz dokaza prejšnjega izreka je razvidno, da sta $\kappa_1$ in $\kappa_2$ lastni vrednosti martrike \begin{equation*}
  \begin{pmatrix}
  E & F \\
  F & G
  \end{pmatrix}^{-1}\begin{pmatrix}
  L & M \\
  M & N
  \end{pmatrix},
\end{equation*}  
  glavni smeri $(\xi_1, \eta_1)^{T}$, $(\xi_2, \eta_2)^{T}$ pa sta njena lastna vektorja. Tej matriki včasih rečemo
  Weingartnova matrika.


\begin{izrek}
\label{izr_izrazava_gaussove_ukrivljenosti}
Za Gaussovo ukrivljenost $\kappa: X \to  \mathbb{R}$ velja zveza \begin{equation*}
\kappa(m) = \frac{(LN - M^2)(m)}{(EG - F^2)(m)}. 
\end{equation*}  
\end{izrek}

\noindent
{\em Dokaz:\/}
Po definiciji je $\kappa(m) = \kappa_1(m) \kappa_2(m)$. Razmislili smo že, da sta glavni ukrivljenosti
lastni vrednosti matrike \begin{equation*}
  \begin{pmatrix}
    E & F \\
    F & G
    \end{pmatrix}_{(m)}^{-1}\begin{pmatrix}
    L & M \\
    M & N
    \end{pmatrix}_{(m)},
\end{equation*}  
  torej bo njun produkt (ki je natanko Gaussova ukrivljenost) enak njeni determinanti. Torej \begin{equation*}
    \kappa(m) = \kappa_1(m) \kappa_2(m) = \det \left(  \begin{pmatrix}
      E & F \\
      F & G
      \end{pmatrix}_{(m)}^{-1}\begin{pmatrix}
      L & M \\
      M & N
      \end{pmatrix}_{(m)} \right) = \frac{(LN - M^2)(m)}{(EG - F^2)(m)}.
  \end{equation*}  
    
\qed

\begin{izrek}
\label{izr_izrazava_povprecne_ukrivljenosti}
 Za povprečno ukrivljenost velja zveza \begin{equation*}
 H(m) = \frac{1}{2} \frac{(LF - 2MF + NE)(m)}{(EG - F^2)(m)}.
 \end{equation*}  
\end{izrek}
\noindent
{\em Dokaz:\/}
Podobno kot v prejšnjem dokazu uporabimo dejstvo, da sta glavni ukrivljenosti lastni vrednosti matrike \begin{equation*}
  \begin{pmatrix}
    E & F \\
    F & G
    \end{pmatrix}_{(m)}^{-1}\begin{pmatrix}
    L & M \\
    M & N
    \end{pmatrix}_{(m)},
  \end{equation*}  
kar pomeni, da bo njuna vsota enaka sledi te matrike. Imamo torej \begin{align*}
  H(m) = \frac{1}{2}(\kappa_1(m) + \kappa_2(m)) &= \frac{1}{2} \mathrm{Tr} \left( \begin{pmatrix}
    E & F \\
    F & G
    \end{pmatrix}_{(m)}^{-1}\begin{pmatrix}
    L & M \\
    M & N
    \end{pmatrix}_{(m)} \right)   \\
     &= \frac{1}{2(EG - F^2)(m)} \mathrm{Tr} \left(  \begin{pmatrix}
      G & -F \\
      -F & E
      \end{pmatrix}_{(m)} \begin{pmatrix}
      L & M \\
      M & N
      \end{pmatrix}_{(m)} \right) \\
      &=  \frac{1}{2(EG - F^2)(m)} \mathrm{Tr} \left( \begin{pmatrix}
        LG - MF & * \\
        * & -MF + NE
        \end{pmatrix}_{(m)} \right) \\
        &= \frac{1}{2} \frac{(LF - 2MF + NE)(m)}{(EG - F^2)(m)}.
\end{align*}
\qed


\begin{opomba}
Zgornja izreka bi lahko dokazali drugače z uporabo dejstva, da imata polinoma $a x^2 + b x + c$ in $x^2 + \frac{b}{a} x + \frac{c}{a}$ isti ničli.
Glavni ukrivljenosti sta ničli kvadratne enačbe \begin{equation*}
 p(\lambda) = a \lambda^2 + b \lambda + c =  \det \left( 
   \begin{pmatrix}
 L & M \\
 M & N
 \end{pmatrix}  - \lambda \begin{pmatrix}
E & F \\
F & G
\end{pmatrix} \right) = 0.
\end{equation*}
Po Vietovih pravilih pa vemo, da je produkt ničel enak izrazu \begin{equation*}
\kappa_1 \kappa_2 = \frac{c}{a}.
\end{equation*}  
Očitno velja, da je $$p(0) = c = \det \begin{pmatrix}
L & M \\
M & N
\end{pmatrix}.$$ Po drugi stani pa lahko dobimo \begin{equation*}
a = \lim_{\lambda \to \infty} \frac{1}{\lambda^2} p(\lambda) = \ldots = \det \begin{pmatrix}
E & F \\
F & G
\end{pmatrix}.
\end{equation*}
Torej bo res \begin{equation*}
\kappa = \kappa_1 \kappa_2 = \frac{c}{a} = \frac{LN - M^2}{EG - F^2}.
\end{equation*}
Po drugi strani pa imamo Vietovo pravilo za vsoto ničel, torej $\kappa_1 + \kappa_2 = -\frac{b}{a}$. Če poračunamo
$b$ po predpisu za polinom $p$, dobimo natanko izraz povprečne ukrivljenosti iz prejšnjega izreka. 
\end{opomba}

\begin{izrek}
\label{izr_glavni_smeri_sta_si_pravokotni}
Glavni smeri $(\xi_1, \eta_1)_{(m)}^{T}$, $(\xi_2, \eta_2)_{(m)}^{T}$ sta si pravokotni v $T_m X$. 
\end{izrek}
\noindent
{\em Dokaz:\/}
Dokaz poteka prek sklepa o sebi adjungiranosti preslikave \begin{equation*}
  \begin{pmatrix}
    E & F \\
    F & G
    \end{pmatrix}^{-1}\begin{pmatrix}
    L & M \\
    M & N
    \end{pmatrix}.
\end{equation*}  
Najprej za večjo preglednost vpeljimo nove oznake \begin{equation*}
A := \begin{pmatrix}
E & F \\
F & G
\end{pmatrix}, \,\,\, B := \begin{pmatrix}
L & M \\
M & N
\end{pmatrix}, \,\,\, C := \begin{pmatrix}
  E & F \\
  F & G
  \end{pmatrix}^{-1}\begin{pmatrix}
  L & M \\
  M & N
  \end{pmatrix} = A^{-1}B.
\end{equation*}  
Ker vemo, da za sebi adjungirane preslikave (ki imajo različne lastne vektorje) velja, da morajo biti ti med seboj pravokotni,
je dovolj dokazati sebi adjungiranost preslikave $C$. Torej za vsaka vektroja $\vec{v}, \vec{w} \in  T_m X$ dokazujemo \begin{equation*}
\langle C\vec{v}, \vec{w} \rangle = \langle \vec{v}, C \vec{w} \rangle.
\end{equation*}  
Pri tem se zavedajmo, da je za poljubna vektorja skalarni produkt določen s pomočjo prve fundamentalne forme. V bazi $\left\{ r_u, r_v\right\} $ je podan z \begin{equation*}
  \langle \vec{v}, \vec{w} \rangle = \vec{v}^{T} A \vec{w}. 
\end{equation*}  
Torej imamo za vsaka $\vec{v}, \vec{w} \in T_mX$ enačbi \begin{align*}
  \langle C\vec{v}, \vec{w} \rangle &= (C\vec{v})^{T} A \vec{w} = \vec{v}^{T} C^{T} A \vec{w}, \\
  \langle \vec{v}, C^{*} \vec{w} \rangle  &= \vec{v}^{T} A C^{*} \vec{w}.
\end{align*}
To pomeni, je zahtevi $C = C^{*}$ ekvivalentno $C^{T}A= AC^{*}$, oziroma $C^{T}A = AC$, kar pokažemo z računom \begin{equation*}
C^{T}A = (A^{-1}B)^{T}A = B^{T}A^{-T}A = BA^{-1}A = B = A A^{-1} B = AC.
\end{equation*}  
\qed

\subsubsection{Interpretacija glavne ukrivljenosti}

Intuitivno je ukrivljenost krivulje hitrost spreminjanja tangente. Hkrati pa velja, da čimbolj je krivulja ukrivljena, tem hitreje
normala spreminja smer. Podoben pojav lahko opazujemo tudi na ploskvah, bolj kot je ploskev ukrivljena, hitreje se spreminja normala.
Naj bo torej točka $m \in  X$, ter $U \subseteq X$ njena okolica, difeomorfna disku. Poglejmo si, kaj se dogaja z normalo po točkah $U$.

\begin{definicija}
\label{def_gaussova_preslikava}
Naj bo $X \subseteq  \mathbb{R}^3$ ploskev. Preslikava, ki točki priredi enotsko normalo, \begin{align*}
  \tilde{n}: X &\longrightarrow S^{2} \\
  m &\longmapsto n(m) \in  (T_mX)^{\perp}; \lvert\lvert n(m) \rvert\rvert = 1,   
\end{align*}se imenuje Gaussova preslikava.
\end{definicija}

Povprečno intenzivnost spreminjanja normale na $U$ lahko merimo s kvocientom površin \begin{equation*}
\frac{A(\tilde{n}_{*}(U))}{A(U)},
\end{equation*}  
kjer je $A(U)$ ploščina okolice $U \subseteq X$, $A(\tilde{n}(U))$ pa je ploščina slike \begin{equation*}
\tilde{n}: U \subseteq X \to \tilde{n}_{*}(U) \subseteq S^2. 
\end{equation*} 

Predpostavimo, da je preslikava $n = \tilde{n} \circ r: V \to S^2$ \begin{center}
  \adjustbox{scale=1.0,center}{
    \begin{tikzcd}
      V && U \subseteq X && {\tilde{n}_{*}(U) \subseteq S^2} 
      \arrow["{r}", from=1-1, to=1-3]
      \arrow["{\tilde{n}}", from=1-3, to=1-5]
      \arrow["n"', bend right=20, from=1-1, to=1-5]
    \end{tikzcd}
  }
\end{center}
parametrizacija ploskve $\tilde{n}_{*}(U) \subseteq S^2$. Po definiciji površine ploskve veljata izraza \begin{align*}
    A(U) &= \int_{V} \lvert\lvert r_u \times  r_v \rvert\rvert   \, du \, dv,  \\
    A(\tilde{n}_{*}(U)) &= \int_{V} \lvert\lvert n_u \times  n_v \rvert\rvert  \, du \, dv. 
\end{align*}
Vemo tudi, da velja \begin{equation*}\label{eq_1}
  \lvert\lvert n_u \times  n_v \rvert\rvert = \sqrt{\langle n_u \times n_v, n_u \times  n_v \rangle } 
\end{equation*}  
in hkrati (TODO: zakaj?) \begin{equation}
  \lvert\lvert n_u \times  n_v \rvert\rvert = \langle n, n_u \times n_v \rangle = \left\langle \frac{r_u \times r_v}{\lvert\lvert r_u \times  r_v \rvert\rvert } , n_u \times n_v \right\rangle .
\end{equation}  
Iz enačbe \href{eq_1}{(1)} sledi po Lagrangevem pravilu za dvojni vektorski produkt še \begin{equation} \label{eq_2}
  \lvert\lvert n_u \times  n_v \rvert\rvert = \frac{1}{\sqrt{EG - F^2} } (\langle r_u, n_u \rangle \langle r_v, n_v \rangle - \langle r_u, n_v \rangle \langle r_v, n_u \rangle  ).
\end{equation}  
Če zdaj enačbo $\langle r_u, n \rangle = 0$ paricalno odvajamo po $u$, dobimo \begin{equation*}
\langle r_{uu}, n \rangle + \langle r_u, n_u \rangle = 0 \implies \langle r_u, n_u \rangle = - L,  
\end{equation*}  
če pa jo odvajamo po $v$, dobimo \begin{equation*}
\langle r_{uv}, n \rangle + \langle r_u, n_v \rangle = 0 \implies  \langle r_u, n_v \rangle = - M
\end{equation*}  
Podobno z odvajanjem enačbe $\langle r_v, n \rangle = 0$ dobimo preostali dve enačbi \begin{align*}
  \langle r_v, n_u \rangle  &= - M, \\
  \langle r_v, n_v \rangle  &= - N.
\end{align*}
Od tod pa iz enačbe \href{eq_2}{(2)} sledi \begin{equation*}
  \lvert\lvert n_u \times  n_v \rvert\rvert = \frac{LN - M^2}{\sqrt{EG - F^2} }.
\end{equation*}  

S pomočjo te izražave lahko zapišemo \begin{equation*}
    \frac{A(\tilde{n}(U))}{A(U)} = \frac{\int_{V} \lvert\lvert n_u \times  n_v \rvert\rvert   \, du \, dv}{\int_{V} \lvert\lvert r_u \times  r_v \rvert\rvert   \, du \, dv} 
    = \frac{\int_{V} \frac{LN - M^2}{\sqrt{EG - F^2} }   \, du \, dv}{\int_{V} \sqrt{EG - F^2} \, du \, dv }.
    \end{equation*}  
Od tod pa s pomočjo izreka o povprečni vrednosti in zveznosti funkcij dobimo, da je \begin{equation*}
\lim_{V \to (u_0, v_0)} \frac{A(\tilde{n}(U))}{A(U)} = \frac{(LN - M^2)(m)}{(EG - F^2)(m)} = \kappa(m).
\end{equation*}  


\subsection{Ploščate ploskve}

\begin{definicija}
\label{def_ploscate_ploskve}
Ploskev $X$ je ploščata, če je njena Gaussova ukrivljenost $\kappa = 0$. 
\end{definicija}

\begin{izrek}
\label{izr_plsokev_je_ploscata_ce_je_del_valja_ali_tangentnopremosnosne_ploskve}
Ploskev $X$ je ploščata, če je del stožca, valja ali tangentno premonosne ploskve.
\end{izrek}

\begin{posledica}
\label{psl_plsokev_je_ploscata_iff_izometricna_kosu_ravnine}
Ploskev $X$ je ploščata natanko tedaj, ko je izometrična kosu ravnine.
\end{posledica}

Za dokaz tega izreka bomo potrebovali naslednji dve lemi.


\begin{lema}
\label{lem_obstoj_funkcij}
Naj bo $X$ poljubna ploskev. Tedaj obstaja za $X$ parametrizacija $s : V \to  X$ ter funkciji $\lambda, \mu : V \to  \mathbb{R}$, za kateri velja
\begin{align*}
  s_u(u,v) &= \lambda(u,v)e_1(u,v),  \\
  s_v(u,v) &= \mu(u,v)e_2(u,v).
\end{align*}
Preslikavi $e_1, e_2: V \to  TX$ označujeta glavni smeri ploskve $X$ glede na parametrizacijo $s$.
\end{lema}

\noindent
{\em Dokaz:\/}
Naj bo $s: V_1 \to  X$ parametrizacija ploskve $X$. Iskano parametrizacijo bomo dobili kot reparametrizacijo $s$.
Lokalno lahko parametrizacijo $s = s(u,v), (u,v) \in V_1$ podamo kot parametrizacijo $s(u,v) = s(\xi, \eta), (\xi, \eta) \in V$.
To nam poda funkciji $s(u,v) = s(\xi(u,v), \eta(u,v))$ ter $s(\xi, \eta) = s(u(\xi, \eta), v(\xi, \eta))$.  
Iz prve enačbe po verižnem pravilu dobimo \begin{align*}
    s_u &= \frac{ \partial \xi }{ \partial u } s_{\xi} + \frac{ \partial \eta }{ \partial u } s_{\eta},  \\
    s_v &= \frac{ \partial \xi }{ \partial v } s_{\xi} + \frac{ \partial \eta }{ \partial v } s_{\eta}.
\end{align*}
Zdaj zapišemo glavni smeri kot linearni kombinaciji vektorjev $s_u$ in $s_v$.
\begin{align*}
    e_1 &= \alpha s_u + \beta s_v, \\
    e_2 &= \gamma s_u + \delta s_v.
\end{align*}
V ti kombinaciji vstavimo izražavi $s_u$ in $s_v$, da dobimo \begin{align*}
  e_1 &= \alpha \left(\frac{ \partial \xi }{ \partial u } s_{\xi} + \frac{ \partial \eta }{ \partial u } s_{\eta}\right) + \beta \left(\frac{ \partial \xi }{ \partial v } s_{\xi} + \frac{ \partial \eta }{ \partial v } s_{\eta}\right), \\
  e_2 &= \gamma \left(\frac{ \partial \xi }{ \partial u } s_{\xi} + \frac{ \partial \eta }{ \partial u } s_{\eta}\right) + \delta \left(\frac{ \partial \xi }{ \partial v } s_{\xi} + \frac{ \partial \eta }{ \partial v } s_{\eta}\right).
\end{align*}
Od tod izrazimo glavni smeri kot lineani kombianciji vektorjev $s_{\xi}, s_{\eta}$. \begin{align*}
    e_{1} &= \left(\frac{ \partial \xi }{ \partial u } \alpha + \frac{ \partial \xi }{ \partial v } \beta \right) s_{\xi} + \left(\frac{ \partial \eta }{ \partial u } \alpha + \frac{ \partial \eta }{ \partial v } \beta \right) s_{\eta}, \\
    e_2 &= \left(\frac{ \partial \xi }{ \partial u } \gamma + \frac{ \partial \xi }{ \partial v } \delta \right) s_{\xi} + \left(\frac{ \partial \eta }{ \partial u } \gamma + \frac{ \partial \eta }{ \partial v } \delta \right) s_{\eta}.
\end{align*}

To lahko bolj kompaktno zapišemo kot \begin{align*}
    e_1 &=  \begin{pmatrix}
    \frac{ \strut\partial \xi }{ \strut\partial u }  & \frac{ \strut\partial \xi }{ \strut\partial v }  \\
    \frac{ \strut\partial \eta }{ \strut\partial u }  & \frac{ \strut\partial \eta }{ \strut\partial v } 
    \end{pmatrix} \begin{pmatrix}
    \alpha \\ \beta 
    \end{pmatrix},     \\
    e_2 &= \begin{pmatrix}
      \frac{ \strut\partial \xi }{ \strut\partial u }  & \frac{ \strut\partial \xi }{ \strut\partial v }  \\
      \frac{ \strut\partial \eta }{ \strut\partial u }  & \frac{ \strut\partial \eta }{ \strut\partial v } 
      \end{pmatrix} \begin{pmatrix}
      \gamma \\ \delta 
      \end{pmatrix}.  
\end{align*}

Zdaj se vprašamo, ali res potrebujemo funkciji $\lambda, \eta$? Če bi obstajala parametrizacija $s$, za katero bi veljalo $s_{\xi} = e_1$, $s_{\eta} = e_2$, potem bi veljalo 
\begin{align*}
    e_1 &= 1 \cdot s_{\xi} + 0 \cdot s_{\eta},  \\
    e_2 &= 0 \cdot s_{\xi} + 1 \cdot s_{\eta}.
\end{align*}
Torej bi morala biti v bazi $\left\{ s_{\xi}, s_{\eta} \right\}$ $e_1$ in $e_2$ standardna enotska vektorja, torej
\begin{align*}
  e_1 &=  \begin{pmatrix}
    \frac{ \strut\partial \xi }{ \strut\partial u }  & \frac{ \strut\partial \xi }{ \strut\partial v }  \\
    \frac{ \strut\partial \eta }{ \strut\partial u }  & \frac{ \strut\partial \eta }{ \strut\partial v } 
    \end{pmatrix} \begin{pmatrix}
  \alpha \\ \beta 
  \end{pmatrix}  = \begin{pmatrix}
    1 \\ 0 
    \end{pmatrix},   \\
  e_2 &= \begin{pmatrix}
    \frac{ \strut\partial \xi }{ \strut\partial u }  & \frac{ \strut\partial \xi }{ \strut\partial v }  \\
    \frac{ \strut\partial \eta }{ \strut\partial u }  & \frac{ \strut\partial \eta }{ \strut\partial v } 
    \end{pmatrix} \begin{pmatrix}
    \gamma \\ \delta 
    \end{pmatrix}  = \begin{pmatrix}
      0 \\ 1 
      \end{pmatrix}.
\end{align*}
To lahko zapišemo z eno samo enačbo kot \begin{equation*}
  \begin{pmatrix}
    \frac{ \strut\partial \xi }{ \strut\partial u }  & \frac{ \strut\partial \xi }{ \strut\partial v }  \\
    \frac{ \strut\partial \eta }{ \strut\partial u }  & \frac{ \strut\partial \eta }{ \strut\partial v } 
    \end{pmatrix} \begin{pmatrix}
      \alpha & \gamma \\ \beta & \delta
      \end{pmatrix} = \begin{pmatrix}
        1 & 0  \\ 0 & 1 
        \end{pmatrix}.
\end{equation*}  
Označimo sedaj \begin{equation*}
\begin{pmatrix}
A & C \\
B & D
\end{pmatrix} = \begin{pmatrix}
  \alpha & \gamma \\ \beta & \delta
  \end{pmatrix}^{-1},
\end{equation*}  
da dobimo pogoj 
\begin{equation*}
  \begin{pmatrix}
    \frac{ \strut\partial \xi }{ \strut\partial u }  & \frac{ \strut\partial \xi }{ \strut\partial v }  \\
    \frac{ \strut\partial \eta }{ \strut\partial u }  & \frac{ \strut\partial \eta }{ \strut\partial v } 
    \end{pmatrix} = \begin{pmatrix}
      A & C \\
      B & D
      \end{pmatrix}.
\end{equation*}  
Vrstici leve matrike sta gradienta funkcij $\xi$ in $\eta$.\begin{align*}
    \operatorname{grad} \xi (u,v) &= (A(u,v), C(u,v)), \\
    \operatorname{grad} \eta (u,v) &= (B(u,v), D(u,v)).
\end{align*}
Od tod pa očitno sledi, da morata biti polji $(A(u,v), C(u,v))$ in $(B(u,v), D(u,v))$ potencialni, kar v splošnem ni res.
Zato moramo zahtevo $s_{\xi} = e_1$, $s_{\eta} = e_2$ omejiti na $s_{\xi} = \lambda e_1$, $s_{\eta} = \mu e_2$.
Tako dobimo pogoj \begin{equation*}
  \begin{pmatrix}
    \frac{ \strut\partial \xi }{ \strut\partial u }  & \frac{ \strut\partial \xi }{ \strut\partial v }  \\
    \frac{ \strut\partial \eta }{ \strut\partial u }  & \frac{ \strut\partial \eta }{ \strut\partial v } 
    \end{pmatrix} \begin{pmatrix}
      \alpha & \gamma \\ \beta & \delta
      \end{pmatrix} = \begin{pmatrix}
        \frac{1}{\lambda} & 0  \\ 0 & \frac{1}{\mu}
        \end{pmatrix}.
\end{equation*}  
Tako imamo \begin{equation*}
  \begin{pmatrix}
    \frac{ \strut\partial \xi }{ \strut\partial u }  & \frac{ \strut\partial \xi }{ \strut\partial v }  \\
    \frac{ \strut\partial \eta }{ \strut\partial u }  & \frac{ \strut\partial \eta }{ \strut\partial v } 
    \end{pmatrix} = \begin{pmatrix}
      \frac{1}{\lambda} & 0  \\ 0 & \frac{1}{\mu} 
      \end{pmatrix}\begin{pmatrix}
      A & C \\
      B & D
      \end{pmatrix} = \begin{pmatrix}
        \frac{1}{\lambda} A & \frac{1}{\lambda} C \\
        \frac{1}{\mu} B & \frac{1}{\mu} D
        \end{pmatrix}.
\end{equation*}  
Zato lahko (TODO: po izreku iz analize 3??)
najdemo taki funkciji $\lambda, \mu$, da bo veljalo \begin{align*}
  \operatorname{grad} \xi (u,v) &= \frac{1}{\lambda(u,v)}(A(u,v), C(u,v)), \\
  \operatorname{grad} \eta (u,v) &= \frac{1}{\mu(u,v)}(B(u,v), D(u,v)).
\end{align*}
\qed

\begin{definicija}
\label{def_premnosna_ploskev}
Naj bo $X \subseteq  \mathbb{R}^3$ ploskev, $\gamma: [a,b] \to X$ krivulja in $\tilde{a}: \mathbb{R}^3 \to \mathbb{R}^3$ vektorsko polje. 
Vektorskemu polju $\tilde{a}$ priredimo vektorsko polje vzdož krivulje $\gamma$ s predpisom $a(u) := \tilde{a}(\gamma(u))$. 
Ploskev $X$ je premonosna ploskev, podana s krivuljo $\gamma$ in vektorskim poljem $a$ vzdolž krivulje $\gamma$, za katero obstaja parametrizacija \begin{equation*}
\rho(u) = \gamma(u) + va(u).
\end{equation*}Če za polje $a$ velja $a(u) = \dot{\gamma}(u)$, potem je ploskev tangetno premonosna.
\end{definicija}


\begin{lema}
\label{lem_tangentno_premonosna_mesani_produkt}
Naj bo $X$ premonosna ploskev, podana s krivuljo $\gamma(u)$ in vektorskim poljem $a(u)$ vzdolž krivulje $\gamma(u)$. Ploskev $X$ je
tangetno premonosna natanko tedaj, ko velja pogoj na mešani produkt \begin{equation*}
\left[ \gamma'(u), a(u), a'(u) \right] = 0.
\end{equation*}  
\noindent
\end{lema}

\noindent
{\em Dokaz:\/}
Dokazati bo treba samo implikacijo v levo smer. Naj bo $X$ premonosna ploskev s parametrizacijo \begin{equation*}
r(u,v) = \gamma(u) + v a(u)
\end{equation*} in naj velja zgornji pogoj na mešani produkt.
Ploskev $X$ bo tangetno premonosna, če nam uspe poiskati takšno krivuljo $\rho(u)$ ter funkciji $f$ in $g$, da velja 
\begin{equation*}
  \rho(u) = \gamma(u) + f(u)a(u),
  \end{equation*}hkrati pa je izpoljnjen še pogoj na vzporednost odvoda krivulje $\rho(u)$ s poljem $a(u)$ \begin{equation*}
    \rho'(u) = g(u) a(u).
    \end{equation*} 
Z odvajanjem $\rho$ dobimo enačbo, ki ji morata zadoščati $f$ in $g$ \begin{equation*}
\rho' = \gamma' + f'a + f a' \implies \gamma' = (g - f') a - fa'.
\end{equation*}  
Iz pogoja na mešani produkt vemo, da so vektorji $\gamma', a, a'$ koplanarni. Zato lahko $\gamma'$ izrazimo kot linearno kombinacijo ostalih dveh, torej
obstajata funkciji $\alpha, \beta$, da velja
\begin{equation*}
\gamma' = \alpha a + \beta a'.
\end{equation*}  
Rešitev zgornjega sistema enačb sta funkciji \begin{align*}
    f &= -\beta, \\
    g &= \alpha + \beta', 
\end{align*}
kar pomeni, da funkcija $\rho$ obstaja, torej je $X$ res tangentno premonosna ploskev.
\qed

\begin{opomba}
V dokazu bi lahko namesto \begin{equation*} r(u,v) = \gamma(u) + v a(u) \end{equation*} pisali 
\begin{equation*} r(u,v) = \gamma(u) + \lambda(v) a(u) \end{equation*} za neko monotono funkcijo $\lambda$.
Monotonost bi privzeli zato, da bi dobili smiselno parametrizacijo.
\end{opomba}

Zdaj se lotimo še dokaza izreka.

\noindent
{\em Dokaz izreka \href{izr_plsokev_je_ploscata_ce_je_del_valja_ali_tangentnopremosnosne_ploskve}:\/}
Predpostavimo, da za vsak $m \in X$ velja $\kappa(m) = 0$, kar brez škode za splošnost implicira $\kappa_2(m) = 0$. Če bi bila za vsak $m \in X$ povprečna vrednost $H(m) = \frac{1}{2}(\kappa_1(m) \kappa_2(m))= 0$ je $X$ ravnina. Predpostavimo torej, da obstaja
$k \in  X$, da je $H(k) \neq 0$.
Po lemi \href{lem_obstoj_funkcij} vemo, da obstaja parametrizacija $r(u,v): V \to X$, za katero velja $r_u = \lambda e_1$, $r_v = \mu e_2$. 
Za takšno parametrizacijo velja $F = \langle r_u, r_v \rangle = \lambda \mu \langle e_1, e_2 \rangle = 0$, torej bo prva fundamentalna forma te parametrizacije enaka \begin{equation*}
I = \begin{pmatrix}
E & 0 \\
0 & G
\end{pmatrix}.
\end{equation*}  
Vemo pa tudi, da sta glavni ukrivljenosti rešitvi enačb \begin{equation*} \det \left( \begin{pmatrix}
  L & M \\
  M & N
  \end{pmatrix} - \kappa_i \begin{pmatrix}
  E & F \\
  F & G
  \end{pmatrix}  \right) = 0. \end{equation*}
V našem primeru je za $i = 2$ \begin{equation*}
  \det  \begin{pmatrix}
    L & M \\
    M & N
    \end{pmatrix} = 0 \implies LN - M^2 = 0.
\end{equation*}  
Spomnimo se, da je normalna ukrivljenost podana s predpisom \begin{equation*}
\kappa_n(\xi, \eta) = \begin{pmatrix}
    \xi & \eta
\end{pmatrix} \begin{pmatrix}
L & M \\
M & N
\end{pmatrix} \begin{pmatrix}
  \xi \\ \eta
\end{pmatrix},
\end{equation*}  
v naši parametrizaciji pa je smer, ki pripada glavni ukrivljenosti $\kappa_2$ v bazi $\left\{ r_u, r_v\right\}$, podana z $\begin{pmatrix}
    0 & \alpha
\end{pmatrix}^{T}.$
Velja torej \begin{equation*}
  \kappa_2 = \begin{pmatrix}
      0 & \alpha
  \end{pmatrix} \begin{pmatrix}
  L & M \\
  M & N
  \end{pmatrix} \begin{pmatrix}
    0 \\ \alpha
  \end{pmatrix} = N \alpha^2.
\end{equation*}
Ker lahko izberemo neničelen $\alpha$, to pomeni, da je $N = 0.$ Iz pogoja $LN - M^2 = 0$ dobimo še $M$ = 0, torej za drugo fundamentalno formo naše parametrizacije $X$ velja \begin{equation*}
II = \begin{pmatrix}
L & 0 \\
0 & 0
\end{pmatrix}.
\end{equation*}  
Zdaj trdimo, da je $(e_2)_v = 0$. Zato bomo vektorsko polje $(e_2)_v$ razvili po bazi $\left\{ e_1, e_2, n\right\}$. Najprej bomo po tej bazi
razvili vekotorja $n_u$ in $n_v$. Pri tem se zavedajmo, da je $e_1 = \frac{r_u}{\lvert\lvert r_u \rvert\rvert }$ in $e_2 = \frac{r_v}{\lvert\lvert r_v \rvert\rvert }$.
Dobimo \begin{equation*}
n_u = \langle n_u, e_1 \rangle e_1 + \langle n_u, e_2 \rangle e_2 + \langle n_u, n \rangle n. 
\end{equation*}  
Iz zveze $\langle n, r_u \rangle = 0$ s parcialnim odvajanjem po $u$ dobimo \begin{equation*}
  \langle n_u, r_u \rangle = - \langle n, r_{uu} \rangle  = -L.
\end{equation*}  
Torej velja \begin{equation*}
\langle n_u, e_1 \rangle = -\frac{1}{\lvert\lvert r_u \rvert\rvert } \langle n, r_{uu} \rangle = -\frac{L}{\sqrt{E}}.   
\end{equation*}  
Na enak način dobimo \begin{equation*}
\langle n_u, e_2 \rangle = -\frac{1}{\lvert\lvert r_v \rvert\rvert } \langle n, r_{uv} \rangle = -\frac{M}{\sqrt{G}} = 0.
\end{equation*}  
Iz parcialnega odvajanja zveze $\langle n, n \rangle = 1$ po $u$ pa dobimo še $\langle n_u, n \rangle = 0$. Torej velja, da je \begin{equation*}
n_u = -\frac{L}{\sqrt{E}} e_1 \implies e_1 =-\frac{\sqrt{E} }{L} n_u.
\end{equation*}  
Z enakim postopkom dobimo, da je v tej bazi $n_v = 0$. Tam je namreč $\langle n_v, r_v \rangle = - N = 0.$
Zdaj izrazimo še vektor $(e_2)_v$ po tej bazi. Po enakih idejah kot prej dobimo \begin{align*}
  \langle (e_2)_v, e_1 \rangle &=  -\frac{\sqrt{E} }{L} \langle (e_2)_v, n_u \rangle = \frac{\sqrt{E}}{L} \langle e_2, (n_v)_u \rangle = 0.  \\
  \langle (e_2)_v, e_2 \rangle   &= 0 \\
  \langle (e_2)_v, n \rangle &= (\langle e_2, n \rangle)_v - \langle e_2, n_v \rangle = 0   
\end{align*}
Torej smo pokazali, da je $(e_2)_v = 0$. Zdaj dokažimo, da je $X$ tangentno premonosna ploskev s premičjem, napetim
na vektorsko polje $e_2$. Zato si oglejmo krivulje $v \mapsto r(u_0, v)$, kjer je točka $u_0$ fiksna. V naši parametrizaciji velja,
da je $r_v = \mu e_2$. Zato je \begin{equation*}
r_{vv} = \mu_v e_2 + \mu \underbrace{ (e_2)_v}_0  = \mu_v e_2.
\end{equation*}  
To pomeni, da bosta v vsaki točki krivulje $v \mapsto r(u_0, v)$ vektorja hitrosti in pospeška vzporedna. Torej bo ta krivulja premica. Naj bo sedaj
$\gamma(u) = r(u, v_0)$. Naša ploskev je premonosna, lahko jo parametriziramo s parametrizacijo \begin{equation*}
r(u,v) = \gamma(u) + f(v) \underbrace{r_v(u, v_0)}_{a(u)}. 
\end{equation*}  
Zdaj izračunajmo mešani produkt \begin{equation*}
\left[ \frac{ \partial \gamma }{ \partial u }, a , \frac{ \partial a }{ \partial u }  \right] = \left[ r_u, r_v, r_{uv} \right]
= \langle r_u \times  r_v, r_{uv} \rangle = \lvert\lvert r_u \times  r_v \rvert\rvert \langle n, r_{uv} \rangle
= \lvert\lvert r_u \times r_v \rvert\rvert \underbrace{M}_0  = 0.
\end{equation*}  
Po lemi \href{lem_tangentno_premonosna_mesani_produkt} sledi, da je $X$ tangentno premonosna ploskev.
\qed

\begin{opomba}
Gaussova ukrivljenost krogle z radijem $R$ je $\frac{1}{R^2}$. To ustreza naši intuiciji, da je ping pong žogica bolj ukrivljena od Zemlje.
\end{opomba}