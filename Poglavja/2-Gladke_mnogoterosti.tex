\section{Gladke vložene ploskve}%
\label{sec:Gladke_vložene_ploskve}

V splošnem bi lahko mnogoterosti obravnavali kot abstraktne
matematične strukture, ki ne prebivajo nujno v evklidskih
prostorih. Pri uvodu v diferencialno geometrijo pa se bomo v
glavnem ukvarjali z eno in dvodimenzionalnimi mnogoterostmi,
vloženimi v prostor $\mathbb{R}^3$. 

\begin{definicija}
\label{def_gladka_vložena_ploskev} Množica
$X \subseteq \mathbb{R}^3$ je gladka vložena ploskev, če
za vsak $m \in  X$ obstaja krogla za $m$ $W \subseteq
\mathbb{R}^n$ in gladka funkcija $f : W \to \mathbb{R}$,
za katero velja 
\begin{enumerate} \item $X \cap W =
f^{*}\left( \left\{ 0\right\}  \right)$ \item $\left(
Df \right)_w \neq 0 $ za vsak $w \in X \cap  W$  
\end{enumerate}    
\end{definicija}

Vložena ploskev $X \subseteq  \mathbb{R}^n$ je tudi abstraktna
mnogoterost. Poglejmo si, kako bi konstruirali atlas na $X$.
Vzemimo točko $m \in  X$. Po definiciji vložene ploskve
obstaja nivojnica $f: W \ni m \to \mathbb{R}$ in vemo, da
$D_mf = \left( \frac{ \partial f }{ \partial x } , \frac{
\partial f }{ \partial y }  , \frac{ \partial f }{ \partial z
} \right)\left( m \right) \neq 0$. Zdaj se spomnimo izreka o
implicitni funkciji. Naj bo $m = \left( x_0, y_0 , z_0 \right)$ in
BŠS naj bo $\frac{ \partial f }{ \partial z }\left( m \right) \neq
0$. Torej obstaja gladka okolica $V \ni \left( x_0, y_0 \right)
\subseteq \mathbb{R}^2$  in gladka funkcija $g: V \to \mathbb{R}$,
da velja $f\left( x, y, g\left( x,y \right) \right) = 0$  za vsak
$\left( x,y \right) \in  V$. Po potrebi lahko množico $W$
zmanjšamo na $W_0 \subseteq  W$, da dobimo difeomorfizem
\begin{align*} r: V &\longrightarrow W_0 \cap  X \\ \left( x,y
\right) &\longmapsto \left( x,y,g\left( x,y \right) \right)
\end{align*} z inverzom \begin{align*} \varphi: W_0 \cap X
&\longrightarrow V \\ \left( x,y,z \right)
&\longmapsto \left( x,y \right). \end{align*}Ta inverz
je v bistvu projekcija na prvi dve koordinati. Če
definiramo $U = W_0 \cap  X$, postane par $\left( U,
\varphi \right)$ karta na $X$.  

\subsection{Metrika na ploskvi}%

Če hočemo meriti razdalje med pari točk na gladki
mnogoterosti, potrebujemo še dodatno strukturo --
metriko. Ta nam omogoča merjenje dolžin krivulj. Če si
predstavljamo krivuljo $\gamma : \left( a,b \right)
\to M$, je najbolj naravna definicija njene dolžine \[
\mathcal{L}\left( \gamma \right) = \int_{a}^{b}
\lvert\lvert \dot{\gamma } \left( t \right) \rvert\rvert
\, dt. \]Znati moramo torej izračunati dolžino oziroma normo
tangentnega vektorja. Najbolje je, če je ta norma porojena s
skalarnim produktom, torej $\lvert\lvert x \rvert\rvert =
\sqrt{\langle x,x \rangle } $.

Naj bo $\langle \cdot , \cdot  \rangle $  neki skalarni
produkt na $\mathcal{V} = \mathbb{R}^n$ in naj bo $\left\{
v_1, \ldots , v_{n} \right\}$  baza za $\mathcal{V}$, ki ni
nujno ortonormirana. Vzemimo vektorja $\vec{x}  =
\sum_{i=1}^{n} a_{i}v_{i}$ in $\vec{y}  = \sum_{i=1}^{n}
b_{i}v_{i}.$ Potem velja, da je skalarni produkt enak \[
\langle \vec{x} , \vec{y} \rangle = \sum_{i,j = 1}^{n}
a_{i}b_{j}\langle v_{i}, v_{j} \rangle =
\begin{pmatrix}
a_1 & \ldots & a_n
\end{pmatrix}
\begin{pmatrix}
\langle a_1, a_1 \rangle & \dots & \langle a_1, a_n
\rangle \\ \vdots & \ddots & \vdots \\ \langle a_n,
a_1 \rangle & \dots & \langle a_1,a_n \rangle
\end{pmatrix}
\begin{pmatrix} b_1 \\ \vdots \\ b_n
\end{pmatrix}. \]
Iz simetričnosti skalarnega produkta
($\langle v_{i}, v_{j} \rangle  = \langle v_{j},  v_{i}
\rangle $) sledi, da je zgornja matrika simetrična. Iz
pozitivne definitnosti skalarnega produkta ($\langle
v_{i}, v_{i} \rangle > 0$) pa sledi še pozitivna 
definitnost te matrike. 

\begin{opomba}
Kvadratne matrike so lahko koordinatni
zapisi linearnih preslikav iz $\mathbb{R}^n \to
\mathbb{R}^n$, lahko pa so tudi koordinatni zapisi
skalarnih produktov. To je odvisno od tega, kako se
matrike transformirajo pri prehodu v različno bazo.

Naj bo $P$ poljubna preslikava med bazama, $L_e$
linearna preslikava glede na bazo $\left\{ e_1, \ldots
, e_{n}\right\}$, $L_f$ pa glede na bazo $\left\{ f_1,
\ldots, f_{n}\right\}.$ Potem iz algebre 1 vemo, da je \[
L_f = PL_eP^{-1}. \]Zdaj pa izpeljimo, kako se
transformira matrika skalarnega produkta. Naj bosta
$a_f = Pa_e$ in $b_f = Pb_e$. Potem dobimo iz enakosti
\begin{align*} \langle a_f, b_f \rangle &= \langle
a_e, b_e \rangle \\ a_f^{T} A_f b_f &= a_e^{T} A_e
b_e \\ a_e^{T} P^{T} A_f P b_e &= a_e^{T} A_e b_e,
\forall a_e, b_e. \end{align*} Od tod sledi, da je
$P^{T}A_fP = A_e$ oziroma zaradi ortogonalnosti
$P$ ekvivalentno \[A_f = PA_eP^{T}.\]

Torej transformacijska pravila določajo vrsto
preslikave, podobno kot pri fiziki. 
\end{opomba}

Preden se lotimo definicije tangentne ravnine, se spomnimo naslednje
definicije. 

\begin{definicija}
\label{def_rang_preslikave}
 Naj bo preslikava $F: W \subseteq  \mathbb{R}^n \to  \mathbb{R}^m$ odvedljiva.
 Rang preslikave $F$ v točki $w \in  W$ je enak rangu matrike $D_wF$. Pravimo,
 da ima $F$ v točki $w \in  W$ maksimalen rang, če ima matrika $D_wF$
 maksimalen rang.
\end{definicija}


\begin{definicija} 
	\label{def_tangentna_ravnina}
	Naj bo $X \subseteq \mathbb{R}^3$ vložena ploskev in točka $m
\in X$. Tangentna ravnina $T_mX$ je množica tangent
vseh krivulj v $X$, ki v času $t = 0$ gredo skozi $m$.
\[ T_mX = \left\{ \dot{\gamma}\left( 0 \right) \mid
\gamma : \left(  - \varepsilon, \varepsilon \right)
\to  X \subseteq  \mathbb{R}^3 \text{ krivulja, } \gamma\left(
0 \right) = m\right\} \]
\end{definicija}


\begin{trditev}
\label{trd_tangentna_ravnina_je_dvodimenzionalen_vektorski_prostor}
$T_mX$ je dvodimenzionalen realni vektorski podprostor v
$\mathbb{R}^3$. 
\end{trditev} 
{\em Dokaz:\/} Naj bo $r : V
\subseteq  \mathbb{R}^2 \to X \subseteq \mathbb{R}^3$ neka regularna
parametrizacija ploskve $X$ (to pomeni, da mora biti rang preslikave
$r$ maksimalen, torej konstantno enak 2) v okolici točke  $m \in  X$. Naj
bo $p = \left( u, v \right) \in  V \subseteq \mathbb{R}^2.$ Pišimo \[ r\left(
p \right) = r\left( u,v \right) = 
\begin{pmatrix} x\left( u,v
\right) \\ y\left( u,v \right) \\ z\left( u,v \right)
\end{pmatrix} . \]
Naj bo $m = r\left( u_0, v_0 \right).$ Trdimo, da
je $ T_mX = \operatorname{im} \left( D_{(u_0, v_0)}r \right) $.
Najprej dokažimo inkluzijo $T_mX \subseteq \operatorname{im} \left(
D_{(u_0, v_0)}r \right)$. 
Naj bo $\gamma: \left( -\varepsilon,
\varepsilon \right) \to X \subseteq \mathbb{R}^2,$ $\gamma \left( 0
\right) = m = r\left( u_0, v_0 \right)$ poljubna krivulja. Direktno
po definiciji tangentne krivulje sledi $\dot{\gamma\left( 0 \right)}
\in T_mX$. 
Dokazati moramo $\dot{\gamma(0)} \in \operatorname{im} \left(D_{\left( u_0, v_0 \right)} r \right)$. Naj bo
$\gamma\left( t \right) : \left( -\varepsilon, \varepsilon \right)\to
V$ podana z $\beta\left( t \right) = r^{-1}\left( \gamma\left( t
\right) \right)$. Ker je praslika preslikave $\beta$ vsebovana v
$\mathbb{R}^2,$ obstajata funkciji $u\left( t \right), v\left( t
\right)$, da je $\beta\left( t \right) = \left( u(t), v(t) \right).$
Pri tem velja, da je $\beta(0) = (u(0), v(0)) = (u_0, v_0).$ Vidimo,
da je $\gamma(t) = r(u(t), v(t)).$ Po verižnem pravilu za odvajanje
imamo \[ \dot{\gamma }(0)= \frac{d}{dt} \big|_{t = 0} \gamma(t) =
\left( D_{ (u_0, v_0)}r \right) \dot{\beta} (0).\]Torej je
$\dot{\gamma} (0) \in \operatorname{im} \left( D_{(u_0,v_0)}r
\right).$ 

Nato dokažimo še obratno inkluzijo $T_mX \supseteq
\operatorname{im} \left( D_{(u_0, v_0)}r \right)$. Vzemimo poljuben
vektor $\omega \in \mathbb{R}^2$ in naj bo $v = \left( D_{(u_0, v_0)r} \cdot w \right)$. Potrebujemo krivuljo $\gamma(t),
\gamma(0) = m,$ za katero bo veljalo $\dot{\gamma} (0) = v.$
Oglejmo si \[ \left( D_{(u_0, v_0)}r \right)(\dot{\beta} (0)) =
\frac{d}{dt} \big|_{t = 0} r(\beta(t)). \]  Trdimo, da za $\gamma(t) =
r(\beta(t))$  velja $\dot{\gamma}(0) \in  T_mX.$ To
je res, saj je $\gamma(t): (-\varepsilon,
\varepsilon)\to  X \subseteq \mathbb{R}^3$, hkrati pa tudi $\gamma(0) = r(\beta(0) )= r(u_0, v_0) = m$. Torej velja,
da je $T_mX = \operatorname{im}(D_{(u_0,v_0)}r)$. Ker smo zahtevali,
da je parametrizacija regularna, je matrika $D_{(u_0,v_0)}r$ ranga 2,
torej je  $T_mX$ dvodimenzionalen vektorski prostor.
\qed

\begin{opomba}
 Tangentna ravnina je pravi vektorski prostor in ne afin kot
 recimo pri analizi 2a.
\end{opomba}

Do nadaljnjega nas bodo zanimale lokalne lastnosti ploskev, zato bomo
delali v glavnem s ploskvami, ki jih lahko pokrijemo z eno samo karto
oziroma z eno samo parametrizacijo.

\begin{definicija}
\label{def_metrika_na_ploskvi}
 Metrika na ploskvi $X \subseteq  \mathbb{R}^3$, opremljeni s
 parametrizacijo $r: V \subseteq  \mathbb{R}^2 \to  X \subseteq
 \mathbb{R}^3$, je preslikava \begin{align*}
 	g: X &\longrightarrow M_2(\mathbb{R}) \\
 	m &\longmapsto 
	\begin{pmatrix}
		g_{11}(m) & g_{12}(m) \\\
		g_{21}(m) & g_{22}(m) \\
	\end{pmatrix},
 \end{align*}
 kjer je za vsak $m \in  X$ matrika $g(m)$ simetrična in pozitivno definitna. To lahko povemo s pogojema
 $\det g(m) > 0$ in $g_{11}(m) >0.$

\end{definicija}
\begin{opomba}
 Za drugo parametrizacijo ploskve $X$ bi dobili druge koeficiente
 matrike.
\end{opomba}
 Naj bo $\gamma: [a,b] \to  X$ krivulja. Njeno parametrizacijo $r$
lahko napišemo v obliki  $\gamma(t) = r(u(t), v(t))$ za primerne
funkcije $u,v : [a,b] \to \mathbb{R}$, $\beta(t) = (u(t), v(t))$,
$\gamma(t) = r(\beta(t)).$ V koordinatah lahko zapišemo 

\[ \gamma(t) =  \begin{pmatrix}x(u(t), v(t))\\ y(u(t), v(t)) \\
z(u(t), v(t)) \end{pmatrix}, \]
\[ \dot{\gamma}(t) = \frac{d}{dt} \big|_{t = t_0} r(\beta(t)) =
(D_{(u_0, v_0)}r)(\dot{\beta}(t_0)) = 
\begin{pmatrix}
	\frac{ \partial x }{ \partial u }  & \frac{ \partial x }{ \partial v }  \\
	\frac{ \partial y }{ \partial u }  & \frac{ \partial y }{ \partial
	v }  \\
	\frac{ \partial z }{ \partial u }  & \frac{ \partial z }{ \partial v }  \\

\end{pmatrix}_{(u_0, v_0)}
\begin{pmatrix}
	\dot{u} \\ \dot{v}
\end{pmatrix}
 = r_u(u_0 , v_0)\dot{u}(t_0) + r_v(u_0 , v_0)\dot{v}(t_0). \]
 To je razvoj vektorja $\dot{\gamma}(t_0)$ po bazi $\left\{ r_u(u_0 ,
v_0), r_v(u_0 , v_0) \right\}$ prostora $T_{\gamma(t_0)}X$, ki pa ni
nujno ortogonalna. Pravzaprav je ortogonalna le v precej posebnih
primerih. 

\begin{definicija}
\label{def_dolzina_krivulje}
Dolžina krivulje $\gamma: [a,b] \to  X \subseteq  \mathbb{R}^3$ glede
na metriko $g$ je v parametrizaciji $r$ podana s formulo 
\[ \mathcal{L}_g(\gamma) = \int_{a}^{b} \sqrt{ 
\begin{pmatrix}
    \dot{u}(t) & \dot{v}(t) \\
\end{pmatrix}
\begin{pmatrix}
	g_{11} & g_{12}  \\
	g_{21} & g_{22} \\
\end{pmatrix}
\begin{pmatrix}
	\dot{u}(t) \\
	\dot{v}(t) \\
\end{pmatrix}
}  \, dt  \]
\end{definicija}

Ustrezni skalarni produkti na ravnini $T_mX$ so glede na
parametrizacijo $r$ podani s predpisi $\left<r_u, r_u \right>_g  =
g_{11}$, $\left<r_u, r_v \right>_g  =
g_{12}$, $\left<r_v, r_v \right>_g  =
g_{22}$. Naj bo sedaj ambientni prostor $\mathbb{R}^3$ opremljen s
fiksnim evklidskim skalarnim produktom, in koeficiente $g_{ij}$
poračunamo z njim (na enak način kot prej). Pri tem uporabimo
naslednje standardne oznake: 
\[ E(u,v) = \left<r_u(u,v), r_u(u,v) \right>,   \,\,\,   F(u,v) =  \left<r_u(u,v),
r_v(u,v) \right>, \,\,\,     G(u,v) =  \left<r_v(u,v), r_v(u,v)
\right>.  \]
Včasih tudi zlorabimo notacijo 
\[ E(m) = E(r(u,v)) = E(u,v). \]

\begin{definicija}
\label{def_prva_fundamentalna_forma}
 Metrika na $X \subseteq  \mathbb{R}^2$, ki je glede na $r: V \to  X
\subseteq  \mathbb{R}^3$ podana z matrično funkcijo \begin{align*}
 	g_f: V &\longrightarrow M_2(\mathbb{R}) \\
	(u,v) &\longmapsto 
\begin{pmatrix}
	E & F \\
	F & G \\
\end{pmatrix}
\begin{pmatrix}
	u \\
	v \\
\end{pmatrix},
\end{align*}
se imenuje prva fundamentalna forma ploskve.
\end{definicija}

\begin{opomba}
 Dolžina krivulje $\mathcal{L}(\gamma)$ glede na prvo fundamentalno
 formo ploskve sovpada z običajno dolžino krivulje: \begin{align*}
    \mathcal{L}(\gamma)  = \int_{a}^{b} \lvert\lvert \dot{\gamma}(t) \rvert\rvert  \, dt  &=
	 \int_{a}^{b} \sqrt{\left<\dot{u}r_u + \dot{v}r_v, \dot{u}r_u + \dot{v}r_v \right> }  \, dt \\
        &= \int_{a}^{b} \sqrt{ 
\begin{pmatrix}
	\dot{u}(t) & \dot{v}(t) \\
\end{pmatrix}
\begin{pmatrix}
	\left<r_u, r_u \right> & \left<r_u, r_v \right>  \\
	\left< r_u, r_v \right>	  & \left<r_v, r_v \right> \\
\end{pmatrix}_{(u(t), v(t))}
\begin{pmatrix}
	\dot{u}(t) \\
	\dot{v}(t) \\
\end{pmatrix}
}  \, dt  \\
 &=  \int_{a}^{b} \sqrt{ 
\begin{pmatrix}
	\dot{u}(t) & \dot{v}(t) \\
\end{pmatrix}
\begin{pmatrix}
	E & F \\
	F & G \\
\end{pmatrix}_{(u(t), v(t))}
\begin{pmatrix}
	\dot{u}(t) \\
	\dot{v}(t) \\
\end{pmatrix}
}  \, dt.  \\
 \end{align*}
\end{opomba}

Izomorfizmi v diferencialni geometriji so izometrije, katerih
definicija pa je nekoliko drugačna, kot bi morda pričakovali.

\begin{definicija}
\label{def_izometrija}
 Preslikava $f: X \to \tilde{X}$ nad dvema ploskvama $X, \tilde{X} \subseteq
 \mathbb{R}^3$ je izometrija, če za vsako krivuljo $\gamma: \left[ a,b
 \right] \to  X$ velja enakost med dolžinama 
 \[ \mathcal{L}_X(\gamma) = \mathcal{L}_{\tilde{X}}(f(\gamma)). \]
\end{definicija}

\begin{opomba}
 Izometrije med ploskvama porodijo izometrije v običajnem metričnem
 smislu.
\end{opomba}

\begin{primer}
 1. fundamentalna forma na sferi glede na sferične koordinate. Če
 odvzamemo iz $S^2$ en poldnevnik, jo lahko parametriziramo s
 sferičnimi koordinatami: 
 \[ r: \begin{pmatrix}
 	u \\ v
 \end{pmatrix} \subseteq  V = \left( -\pi, \pi \right) \times \left(
 0, \pi  \right) \subseteq  \mathbb{R}^2 \to  \mathbb{R}^3
  \] s predpisom \[ r(u,v) = 
\begin{pmatrix}\cos u \cos v \\ \cos u \sin v \\ \sin u
\end{pmatrix}.\]
Potem dobimo parcialna odvoda 
\[   r_u(u,v) = \begin{pmatrix}-\sin u \cos v \\ -\sin u \sin v \\ \cos u
\end{pmatrix}, \,\,  r_v(u,v) = 
\begin{pmatrix}-\cos u \sin  v \\ \cos u \cos v \\ 0
\end{pmatrix}.\]
Od tod sledi 
\[ \begin{align*}
    E &= \langle r_u, r_u \rangle = 1, \\
    F &= \langle r_u, r_v \rangle = 0, \\
	G &= \langle r_v, r_v \rangle = \cos^2 u,
\end{align*} \] kar lahko zapišemo v obliki 
\[ 
\begin{pmatrix}
	E & F \\
	F & G \\
\end{pmatrix}
=
\begin{pmatrix}
	1 & 0 \\
	0 & \cos^2 u \\
\end{pmatrix}
.
\]
\end{primer}

\begin{primer}
 Naj bo $X$ rotacijska ploskev, ki jo dobimo, če krivuljo $x =  f(z)$
 zavrtimo okoli osi $z$: 
 \[ r: \left( u,v \right) \mapsto  
\begin{pmatrix}
	f(u) \cos v \\
	f(u) \sin v \\
	u \\
\end{pmatrix}
.\]Potem imamo odvoda 
\[   r_u(u,v) = \begin{pmatrix} f'(u) \cos v \\ f'(u) \sin v \\ 1
\end{pmatrix}, \,\,  r_v(u,v) = 
\begin{pmatrix}-f(u) \sin  v \\ f(u) \cos v \\ 0
\end{pmatrix}.\]
Od tod sledi 
\[ \begin{align*}
    E &= \langle r_u, r_u \rangle = 1 + f'(u)^2, \\
    F &= \langle r_u, r_v \rangle = 0, \\
	G &= \langle r_v, r_v \rangle = f(u)^2,
\end{align*} \] kar lahko zapišemo v obliki 
\[ 
\begin{pmatrix}
	E & F \\
	F & G \\
\end{pmatrix}
=
\begin{pmatrix}
	1 + f'(u)^2 & 0 \\
	0 & f(u)^2 \\
\end{pmatrix}
.
\]
\end{primer}

Naslednji izrek nam pove povezavo med 1. fundamentalno formo in
izometričnostjo ploskev.

\begin{izrek}
\label{izr_izometricnost_in_prva_forma}
Naj bo $X \to  \tilde{X}$ izometrija med ploskvama. Tedaj obstaja par parametrizacij
$r : V \to X$ in $\tilde{r}: V \to  \tilde{X}$, da za pripadajoči
fundamentalni formi velja 
\[ 
\begin{pmatrix}
	E & F \\
	F & G \\
\end{pmatrix}_{(u,v)}
=
\begin{pmatrix}
	\tilde{E}  & \tilde{F}  \\
	\tilde{F}  & \tilde{G}  \\
\end{pmatrix}_{(u,v)}
.\]
Velja tudi obratno, torej če za ploskvi $X, \tilde{X} $ obstajata
parametrizaciji $r$ in $\tilde{r}$, za kateri velja zgornji sistem
enačb, potem sta $X$ in $\tilde{X}$ izometrični.
\end{izrek}
{\em Dokaz:\/}
Pokažimo najprej obrat $(\impliedby)$. Recimo, da obstajata parametrizaciji $r: V
\to  X$ in $\tilde{r}: V \to  \tilde{X} $, da velja 
\[ 
\begin{pmatrix}
	\tilde{E}  & \tilde{F}  \\
	\tilde{F}  & \tilde{G}  \\
\end{pmatrix}_{(u,v)}
.\] 
% TODO ali je res smisleno, da je indeks _r? Verjetno je bolj smiselno, da je _X, ker je dolžina itak neodvisna od parametrizacije  
Naj bo $f = \tilde{r} \circ  r^{-1}$. Preveriti moramo, da je $f$
izometrija. Naj bo $\gamma: [a,b] \to  X$ poljubna krivulja in
primerjamo dolžini $\mathcal{L}_r(\gamma)$ ter
$\mathcal{L}_{\tilde{r}}(f(\gamma))$. Potem obstaja krivulja $\beta:
[\alpha,\beta] \to  V$ za katero velja $\gamma(t) = r(\beta(t))$.
Potem za $f(\gamma(t))$ velja 
\[ f(\gamma(t)) = f(r(\beta(t))) = \tilde{r}(\beta(t)).\]
Ker je $\beta(t)$ ravninska krivulja, velja 
\[ \begin{align*}
    \gamma(t) &= r(\beta(t)) = r(u(t), v(t)), \\
    \tilde{\gamma}(t) &= \tilde{r} (\beta(t)) = \tilde{r} (u(t), v(t)).   
\end{align*} \]
Torej imamo enačbi 
\[ \mathcal{L}_r(\gamma) = \int_{a}^{b} \sqrt{E(u,v)\dot{u}^2 +
2F(u,v)\dot{u}\dot{v} + G(u,v)\dot{v}^2} \, dt, \]
\[ \mathcal{L}_{\tilde{r}}(\tilde{\gamma}) = \int_{a}^{b} \sqrt{\tilde{E} (u,v)\dot{u}^2 +
2\tilde{F}(u,v)\dot{u}\dot{v} + \tilde{G}(u,v)\dot{v}^2} \, dt.  \]
Ker so posamezni sumandi enaki, sta izraza enaka, torej je $f =
\tilde{r} \circ r^{-1}$ res izometrija.

Zdaj dokažimo še $(\implies)$. Denimo, da imamo med ploskvama $X$ in $\tilde{X}$
izomerijo $f: X \to  \tilde{X}.$ Naj bo $\gamma(t) = r(\beta(t)) =
(u_0 + t, v_0)$ in $t \in  (0, \varepsilon).$ Pri tem je točka $(u_0,
v_0) \in V$ poljubna in določa točko $p = r(u_0, v_0) \in  X$. Zdaj
izračunamo 
\[ 
\mathcal{L}_r(\gamma(t)) = \int_{0}^{\varepsilon} \lvert\lvert
\dot{\gamma}(t) \rvert\rvert   \, dt = \int_{0}^{\varepsilon} \sqrt{\langle
\dot{\gamma}(t), \dot{\gamma}(t)  \rangle} \, dt =
\int_{0}^{\varepsilon} \sqrt{\langle r_u, r_u \rangle}   \, dt =
\int_{0}^{\varepsilon}\sqrt{  E(u_0 + t, v_0) }  \, dt. 
\]
Sedaj si oglejmo $f(\gamma (t)) = \tilde{r}(\beta(t)) = \tilde{r}(u_0
+ t, v_0)$. Potem imamo:

\[ \begin{align*}
  \mathcal{L}_{\tilde{r}}(f(\gamma(t))) &= \int_{0}^{\varepsilon} \lvert\lvert
  \dot{\tilde{\gamma}}(t) \rvert\rvert   \, dt  \\ &= 
  \int_{0}^{\varepsilon} \sqrt{\langle\dot{\tilde{\gamma}}(t), \dot{\tilde{\gamma}}(t)  \rangle} \, dt  \\ &=
  \int_{0}^{\varepsilon} \sqrt{\langle \tilde{r}_u(u_0 + t, v_0), \tilde{r}_u(u_0 + t, v_0) \rangle}   \, dt \\ &= 
  \int_{0}^{\varepsilon} \tilde{r}(u_0 + t, v_0)  \, dt. 
\end{align*} \]
Po predpostavki o izometričnosti velja 
\[ \begin{align*}
    \mathcal{L}_r(\gamma) &= \mathcal{L}_{\tilde{r}}(f(\gamma)), \\
    \int_{0}^{\varepsilon} \sqrt{E(u_0 + t, v_0)}   \, dt  &= \int_{0}^{\varepsilon} \sqrt{\tilde{E}(u_0 + t, v_0)} \, dt.
\end{align*} \]
Po izreku o povprečni vrednosti obstajata neka $\hat{t}, \hat{\hat{t}} \in  (0, \varepsilon)$, da velja
\[ E(u_0 + \hat{t} , v_0) \varepsilon = \tilde{E}(u_0 + \hat{\hat{t}} , v_0) \varepsilon\]in če pošljemo $\varepsilon \to  0$,
zaradi zveznosti funkcij dobimo 
\[ E(u_0, v_0) = \tilde{E}(u_0, v_0). \]
Še lažje ta rezultat dobimo tako, da na obeh straneh odvajamo po $\varepsilon$ in vstavimo $\varepsilon = 0$.

Če zdaj vzamemo $\beta(t) = (u_0, v_0 + t)$, dobimo po enakem postopku kot prej 
\[ G(u_0, v_0) = \tilde{G}(u_0, v_0). \]
Nato vzamemo $\beta(t) = (u_0 + t, v_0 + t)$ in imamo
\[ \begin{align*}
  \mathcal{L}_r(\gamma) &= \int_{0}^{\varepsilon} \sqrt{E(u_0 + t, v_0 + t) + 2F(u_0 + t, v_0 + t) + G(u_0 + t, v_0 + t)}   \, dt  \\
  = \mathcal{L}_{\tilde{r}}(\tilde{\gamma}) &= \int_{0}^{\varepsilon} \sqrt{\tilde{E}(u_0 + t, v_0 + t) + 2\tilde{F}(u_0 + t, v_0 + t) + \tilde{G}(u_0 + t, v_0 + t)}   \, dt. 
\end{align*} \]

Če zdaj zopet odvajamo po $\varepsilon$ in vstavimo $\varepsilon = 0$, dobimo enakost integrandov v točki $(u_0, v_0)$, od koder sledi še zadnja zahteva
\[ F(u_0, v_0) = \tilde{F}(u_0, v_0). \]
\qed

\begin{primer}
 Ali je stožec brez ene tvorilke izometričen kosu ravnine? Naj bo podan 
 \[ S = \left\{ (x,y,z)  \middle|\, x^2 + y^2 + z^2 = 1 \right\}.\]
 Parametriziramo ga z
 \[ r(u,v) = (u \cos v, u \sin v, u) \]in po znanem postopku dobimo 
 \[ \begin{pmatrix}
  E & F\\
  F & G\\
 \end{pmatrix}_{(u,v)} 
 = \begin{pmatrix}
  2 & 0\\
  0 & u^2\\
 \end{pmatrix}.\]
Pričakujemo, da bo $S$ izometričen nekemu krožnemu izseku, ki ga parametriziramo z 
\[ \tilde{r}(u,v) = (\alpha u \cos(\beta v), \alpha u \sin(\beta v), 0).\]Tako dobimo sistem enačb 
\[ \begin{pmatrix}
  \tilde{E} & \tilde{F}\\
  \tilde{F} & \tilde{G}\\
 \end{pmatrix}_{(u,v)} 
 = \begin{pmatrix}
  \alpha^2 & 0\\
  0 & \alpha^2 \beta^2 u^2\\
 \end{pmatrix}.\]
Pogoj  \[ \begin{pmatrix}
  E & F\\
  F & G\\
 \end{pmatrix}_{(u,v)}
 = \begin{pmatrix}
  \tilde{E} & \tilde{F}\\
  \tilde{F} & \tilde{G}\\
 \end{pmatrix}_{(u,v)}\]je izpolnjen pri $\alpha = \sqrt{2},\, \beta = \frac{1}{\sqrt{2}}$ (če bi zamenjali predznake $\alpha$ in $\beta$, bi dobili drugačne parametrizacije).
 Torej je stožec $S$ izometričen krožnemu izseku s parametrizaijo  
 \[ \tilde{r}(u,v) = \left(\sqrt{2}  u \cos(\frac{1}{\sqrt{2}} v), \sqrt{2}  u \sin(\frac{1}{\sqrt{2}} v), 0 \right).\]
\end{primer}

Na tej točki se pojavi naravno vprašanje: ali znamo poiskati vse ploskve v $\mathbb{R}^3$, ki so izomertične ravnini?
Izkaže se, da znamo, saj velja naslednji izrek.

\begin{izrek}
\label{izr_izometricnost_ploskev_ravnini}
  Naj bo ploskev $X$ izometrična kakšnemu kosu ravnine. Potem je $X$ bodisi stožec, valj, ali kakšna tangentna premonosna ploskev.
\end{izrek}

\begin{definicija}
\label{def_tangentno_premonsna_ploskev}
 Naj bo $\gamma: [a,b] \to  \mathbb{R}^3$ prostorska krivulja, parametriziarana z naravnim parametrom. Tangentna premonosna ploskev,
 podana s krivuljo $\gamma$, je del prostora $\mathbb{R}^3$, ki ga opiše tangenta na $\gamma(t)$ na intervalu $t \in  [a,b]$.
\end{definicija}

Če je $\gamma = \gamma(u)$ naravna parametrizacija, potem je smiselna parametrizacija tangentno premonosne ploskve $X$
podana z 
\[ r(u,v) = \gamma(u) + v \dot{\gamma}(u).\]Ker je $\gamma$ naravna parametrizacija, velja kot prvo 
\[ \lvert\lvert \dot{\gamma}(u) \rvert\rvert = \langle \dot{\gamma}(u), \dot{\gamma}(u) \rangle  = 1. \]Če to zvezo odvajamo, dobimo 
\[ \langle \ddot{\gamma}(u) , \dot{\gamma}(u) \rangle + \langle \dot{\gamma}(u) , \ddot{\gamma}(u) \rangle = 0  \]
in iz simetričnosti skalarnega produkta sledi 
\[ \langle \ddot{\gamma}(u) , \dot{\gamma}(u) \rangle = 0. \]
Torej je pospešek pri naravni parametrizaciji vedno pravokoten na hitrost. To lahko opazimo, če se v avtu peljemo s konstantno hitrostjo. Pospešek bomo čutili
samo v ovinkih in to pravokotno glede na smer vožnje.

\begin{definicija}
\label{def_fleksijska_ukrivljenost}
 Fleksijska ukrivljenost naravno parametrizirane krivulje $\gamma(u)$ je podana z 
 \[ \kappa(u) = \sqrt{\langle \ddot{\gamma}(u), \ddot{\gamma}(u) \rangle } = \lvert\lvert \ddot{\gamma}(u) \rvert\rvert. \]
\end{definicija}

Izračunajmo 1. fundamentalno formo tangentno premonosne ploskve. Ker velja zveza
\[ r(u,v) = \gamma(u) + v \dot{\gamma}(u),\]takoj dobimo 

\[ \begin{align*}
    r_u(u,v) &= \dot{\gamma}(u) + v \ddot{\gamma}(u), \\
    r_v(u,v) &= \dot{\gamma}(u). 
\end{align*} \]
Če od tod po znanem postopku poračunamo koeficiente 1. fundamentalne forme (pri čemer upoštevamo, da je pospešek pravokoten na hitrost), dobimo 
\[ \begin{pmatrix}
E & F \\
F & G
\end{pmatrix}_{(u,v)} = \begin{pmatrix}
  1 + v^2 \kappa^2 & 1\\
  1 & 1\\
\end{pmatrix}.\]

Torej vidimo, da je matrika 1. fundamentalne forme zares odvisna samo od fleksijske ukrivljenosti.

\begin{trditev}
\label{trd_obstoj_krivulje_glede_na_funkcijo_fleksijske_ukrivljenosti}
 Naj bo podana funkcija $\kappa(u)$. Potem obstaja naravno parametrizirana ravninska krivulja $\gamma: [\alpha, \beta] \to  \mathbb{R}^2$, katere fleksijska ukrivljenost v točki
  $\gamma(u)$ je enaka $\kappa(u)$.
\end{trditev}

{\em Dokaz:\/}
 Ker je krivulja $\gamma$ ravninska, lahko zapišemo 
 \[ \begin{align*}
     \gamma(u) &= (x(u), y(u)), \\
     \dot{\gamma}(u) &= (\dot{x}(u), \dot{y}(u)), \\ 
     \ddot{\gamma}(u) &= (\ddot{x}(u), \ddot{y}(u)).
 \end{align*} \]Ker je parametrizacija naravna, imamo še sistem enačb
 \[ \begin{align*}
  \lvert\lvert \dot{\gamma}(u) \rvert\rvert &= 1,   \\
  \lvert\lvert \ddot{\gamma}(u) \rvert\rvert &= \kappa(u) ,   \\
    \langle \ddot{\gamma}, \dot{\gamma} \rangle   &= 0.
 \end{align*} \]
 Enotski vektor, pravokoten na $\dot{\gamma}(u) = (\dot{x}(u), \dot{y}(u))$, je vektor $(\dot{y}(u), -\dot{x}(u))$. Ta vektor je vzporeden vektorju pospeška,
 torej bo za neko funkcijo $k : [\alpha, \beta]  \to \mathbb{R}$ veljalo 
 \[ (\ddot{x}(u), \ddot{y}(u)) = k(u) (\dot{y}(u), -\dot{x}(u)). \]Če obe strani enačbe normiramo, iz prejšnega sistema enačb vidimo, da mora priti natanko 
 \[ (\ddot{x}(u), \ddot{y}(u)) = \kappa(u) (\dot{y}(u), -\dot{x}(u)).\]To je sistem navadnih diferencialnih enačb 
 \[ \begin{align*}
     \ddot{x}(u) &= \kappa(u) \dot{y}(u), \\
     \ddot{y}(u) &= -\kappa(u) \dot{x}(u).
 \end{align*} \]
Pri analizi 3 smo (bomo čez 14 dni?!) dokazali eksistenčni izrek za obstoj rešitev tega sistema. Z drugimi besedami, obstaja
naravno parametrizirana krivulja $\gamma(u) = (x(u), y(u))$, za katero za vsak $u \in [\alpha, \beta]$ velja $\lvert\lvert \ddot{\gamma}(u) \rvert\rvert = \kappa(u)$.
\qed

\begin{opomba}
 Pri določenih $(u,v)$ nam krivulja $\gamma(u)$ podaja tangentno premonosno ploskev. Ta ploskev je del ravnine, v kateri
 leži krivulja krivulja $\gamma(u)$. Po izreku iz prejšnjih predavanj (TODO ali \href{izr_izometricnost_in_prva_forma} ali \href{izr_izometricnost_ploskev_ravnini}) je ta ploskev izometrična nekemu kosu ravnine.
\end{opomba}

S tem razmislekom smo dokazali izrek.

\begin{izrek}
\label{izr_tangentno_premonosna_ploskev_izometricna_kosu_ravnine}
  Vsaka tangentno premonosna ploskev je izometrična kosu ravnine.
\end{izrek}

\begin{definicija}
\label{def_ploscina_plosvke}
  Ploščina ploskve $X$, parametrizirane z $r: V sun \mathbb{R}^2 \to  X \subseteq  \mathbb{R}^3$ je podana z 
  \[ A(X) = \int_{V} \lvert\lvert r_u \times  r_v \rvert\rvert   \, du \, dv.\] 
\end{definicija}

\begin{opomba}
 Da je ta definicija dobra, moramo še preveriti.
\end{opomba}

\begin{opomba}
 Ker velja zveza
 \[ \lvert\lvert r_u \times  r_v \rvert\rvert^2  = \langle r_u, r_u \rangle \langle r_v , r_v \rangle  - \langle r_u, r_v \rangle^2 = EG - F^2, \]
lahko ploščino izrazimo tudi kot 
 \[ A(X) = \int_{V} \sqrt{EG - F^2}  \, du \, dv = \int_{V} \sqrt{\det \begin{pmatrix}
 E & F \\
 F & G
 \end{pmatrix}}  \, du \, dv. \]
\end{opomba}

\begin{definicija}
\label{def_prehodna_preslikava_med_parametrizacijama}
 Naj bosta $r: V_1 \subseteq  \mathbb{R}^2 \to  X \subseteq  \mathbb{R}^3$ in $\tilde{r}: V_2 \subseteq  \mathbb{R}^2 \to  X \subseteq  \mathbb{R}^3$ različni regularni parametrizaciji ploskve $X$.
 Potem je preslikava \begin{align*}
  g = \tilde{r}^{-1} \circ r: V_1 &\longrightarrow V_2 \\
  (u,v) &\longmapsto (\tilde{u}(u,v), \tilde{v}(u,v)),
 \end{align*}
 prehodna preslikava med parametrizacijama $r$ in $\tilde{r}$.
\end{definicija}

Velja $r(u,v) = \tilde{r}(g(u,v))$. Poglejmo si, kaj se zgodi z matriko prve fundamentalne forme transformaciji med parametrizacijama.

\subsection{Transformacijska pravila za I-forme}

Naj bosta $r: V_1 \to  X$ in $\tilde{r}: V_2 \to  X$ dve parametrizaciji iste ploskve. Med njima velja $r(u,v) = \tilde{r}(\tilde{u}(u,v), \tilde{v}(u,v))$.
To nam da prehodno preslikavo $g: \tilde{r}^{-1} \circ  r : V_1 \to  V_2$. Vektor, razvit po bazi $\left\{ r_u, r_v \right\}$ bomo skušali
razviti bo bazi $\left\{ \tilde{r}_{\tilde{u}}, \tilde{r}_{\tilde{v}} \right\}$. Imamo 
\[ \begin{align*}
  r(u,v) &= \tilde{r}(\tilde{u}(u,v), \tilde{v}(u,v)), \\
  r_u   &= \tilde{r}_{\tilde{u}} \tilde{u}_u + \tilde{r}_{\tilde{v}} \tilde{v}_u, \\ 
  r_v &= \tilde{r}_{\tilde{u}} \tilde{u}_v + \tilde{r}_{\tilde{v}} \tilde{v}_v.
\end{align*}alii \] 
Zdaj hočemo vektor $a r_u + b r_v$ zapisati v obliki $\alpha \tilde{r}_u + \beta \tilde{r}_v$. Dobimo 
\[ a r_u + b r_v  = a(\tilde{r}_{\tilde{u}} \tilde{u}_u + \tilde{r}_{\tilde{v}} \tilde{v}_u) + b (\tilde{r}_{\tilde{u}} \tilde{u}_v + \tilde{r}_{\tilde{v}} \tilde{v}_v) 
= \underbrace{(a \tilde{u}_u + b \tilde{u}_v)}_{\alpha} \tilde{r}_{\tilde{u}} + \underbrace{(a \tilde{v}_u + b \tilde{v}_v)}_{\beta} \tilde{r}_{\tilde{v}}. \]
Torej za vsak par vektorjev $(a, b)^{T}$, $(\alpha, \beta)^{T}$ velja zveza 
\[ \begin{pmatrix}
  \alpha\\
  \beta\\
\end{pmatrix} = 
\underbrace{\begin{pmatrix}
  \tilde{u}_u & \tilde{u}_v\\
  \tilde{v}_u & \tilde{v}_v \\
\end{pmatrix}}_{\text{Jac}(g)}  
\begin{pmatrix}
  a\\
  b\\
\end{pmatrix}. \]

Torej za vse pare vektorjev velja 
\[ \begin{align*} 
    \left\langle \begin{pmatrix}
      a\\
      b\\
    \end{pmatrix}, \begin{pmatrix}
      c\\
      d\\
    \end{pmatrix} \right\rangle  &= \left\langle \begin{pmatrix}
      \alpha\\
      \beta  \\
    \end{pmatrix}, \begin{pmatrix}
      \gamma\\
      \delta  \\
    \end{pmatrix} \right\rangle  \\
    \begin{pmatrix}
      a & b\\
    \end{pmatrix} \begin{pmatrix}
    E & F \\
    F & G
    \end{pmatrix} \begin{pmatrix}
      c\\
      d  \\
    \end{pmatrix}  &= \begin{pmatrix}
      \alpha & \beta\\
    \end{pmatrix} \begin{pmatrix}
    \tilde{E} & \tilde{F} \\
    \tilde{F} & \tilde{G}
    \end{pmatrix} \begin{pmatrix}
      \gamma \\
      \delta  \\
    \end{pmatrix} \\ 
    &= \begin{pmatrix}
      a & b\\
    \end{pmatrix} 
    \operatorname{Jac}(g)^{T}
    \begin{pmatrix}
      \tilde{E} & \tilde{F} \\
      \tilde{F} & \tilde{G}
      \end{pmatrix}
    \operatorname{Jac}(g)
    \begin{pmatrix}
      c\\
      d  \\
    \end{pmatrix}.
\end{align*} \]
Od tod sledi naslednji izrek.

\begin{izrek}
\label{izr_transformacija_1_forme}
  Naj za parametrizaciji $r: V_1 \to  X$ in $V_2 \to X$ velja 
  \[ r(u,v) = \tilde{r}(\tilde{u}(u,v), \tilde{v}(u,v)) = \tilde{r}(g(u,v)).\]Potem za I-formi glede na ti parametrizaciji velja 
  \[ \begin{pmatrix}
  E & F \\
  F & G
  \end{pmatrix} = \operatorname{Jac}(g)^{T}
  \begin{pmatrix}
    \tilde{E} & \tilde{F} \\
    \tilde{F} & \tilde{G}
    \end{pmatrix}
  \operatorname{Jac}(g). \]
\end{izrek}

\begin{posledica}
\label{psl_dobra_definiranost_ploscine}
 Definicija ploščine \href{def_ploscina_plosvke} je dobra.
\end{posledica}
\noident
{\em Dokaz:\/}
 Dokazujemo 
 \[ A(X) = \int_{V_1} \sqrt{EG - F^2}   \, du \, dv = \int_{V_2}  \sqrt{\tilde{E}\tilde{G} -  \tilde{F}^2}  \, d \tilde{u} \, d \tilde{v}.\]
 Po izreku o transformaciji I-forme \ref{izr_transformacija_1_forme} velja 
 \[ \begin{align*}
     A(X) &= \int_{V_1} \sqrt{\det \begin{pmatrix}
     E & F \\
     F & G
     \end{pmatrix}}   \, du \, dv \\
      &= \int_{V_1}  \sqrt{\det \left( \text{Jac}(g)^{T}
      \begin{pmatrix}
        \tilde{E} & \tilde{F} \\
        \tilde{F} & \tilde{G}
        \end{pmatrix}
      \text{Jac}(g) \right) }   \, du \, dv \\ 
      &= \int_{V_1} \sqrt{\begin{pmatrix}
        \tilde{E} & \tilde{F} \\
        \tilde{F} & \tilde{G}
      \end{pmatrix}} \left| \det \left( \operatorname{Jac}(g) \right) \right|  \, du \, dv \\ 
     \text{(uvedba novih spremenljivk)} &= \int_{V_2} \sqrt{\begin{pmatrix}
      \tilde{E} & \tilde{F} \\
      \tilde{F} & \tilde{G}
      \end{pmatrix}}   \, d \tilde{u} \, d \tilde{v}.  
 \end{align*} \]
\qed

Recimo, da imamo podano $V \subseteq  \mathbb{R}^2$ in matrično funkcijo \begin{align*}
  M: V &\longrightarrow M_2(\mathbb{R})_\text{simetrične, pozitivno definitne}  \\
  (u,v) &\longmapsto \begin{pmatrix}
  E & F \\
  F & G
  \end{pmatrix}_{(u,v)}.
\end{align*} 
Zanimivo vprašanje se glasi: Ali lahko to funkcijo realiziramo v vloženi mnogoterosti? Odgovor je da (ampak lokalno, ker bi lahko prišlo do problemov s samopresečišči).

\subsection{Ukrivljenost}

\subsubsection{Druga fundamentalna forma}

Naj bo $X \subseteq \mathbb{R}^3$ ploskev s parametrizacijo $r: V \subseteq \mathbb{R}^2 \to  X$.
Intuitivno je ukrivljenost ploskve $X$ v točki $r(u,v) = m \in X$ `hitrost` oddaljevanja $X$ od tangentne ravnine $T_{m}X$.
Naj bo $n$ normala na ravnino $T_{m}X$ v točki $m$. Naj bo torej točka $r(u', v') = r(u + \Delta u, v + \Delta v)$ blizu točke $m = r(u,v)$ in izmerimo razdaljo od točke $r(u', v')$ do $T_mX$.

Dobimo 
\[ d = \langle n, r(u', v') - r(u,v) \rangle.\]
Ker sta spremembi $\Delta u$ in $\Delta v$ majhni, naredimo Taylorjev razvoj, ter izrazimo 
\[ r(u', v') - r(u,v) = r_u \Delta u + r_v \Delta v + \frac{1}{2} \left(r_{uu} (\Delta u)^2 + 2r_{uv} \Delta u \Delta v + r_{vv} (\Delta v)^2 \right) + \ldots \]

Ker vemo tudi, da je normala pravokotna na vektorja $r_u$ in $r_v$, lahko zapišemo 
\[ d = \langle n, r(u', v') - r(u,v) \rangle \approx \underbrace{\langle r_u, n \rangle}_0   \Delta u + \underbrace{\langle r_v, n \rangle}_0  \Delta v  
+ \frac{1}{2} \left( \langle r_{uu}, n \rangle (\Delta u)^2 +  2\langle r_{uv}, n \rangle \Delta u \Delta v + \langle r_{vv}, n \rangle (\Delta v)^2 \right).\]
oziroma 
\[ d \approx \frac{1}{2} \left( \langle r_{uu}, n \rangle (\Delta u)^2 +  2\langle r_{uv}, n \rangle \Delta u \Delta v + \langle r_{vv}, n \rangle (\Delta v)^2 \right). \]

Zdaj lahko smiselno definiramo drugo fundamentalno formo ploskve.

\begin{definicija}
\label{def_druga_fundamentalna_forma_ploskve}
 Druga fundamentaa forma ploslve $X$ v točki $m = r(u,v)$ je podana z matriko 
 \[ \begin{pmatrix}
 L & M \\
 M & N
 \end{pmatrix}_{(u,v)} = \begin{pmatrix}
 L(u,v) & M(u,v) \\
 M(u,v) & N(u,v)
 \end{pmatrix}, \]
 kjer so $L, M, N: V \to  \mathbb{R}$ funkcije s predpisi 
 \[ \begin{align*}
    L(u,v)  &= \langle r_{uu}(u,v), n(u,v) \rangle  \\
    M(u,v)  &= \langle r_{uv}(u,v), n(u,v) \rangle  \\
    N(u,v)  &= \langle r_{vv}(u,v), n(u,v) \rangle  
 \end{align*} \]

\end{definicija}

\begin{opomba}
    Za drugo fundamentalno formo je res nujen skalarni produkt v $\mathbb{R}^3$.
\end{opomba}

