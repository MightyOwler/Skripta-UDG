% 


\section{Uvod}% \label{sec:Uvod}

\begin{definicija}
\label{def_mnt}
Topološki prostor $M$
je $n$-dimenzionalna mnogoterost, če za vsak $m \in M$
obstaja okolica $m \in U \subseteq M$ in homeomorfizem
$\varphi : U \to V^{\text{odp}} \subseteq
\mathbb{R}^n$ (pri tem je $V \approx B^{n}$).
\end{definicija}


\begin{primer}
Naslednje množice so primeri mnogoterosti.
\begin{enumerate}
	\item $M = \mathbb{R}^n$ je
$n$-dimenzionalna mnogoterost,
	\item $S^1$ je
$1$-dimenzionalna mnogoterost,
  \item $S^{n} =
\left\{\left( x_1, x_2, \ldots, x_n, x_{n+1}
\right) \middle| \sum_{j=1}^{n+1} x_{i}^2 =
1\right\} \subseteq \mathbb{R}^{n+1}$ je
$n$-dimenzionalna mnogoterost,
	\item Projektivni
prostori $\mathbb{R}P^{n} = \bigslant{B^{n}}{\sim} $,
kjer je $\vec{x} \sim \vec{y}  \iff \vec{y}  =
-\vec{x}$ so $n$-dimenzionalne mnogoterosti.
	\item Grupa \[ \operatorname{SU}\left( 2 \right) = \left\{ g
= \begin{pmatrix}
\alpha & \beta \\ -
\overline{\beta}  & \overline{\alpha}  \\
\end{pmatrix}
\middle|\, \alpha, \beta \in  \mathbb{C}, \, \det
g = 1 \right\} 	\]je $3$-dimenzionalna mnogoterost. Topološko in
geometrijsko je namreč $\operatorname{SU}\left( 2 \right) = S^3.$ To
je primer Lijeve grupe.
	\item Grupa \[
\operatorname{SO}\left( 3 \right) =
\left\{ g =
\begin{pmatrix} a_{11} &
a_{12} & a_{13} \\ a_{21} & a_{22}
& a_{23} \\ a_{31} & a_{32}
& a_{33} \\
\end{pmatrix}
\middle|\, g^{T} = g^{-1}, \, \det  g
= 1 \right\}. \] Izkaže se, da je
$\operatorname{SO}\left( 3\right ) =
\bigslant{B^3}{\sim } = \mathbb{R}P^3.$ To
velja, ker vsaka preslikava iz
$\operatorname{SO}\left( 3\right )$
predstavlja rotacijo prostora, vsako
rotacijo pa lahko predstavimo z osjo in
velikostjo kota vrtenja. Pri tem kota $\pi$
in $-\pi$ predstavljata vrtenje za isti
kot. Če točki v krogli $B\left( 0, \pi
\right)^{3} \approx B^3$ priredimo os in njeno
razdaljo od izhodišča proglasimo za velikost kota vrtenja
ter enačimo iste rotacije, dobimo natanko projektivni prostor
$\mathbb{R}P^3$.
\end{enumerate}
\end{primer}

\subsection{Gladke mnogoterosti}% \label{sub:Gladke_mnogoterosti}

Na topoloških mnogoterostih bi radi znali odvajati različne objekte,
kot so na primer funkcije, krivulje, tenzorji itd. Zato moramo
mnogoterosti opremiti z dodatno strukturo. Za začetek se spomnimo
definicije odvedljivosti preslikav v evklidskih prostorih.

\begin{definicija}
Preslikava $F: W^{\text{odp}} \subseteq
\mathbb{R}^n \to \mathbb{R}^n$ je odvedljiva v točki $w \in W$, če
obstaja linearna preslikava $A : \mathbb{R}^n \to \mathbb{R}^n$ in
preslikava $\mathcal{O}: W \to \mathbb{R}^n$, da za vse ustrezne
argumente velja \[ F\left( w + h \right) = F\left( w \right) + Ah
+ \mathcal{O}\left( h \right)\] in $\lim_{h \to 0}
\frac{\lvert\lvert \mathcal{O}\left( h \right) \rvert\rvert
}{\lvert\lvert h \rvert\rvert } = 0.$ Odvod preslikave $F$ v točki $w$
je preslikava $A = D_wF = \left( DF \right)_w.$
\end{definicija}


\begin{definicija}
\label{def_odvedljivost}
Preslikava $F: W^{\text{odp}} \subseteq  \mathbb{R}^n \to \mathbb{R}^n$ je odvedljiva
na množici $W$, če je odvedljiva v vsaki točki $w \in
W$.
\end{definicija}

\begin{definicija} \label{def_difeomorfizem} Difeomorfizem je
bijektivna odvedljiva preslikava, ki ima odvedljiv inverz. 
\end{definicija}

\begin{definicija}
\label{def_atlas}
Naj bo $M$ $n$-dimenzionalna mnogoterost. Gladek atlas $\mathcal{U}$ na $M$ je družina parov
$\mathcal{U} = \left\{ \left( U_\alpha, \varphi_\alpha\right)
\mid \alpha \in  A \right\},$ če za vsak $\alpha \in A$ velja:
\begin{enumerate}
 \item $U_\alpha^{\text{odp}} \subseteq M$ \item
$\varphi_\alpha: U_\alpha \to V_\alpha \subseteq \mathbb{R}^n$
je homeomorfizem za nek $V_\alpha \subseteq \mathbb{R}^n$
 \item $\left\{ U_\alpha \middle|\, \alpha\in  A \right\}$ je
pokritje $M$ \item za vsaka $\alpha, \beta \in A$ je
preslikava $g_{\alpha \beta}  = \varphi_\beta \circ
\varphi_\alpha^{-1}: (\varphi_\alpha)_{*}(U_\alpha \cap
U_\beta) \to(\varphi_\beta)_{*}(U_\alpha \cap  U_\beta)$
difeomorfizem 
\end{enumerate}
Dodatek: Če so vse prehodne
preslikave $g_{\alpha \beta}$  $k$-difeomorfizmi z zveznim $k$-tim
odvodom, imamo $\mathcal{C}^{k}$-atlas. Če so vse preslikave gladke,
imamo $\mathcal{C}^{\infty}$-atlas, če so vse analitične, pa
$\mathcal{C}^{\omega}$-atlas.     
\end{definicija}

\begin{opomba}
Preslikava $g_{\alpha \beta}$  iz prejšnje definicije
je preslikava iz $U_\alpha \subseteq \mathbb{R}^n \to
\mathbb{R}^n$. Torej jo znamo odvajati in vemo, da je v izbranih
koordinatah na $\mathbb{R}^n$ matrika odvoda enaka Jacobijevi
matriki: \[ F\left( x_1, \ldots, x_{n} \right) =
\begin{pmatrix}
F_1(x_1, \dots, x_n)\\ \vdots\\ F_n(x_1, \dots, x_n)
\end{pmatrix}
\implies D_wF = \begin{pmatrix}
\frac{\partial F_1}{\partial x_1} & \dots & \frac{\partial
F_1}{\partial x_n} \\ \vdots & \ddots & \vdots \\
\frac{\partial F_n}{\partial x_1} & \dots & \frac{\partial
F_n}{\partial x_n} \end{pmatrix}_w	. \]
\end{opomba}

\begin{definicija}
\label{def_gladka_mnt}
Topološka mnogoterost $M$,
ki premore kakšen gladek atlas, je gladka mnogoterost.
\end{definicija}

Za motivacijo naslednje definicije se spomnimo dejstva, da vemo,
kakšne so gladke preslikave iz $\mathbb{R}^n \to  \mathbb{R}^n$. Nismo
pa še definirali gladkih preslikav iz mnogoterosti $M \to \mathbb{R}$. 

\begin{definicija}
\label{def_gladke_preslikave}
Naj bo $M$
$n$-dimenzionalna mnogoterost. Funkcija $f : M \to  \mathbb{R}$ je
gladka, če je gladka vsaka preslikava $f \circ \varphi_\alpha^{-1}
: V_\alpha \subseteq \mathbb{R}^n \to  \mathbb{R}$.
\end{definicija}


\begin{definicija}
\label{def_gladka_krivulja}
Naj bo $\left( M,\mathcal{U} \right)$  gladka mnogoterost. Krivulja $\gamma :
\left( a, b \right) \to  M$ je gladka krivulja v $M$, če za
$\forall  \alpha \in A$ velja, da je $\varphi_\alpha \circ \gamma
: \left( a, b \right) \to  V_\alpha \subseteq \mathbb{R}^n$ gladka
krivulja v $V_\alpha \subseteq  \mathbb{R}^n$.
\end{definicija}

\begin{definicija} \label{def_ekvivalnentnost_atlasov} Atlasa
$\mathcal{U} = \left\{ \left( U_\alpha, \varphi_\alpha \right)
\middle|\, \alpha \in A\right\}$ in $\mathcal{V} = \left\{
\left( W_\beta, \varphi_\beta \right) \middle|\,  \beta
\in  B\right\}$ na mnogoterosti $M$ sta ekvivalentna, če za
vsak par $\left( \alpha, \beta \right) \in  A \times  B$ iz
$U_\alpha \cap  W_\beta \neq \emptyset$ sledi, da je \[
\psi_\beta \circ \varphi_\alpha^{-1} : \left(
\varphi_\alpha \right)_{*}\left( U_\alpha \cap W_\beta \right)
\subseteq  \mathbb{R}^n \to \left( \psi_\beta \right)_{*}\left(
U_\alpha \cap W_\beta \right)\subseteq \mathbb{R}^n \] difeomorfizem.
\end{definicija}

\begin{opomba}
Ekvivalentnost atlasov je ekvivalenčna relacija,
ekvivalenčni razred atlasa $\mathcal{U}$ označimo z $\left[
\mathcal{U} \right].$ 
\end{opomba}

\begin{definicija} \label{def_gladka_struktura} Naj bo $M$ topološka
mnogoterost in $\mathcal{U}$ gladek atlas na $M$. Potem je $\left[
\mathcal{U} \right]$ gladka struktura na $M$.    
\end{definicija}

\begin{opomba} Dejstvo, da lahko obstajajo kakšne netrivialne
(eksotične strukture) na mnogoterostih, je zelo netrivialno. Iz
\href{https://en.wikipedia.org/wiki/Donaldson%27s_theorem}{Donaldsonovega}
in \href{https://mathworld.wolfram.com/FreedmanTheorem.html}{Freedmanovega}
izreka sledi, da ima $\mathbb{R}^{4}$ neštevno neskončno
eksotičnih gladkih struktur. Vsi ostali $\mathbb{R}^n$ imajo
zgolj svojo trivialno in nobene eksotične.
\end{opomba}