\section{Geodetske krivulje}

\begin{definicija}
\label{def_geodetska_krivulja}
Naj bo $X$ ploskev in $m_1, m_2 \in X$. Naj bo $\Gamma$ družina krivulj $\gamma_i: [a,b] \to X$, za katere velja $\gamma_i(a) = m_1$ in $\gamma_i(b) = m_2$.
Krivulja $\gamma \in  \Gamma$ je geodetska krivulja, če je najkrajša krivulja iz družine $\Gamma$, glede na prvo fundamentalno formo na $X$. 
\end{definicija}

\begin{definicija}
\label{def_dolzinski_funkcional}
Naj bo $X$ ploskev in $m_1, m_2 \in X$. Naj bo $\gamma: [a,b] \to  X$ krivulja, za katero velja $\gamma(a) = m_1$ in $\gamma(b) = m_2$. Potem presliakvo \begin{equation*}
\mathcal{L}(\gamma) = \int_{a}^{b} \sqrt{E \dot{u}^2(u(t), v(t)) + 2F(u(t), v(t)) \dot{u} \dot{v} + G(u(t), v(t)) \dot{v}^2}  \, dt 
\end{equation*}  
imenujemo dolžinski funkcional.
\end{definicija}

Naj bo sedaj \begin{equation*}
\mathcal{V} = \left\{ \gamma: [a,b] \to  X, \gamma \text{ gladka krivulja}, \gamma(a) = m_1, \gamma(b) = m_2 \right\}. 
\end{equation*}
\begin{opomba}
Takšni družini krivulj rečemo tudi variacija. Od tod pride ime "variacijski račun".
\end{opomba}
Potem imamo funkcional $\mathcal{L}: \mathcal{V} \to  \mathbb{R}$. Ekstreme tega funkcionala bomo iskali tako, kot iščemo ekstreme
funkcij ene ali dveh spremenljivk. Kandidati za ekstreme splošne funkcije $F: \Omega \subseteq \mathbb{R}^n \to  \mathbb{R}$ so stacionarne točke $F$, torej
točke $x_0 \in \mathbb{R}^n$, da je $D_{x_0}F = 0$. Vemo pa, da bo $D_{x_0}F = 0$ natanko tedaj, ko za vsak dovolj majhen $\varepsilon > 0$ in za vsako krivuljo
$\beta: [- \varepsilon, \varepsilon] \to  \Omega$, za katero je $\beta(0) = x_0$, za vsak $t \in [-\varepsilon, \varepsilon]$ velja \begin{equation*}
\frac{d}{dt}  \bigg|_{t = 0} F(\beta(t)) = (D_{x_0}F) \dot{\beta}(0) = 0. 
\end{equation*}  
To je direktna posledica dejstva, da je lahko vektor $\dot{\beta}(0) \in  \mathbb{R}^n$ poljuben.

Vrnimo se k funkcionalu $\mathcal{L}$. Ne bomo opazovali odvoda $\mathcal{L}$, temveč le smerne odvode $\mathcal{L}$ po vseh poteh, ki nas zanimajo. Krivulje $\gamma_i$ v evklidskem prostoru so točke v prostoru krivulj $\mathcal{V}$. Zanima nas,
kaj so krivulje v prostoru $\mathcal{V}$.

\begin{definicija}
\label{def_krivulja_v_prostoru_krivulj}
Krivulja v prostoru krivulj $\mathcal{V}$ je preslikava \begin{align*}
    \Gamma  (-\varepsilon, \varepsilon): &\longrightarrow \mathcal{V} \\
    s &\longmapsto \gamma_s,
\end{align*}
za katero za vsak $t \in  [a,b]$ velja $\Gamma(s,t) = \gamma_s(t)$. Hkrati morata biti za vsak $s \in (-\varepsilon, \varepsilon)$ izpoljnena pogoja \begin{align*}
    \Gamma(s)(a) &= m_1 = \gamma_s(a), \\
    \Gamma(s)(b) &= m_2 = \gamma_s(b).
\end{align*}
\end{definicija}

Posodobimo sedaj našo definicijo geodetke.

\begin{definicija}
\label{def_posodobljne_geodetkse_krivulje}
Krivulja $\gamma : [a,b] \to  X$ je geodetska krivulja, če je stacionarna točka dolžinskega funkcionala, torej če je \begin{equation*}
\frac{ \partial  }{ \partial s} \bigg|_{s = 0} \mathcal{L}(\Gamma(t,s)) = (D_\gamma \mathcal{L}) \left( \frac{ \partial  }{ \partial s} \bigg|_{s = 0} \Gamma(t,s) \right) . 
\end{equation*}    
\end{definicija}

\begin{opomba}
Zgornja enakost je nekoliko sumljiva, ker imamo na levi strani šibek odvod, na desni pa krepek.
\end{opomba}

Naj bo $\gamma(t) = r(u(t), v(t)): [a,b] \to X$ krivulja. Potem je \begin{equation*}
\mathcal{L}(\gamma(t)) = \int_{a}^{b} \sqrt{E\dot{u}^2 + 2F\dot{u}\dot{v} + G\dot{v}^2}  \, dt. 
\end{equation*}  
Naj bo $\Gamma : [a,b] \times  (-\varepsilon, \varepsilon) \to  X$ pot v $\mathcal{V}$, torej variacija $\gamma$. To pomeni, da velja za vse $(t,s) \in [a,b] \times  (-\varepsilon, \varepsilon)$ sistem enačb
\begin{align*}
    \Gamma(t,s) &= r(u(t,s), v(t,s)) \\
     u(t,s) &= u(t), \\
     v(t,s) &= v(t), \\
     u(a,s) &= v(s), \\
     u(a,s) &= u(a), \\
     v(a,s) &= v(a), \\
     u(b,s) &= u(b), \\
     v(b,s) &= v(b). 
\end{align*}
Potem imamo \begin{align*}
    \frac{ \partial  }{ \partial s} \bigg|_{s = 0} \mathcal{L}(\Gamma(t,s))  &= \frac{ \partial  }{ \partial s } \bigg|_{s = 0}  \left(\underbrace{\int_{a}^{b} \sqrt{E\dot{u}^2 + 2F\dot{u}\dot{v} + G\dot{v}^2}  \, dt }_{R(t,s)}\right)_{E = E(u,v), u = u(t,s), \dot{u} = \dot{u}(t,s), \ldots} \\
    &=  \frac{1}{2} \int_{a}^{b} \frac{1}{\sqrt{R(t,s)}} \frac{ \partial  }{ \partial s } R(t,s)  \, dt \\
    &=  \frac{1}{2} \int_{} \bigg[\frac{1}{\sqrt{R(t,s)}} (E_u \dot{u} u_s + 2F_u \dot{u} \dot{v} u_s + G_u u_s + E_v \dot{u}^2 v_s + 2F_v \dot{u} \dot{v} v_s + G_v \dot{v}^2 v_s) \\
    &+  2(E \dot{u} \dot{u}_s + F \dot{v} \dot{u}_s + F\dot{u} \dot{v}_s + G \dot{v} \dot{v}_s) \bigg] \, dt \\
    &= \frac{1}{2} \int_{a}^{b}  \frac{1}{\sqrt{R(t,s)}} \bigg[ (Eu \dot{u}^2 + F_u \dot{u} \dot{v} + G_u \dot{v}^2) u_s + (E_v \dot{u}^2 + 2F_v \dot{u} \dot{v} + G_v \dot{v}^2) \bigg] \, dt \\
    &+ \int_{a}^{b} \frac{1}{\sqrt{R(t,s)}} [(E\dot{u} + F\dot{v})\dot{u}_s + (F\dot{u} + G\dot{v}) \dot{v}_s] \, dt = (*)
\end{align*}  

Na tej točki velja omeniti, da če imamo podano variacijo $\Gamma(t,s) = r(u(t,s), v(t,s))$, potem tangento podaja predpis \begin{equation*}
\frac{ \partial  }{ \partial s } (u(t,s), v(t,s)) = (u_s(t,s), v_s(t,s)).
\end{equation*}  
To pomeni, da bomo morali v zgornjem izrazu dobiti $(u_s(t,s), v_s(t,s))$. Nadaljujemo s per partesom: \begin{align*}
    (*) &= \frac{1}{2} \int_{a}^{b}  \frac{1}{\sqrt{R(t,s)}} \bigg[ (Eu \dot{u}^2 + F_u \dot{u} \dot{v} + G_u \dot{v}^2) u_s + (E_v \dot{u}^2 + 2F_v \dot{u} \dot{v} + G_v \dot{v}^2) \bigg]v_s \, dt \\
    &- \int_{a}^{b} \frac{ \partial  }{ \partial t } \left( \frac{1}{\sqrt{R(t,s)}} (E\dot{u} + F\dot{v}) \right) u_s  + \frac{ \partial  }{ \partial t } \left( \frac{1}{\sqrt{R(t,s)}} (F\dot{u} + G\dot{v}) \right) v_s \, dt.
\end{align*}
Glede na začetne pogoje vemo, da mora biti $u_s(a,s) = u_s(b,s) = 0$, zato bo prvi celoten prvi del izraza enak 0.
TODO: Tukaj nisem prepričan, da je vse prav, kot smo pisali. Bom pregledal čez vikend.

\ldots

Pridemo do tega, da je krivulja $\gamma(t)$ stacionarna točka funkcionala $\mathcal{L}: \mathcal{V} \to  \mathbb{R}$, če je \begin{equation*}
\frac{ \partial  }{ \partial s } \mathcal{L}(\Gamma(t,s)) = 0,\,\,\, \forall (t,s).  
\end{equation*}  
To formuliramo z naslednjim izrekom, ki je ena izmed osnovnih variant osnovnega izreka o variacijskem računu.

\begin{izrek}
\label{izr_osnovni_izrek_o_variacijskem_racunu}
Naj bosta funkciji $P, Q: [a,b] \to  \mathbb{R}$ podani s predpisi \begin{align*}
    P(t) &= \frac{1}{2} R^{-\frac{1}{2}} (E_u \dot{u}^2 + 2F_u \dot{u} \dot{v} + G_u \dot{v}^2) - \frac{d}{dt} (R^{-\frac{1}{2}}(E\dot{u} + F\dot{v})), \\
    Q(t) &= \frac{1}{2} R^{-\frac{1}{2}} (E_v \dot{u}^2 + 2F_v \dot{u} \dot{v} + G_v \dot{v}^2) - \frac{d}{dt} (R^{-\frac{1}{2}}(F\dot{u} + G\dot{v})).
\end{align*}

Naj bo $\Gamma(t) = (u(t,s), v(t,s)) : [a,b] \times (-\varepsilon, \varepsilon) \to \mathbb{R}^2$ variacija. Potem velja enakost \begin{equation*}
 \int_{a}^{b}  \left( P(t) \frac{ \partial u }{ \partial s }(t, s)  + Q(t) \frac{ \partial v }{ \partial s }(t, s) \right)  \, dt = 0 
 \end{equation*}  
natanko tedaj, ko je $P(t) = 0$ in $Q(t) = 0$.
\end{izrek}

\noindent
{\em Dokaz:\/}
Dokazujemo s protislovjem. Recimo, da obstaja $t_0 \in (a,b)$, da je $P(t_0) \neq 0$. Brez škode za splošnost lahko privzamemo
$P(t_0) = c > 0$. Ker je $P$ zvezna, obstaja $\delta > 0$, da je $P(t) > \frac{c}{2}$ za vsak $t \in (t_0 - \delta, t_0 + \delta)$. 
Naj bo \begin{equation*}
(u(t,s), v(t,s)) = (u(t) + s \varphi(t), v(t)),
\end{equation*}  
kjer je $\varphi$ testna funckija, torej velja $\varphi(t) \ge 0$ na $(t_0 - \delta, t_0 + \delta)$ ter $\varphi(t) = 0$ na $[a,b] \setminus [t_0 - \delta, t_0 + \delta]$.
Potem veljata zvezi $\frac{ \partial u }{ \partial s }(t)= \varphi(t)$ ter $\frac{ \partial v }{ \partial s }(t) = 0$. Za to variacijo imamo
\begin{align*}
    \int_{a}^{b} (P(t)\frac{ \partial u }{ \partial s } (t) + Q(t)\frac{ \partial v }{ \partial s } (t)  ) \, dt &= \int_{a}^{b} P(t) \varphi(t) \, dt     \\
     &= \int_{t_0 - \delta}^{t_0 + \delta} P(t) \varphi(t)  \, dt > \frac{c}{2}  \int_{t_0 - \delta}^{t_0 + \delta} \varphi(t) \, dt > 0.
\end{align*} 
S tem smo pokazali dokaz za $P$, za $Q$ sledi simetrično.
\qed

To pomeni, da morata za geodetski krivulji veljati diferencialni enačbi drugega reda \begin{align*}
    0 &= \frac{1}{2} R^{-\frac{1}{2}} (E_u \dot{u}^2 + 2F_u \dot{u} \dot{v} + G_u \dot{v}^2) - \frac{d}{dt} (R^{-\frac{1}{2}}(E\dot{u} + F\dot{v})), \\
    0 &= \frac{1}{2} R^{-\frac{1}{2}} (E_v \dot{u}^2 + 2F_v \dot{u} \dot{v} + G_v \dot{v}^2) - \frac{d}{dt} (R^{-\frac{1}{2}}(F\dot{u} + G\dot{v})).
\end{align*}
Imenujemo jih Euler-Lagrangevi enačbi za geodetke.

\begin{opomba}
Če je $\gamma(t) = r(u(t), v(t))$ naravna parametrizacija, potem je $R = 1$ in se enačbi nekoliko poenostavita:
\begin{align*}
    0 &= \frac{1}{2} (E_u \dot{u}^2 + 2F_u \dot{u} \dot{v} + G_u \dot{v}^2) - \frac{d}{dt} (E\dot{u} + F\dot{v}), \\
    0 &= \frac{1}{2} (E_v \dot{u}^2 + 2F_v \dot{u} \dot{v} + G_v \dot{v}^2) - \frac{d}{dt} (F\dot{u} + G\dot{v}).
\end{align*}
\end{opomba}

\begin{izrek}
\label{izr_karakterizacija_geodetke_pri_naravni_parametrizaciji}
 Naj bo $\gamma(t)$ naravno parametrizirana krivulja. Potem je $\gamma(t)$ geodetka natanko tedaj, ko je \begin{equation*}
 \ddot{\gamma} \in (T_{\gamma(t)}X)^{\perp}.
 \end{equation*}  
Z drugimi besedami, odvod mora biti vedno pravokoten na tangentno ravnino.
\end{izrek}
\noindent
{\em Dokaz:\/}
Za domačo nalogo?
\qed