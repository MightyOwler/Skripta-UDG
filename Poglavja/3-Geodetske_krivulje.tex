\section{Geodetske krivulje}

\begin{definicija}
\label{def_geodetska_krivulja}
Naj bo $X$ ploskev in $m_1, m_2 \in X$. Naj bo $\Gamma$ družina krivulj $\gamma_i: [a,b] \to X$, za katere velja $\gamma_i(a) = m_1$ in $\gamma_i(b) = m_2$.
Krivulja $\gamma \in  \Gamma$ je geodetska krivulja, če je najkrajša krivulja iz družine $\Gamma$. 
\end{definicija}

\begin{definicija}
\label{def_dolzinski_funkcional}
Naj bo $X$ ploskev in $m_1, m_2 \in X$. Naj bo $\gamma: [a,b] \to  X$ krivulja, za katero velja $\gamma(a) = m_1$ in $\gamma(b) = m_2$. Potem preslikavo \begin{equation*}
\mathcal{L}(\gamma) = \int_{a}^{b} \sqrt{E \dot{u}^2(u(t), v(t)) + 2F(u(t), v(t)) \dot{u} \dot{v} + G(u(t), v(t)) \dot{v}^2}  \, dt 
\end{equation*}  
imenujemo dolžinski funkcional.
\end{definicija}

Naj bo sedaj \begin{equation*}
\mathcal{V} = \left\{ \gamma: [a,b] \to  X, \gamma \text{ gladka krivulja}, \gamma(a) = m_1, \gamma(b) = m_2 \right\}. 
\end{equation*}
\begin{opomba}
Takšni družini krivulj rečemo tudi variacija. Od tod pride ime "variacijski račun".
\end{opomba}
Ekstreme funkcionala $\mathcal{L}: \mathcal{V} \to  \mathbb{R}$ bomo iskali tako, kot iščemo ekstreme
funkcij ene ali dveh spremenljivk. Kandidati za ekstreme splošne funkcije $F: \Omega \subseteq \mathbb{R}^n \to  \mathbb{R}$ so stacionarne točke $F$, torej
točke $x_0 \in \mathbb{R}^n$, da je $D_{x_0}F = 0$. Vemo pa, da bo $D_{x_0}F = 0$ natanko tedaj, ko za vsak dovolj majhen $\varepsilon > 0$ in za vsako krivuljo
$\beta: [- \varepsilon, \varepsilon] \to  \Omega$, za katero je $\beta(0) = x_0$, velja \begin{equation*}
\frac{d}{dt}  \bigg|_{t = 0} F(\beta(t)) = (D_{x_0}F) \dot{\beta}(0) = 0. 
\end{equation*}  
% To je direktna posledica dejstva, da je lahko vektor $\dot{\beta}(0) \in  \mathbb{R}^n$ poljuben.  TODO poglej, kaj je s to izjavo

Vrnimo se k funkcionalu $\mathcal{L}$. Ne bomo opazovali odvoda $\mathcal{L}$, temveč le smerne odvode $\mathcal{L}$ po vseh poteh, ki nas zanimajo. Krivulje $\gamma_i$ v evklidskem prostoru so točke v prostoru krivulj $\mathcal{V}$. Zanima nas,
kaj so krivulje v prostoru $\mathcal{V}$.

\begin{definicija}
\label{def_krivulja_v_prostoru_krivulj}
Krivulja v prostoru krivulj $\mathcal{V}$ je preslikava \begin{align*}
    \Gamma : (-\varepsilon, \varepsilon) &\longrightarrow \mathcal{V} \\
    s &\longmapsto \gamma_s,
\end{align*}
za katero za vsak $t \in  [a,b]$ velja $\Gamma(t,s) := \Gamma(s)(t) = \gamma_s(t)$. Hkrati morata biti za vsak $s \in (-\varepsilon, \varepsilon)$ izpoljnena pogoja \begin{align*}
    \Gamma(a,s) &= m_1 = \gamma_s(a), \\
    \Gamma(b,s) &= m_2 = \gamma_s(b).
\end{align*}
\end{definicija}

Posodobimo sedaj našo definicijo geodetke.

\begin{definicija}
\label{def_posodobljne_geodetkse_krivulje}
Krivulja $\gamma : [a,b] \to  X$ je geodetska krivulja, če je stacionarna točka dolžinskega funkcionala, torej če je \begin{equation*}
\frac{ \partial  }{ \partial s} \bigg|_{s = 0} \mathcal{L}(\Gamma(t,s)) = (D_\gamma \mathcal{L}) \left( \frac{ \partial  }{ \partial s} \bigg|_{s = 0} \Gamma(t,s) \right) = 0 . 
\end{equation*}    
\end{definicija}

\begin{opomba}
Zgornja enakost je nekoliko sumljiva, ker imamo na levi strani šibek odvod, na desni pa krepek.
\end{opomba}

Sedaj izpeljimo diferencialne enačbe, ki jim morajo ustrezati geodetske krivulje. Naj bo $\gamma(t) = r(u(t), v(t)): [a,b] \to X$ krivulja. Potem je dolžinski funkcional podan s predpisom \begin{equation*}
\mathcal{L}(\gamma(t)) = \int_{a}^{b} \sqrt{E\dot{u}^2 + 2F\dot{u}\dot{v} + G\dot{v}^2}  \, dt. 
\end{equation*}  
Naj bo $\Gamma : [a,b] \times  (-\varepsilon, \varepsilon) \to  X$ pot v $\mathcal{V}$, torej variacija $\gamma$. To pomeni, da velja za vse $(t,s) \in [a,b] \times  (-\varepsilon, \varepsilon)$ sistem enačb
\begin{align*}
    \Gamma(t,s) &= r(u(t,s), v(t,s)) \\
     u(t,0) &= u(t), \\
     v(t,0) &= v(t), \\
     u(a,s) &= u(a), \\
     v(a,s) &= v(a), \\
     u(b,s) &= u(b), \\
     v(b,s) &= v(b). 
\end{align*}
Potem imamo \begin{align*}
    \frac{ \partial  }{ \partial s} \bigg|_{s = 0} \mathcal{L}(\Gamma(t,s))  &= \frac{ \partial  }{ \partial s } \bigg|_{s = 0}  \int_{a}^{b} \sqrt{\underbrace{E\dot{u}^2 + 2F\dot{u}\dot{v} + G\dot{v}^2}_{R(t,s)}}  \, dt  \\
    &=  \frac{1}{2} \int_{a}^{b} R^{-\frac{1}{2}} \frac{ \partial  }{ \partial s } \bigg|_{s = 0} R(t,s)  \, dt \\
    &=  \frac{1}{2} \int_{} \bigg[R^{-\frac{1}{2}} (E_u \dot{u}^2 u_s + 2F_u \dot{u} \dot{v} u_s + G_u \dot{v}^2 u_s + E_v \dot{u}^2 v_s + 2F_v \dot{u} \dot{v} v_s + G_v \dot{v}^2 v_s) \\
    &+  2(E \dot{u} \dot{u}_s + F \dot{v} \dot{u}_s + F\dot{u} \dot{v}_s + G \dot{v} \dot{v}_s) \bigg]_{s = 0} \, dt \\
    &= \frac{1}{2} \int_{a}^{b}  R^{-\frac{1}{2}} \bigg[ (E_u \dot{u}^2 + 2F_u \dot{u} \dot{v} + G_u \dot{v}^2) u_s + (E_v \dot{u}^2 + 2F_v \dot{u} \dot{v} + G_v \dot{v}^2)v_s \bigg]_{s = 0} \, dt \\
    &+ \int_{a}^{b} R^{-\frac{1}{2}} \bigg[(E\dot{u} + F\dot{v})\dot{u}_s + (F\dot{u} + G\dot{v}) \dot{v}_s \bigg]_{s = 0} \, dt = (*)
\end{align*}  

Na tej točki velja omeniti, da če imamo podano variacijo $\Gamma(t,s) = r(u(t,s), v(t,s))$, potem tangento podaja predpis \begin{equation*}
\frac{ \partial  }{ \partial s } (u(t,s), v(t,s)) = (u_s(t,s), v_s(t,s)).
\end{equation*}  
To pomeni, da bomo morali v zgornjem izrazu dobiti $(u_s(t,s), v_s(t,s))$. Nadaljujemo s per partesom drugega člena, pri čemer integriramo faktorja $\dot{u}_s$ in $\dot{v}_s$: \begin{align*}
    (*) &= \frac{1}{2} \int_{a}^{b}  R^{-\frac{1}{2}} \bigg[ (E_u \dot{u}^2 + 2F_u \dot{u} \dot{v} + G_u \dot{v}^2) u_s + (E_v \dot{u}^2 + 2F_v \dot{u} \dot{v} + G_v \dot{v}^2)v_s \bigg]_{s = 0} \, dt \\
    &- \int_{a}^{b} \frac{ d  }{ dt } \left( R^{-\frac{1}{2}} (E\dot{u} + F\dot{v}) \right) u_s  + \frac{ d }{ dt } \left( R^{-\frac{1}{2}} (F\dot{u} + G\dot{v}) \right) v_s \, dt.
\end{align*}
Glede na začetne pogoje vemo, da mora biti $u_s(a,s) = u_s(b,s) = 0$, zato bo prvi celoten prvi del izraza enak 0: \begin{equation*}
\bigg|_{s = 0} R^{-\frac{1}{2}} (E\dot{u} + F\dot{v}) u_s(t,s) \bigg|_a^{b} = 0
\end{equation*}  
in enako za člen $R^{-\frac{1}{2}} (E\dot{u} + F\dot{v}) v_s(t,s)$. Zdaj le še povzemimo vse skupaj.
Prišli smo do tega, da je krivulja $\gamma(t)$ stacionarna točka funkcionala $\mathcal{L}: \mathcal{V} \to  \mathbb{R}$, če je za vsak par $(t,s)$\begin{equation*}
\frac{ \partial  }{ \partial s } \mathcal{L}(\Gamma(t,s)) = \int_{a}^{b} P(t) \frac{ \partial u }{ \partial s } + Q(t) \frac{ \partial v }{ \partial s }  \, dt  = 0  
\end{equation*}
za funkciji \begin{align*}
    P(t) &= \frac{1}{2} R^{-\frac{1}{2}} (E_u \dot{u}^2 + 2F_u \dot{u} \dot{v} + G_u \dot{v}^2) - \frac{d}{dt} (R^{-\frac{1}{2}}(E\dot{u} + F\dot{v})), \\
    Q(t) &= \frac{1}{2} R^{-\frac{1}{2}} (E_v \dot{u}^2 + 2F_v \dot{u} \dot{v} + G_v \dot{v}^2) - \frac{d}{dt} (R^{-\frac{1}{2}}(F\dot{u} + G\dot{v})).
\end{align*}
To formuliramo z naslednjim izrekom, ki je ena izmed osnovnih variant osnovnega izreka o variacijskem računu.

\begin{izrek}
\label{izr_osnovni_izrek_o_variacijskem_racunu}
Naj bosta $P, Q: [a,b] \to  \mathbb{R}$ (zvezni?) funkciji in naj bo $\Gamma(t) = (u(t,s), v(t,s)) : [a,b] \times (-\varepsilon, \varepsilon) \to \mathbb{R}^2$ variacija, torej $u(t, 0) = u(t), v(t, 0) = v(t), u(a,s) = u(a), v(a,s) = v(a), u(b,s) = u(b)$ in $v(b,s) = v(b)$. Potem velja enakost \begin{equation*}
 \int_{a}^{b}  \left( P(t) \frac{ \partial u }{ \partial s }(t, s)  + Q(t) \frac{ \partial v }{ \partial s }(t, s) \right)  \, dt = 0 
 \end{equation*}  
natanko tedaj, ko je $P(t) = 0$ in $Q(t) = 0$.
\end{izrek}

\noindent
{\em Dokaz:\/}
Dokazujemo s protislovjem. Recimo, da obstaja $t_0 \in (a,b)$, da je $P(t_0) \neq 0$. Brez škode za splošnost lahko privzamemo
$P(t_0) = c > 0$. Ker je $P$ zvezna, obstaja $\delta > 0$, da je $P(t) > \frac{c}{2}$ za vsak $t \in (t_0 - \delta, t_0 + \delta)$. 
Naj bo \begin{equation*}
(u(t,s), v(t,s)) = (u(t) + s \varphi(t), v(t)),
\end{equation*}  
kjer je $\varphi$ testna funckija, torej velja $\varphi(t) > 0$ na $(t_0 - \delta, t_0 + \delta)$ ter $\varphi(t) = 0$ na $[a,b] \setminus [t_0 - \delta, t_0 + \delta]$.
Potem veljata zvezi $\frac{ \partial u }{ \partial s }(t)= \varphi(t)$ ter $\frac{ \partial v }{ \partial s }(t) = 0$.
Za to variacijo imamo
\begin{align*}
    \int_{a}^{b} \left(P(t)\frac{ \partial u }{ \partial s } (t) + Q(t)\frac{ \partial v }{ \partial s } (t)  \right) \, dt &= \int_{a}^{b} P(t) \varphi(t) \, dt     \\
     &= \int_{t_0 - \delta}^{t_0 + \delta} P(t) \varphi(t)  \, dt > \frac{c}{2}  \int_{t_0 - \delta}^{t_0 + \delta} \varphi(t) \, dt > 0.
\end{align*} 
S tem smo pokazali dokaz za $P$, za $Q$ sledi simetrično.
\qed

\begin{posledica}
\label{psl_Euler_Lagrangeevi_enacbi_za_geodetke}
Za geodetsko krivuljo $\gamma(t) = r(u(t), v(t))$ morata veljati diferencialni enačbi drugega reda \begin{align*}
    0 &= \frac{1}{2} R^{-\frac{1}{2}} (E_u \dot{u}^2 + 2F_u \dot{u} \dot{v} + G_u \dot{v}^2) - \frac{d}{dt} (R^{-\frac{1}{2}}(E\dot{u} + F\dot{v})), \\
    0 &= \frac{1}{2} R^{-\frac{1}{2}} (E_v \dot{u}^2 + 2F_v \dot{u} \dot{v} + G_v \dot{v}^2) - \frac{d}{dt} (R^{-\frac{1}{2}}(F\dot{u} + G\dot{v})).
\end{align*}
Imenujemo ju Euler-Lagrangevi enačbi za geodetke.
\end{posledica}


\begin{opomba}
Če je $\gamma(t) = r(u(t), v(t))$ naravna parametrizacija, potem je $R = 1$ in se enačbi nekoliko poenostavita:
\begin{align*}
    0 &= \frac{1}{2} (E_u \dot{u}^2 + 2F_u \dot{u} \dot{v} + G_u \dot{v}^2) - \frac{d}{dt} (E\dot{u} + F\dot{v}), \\
    0 &= \frac{1}{2} (E_v \dot{u}^2 + 2F_v \dot{u} \dot{v} + G_v \dot{v}^2) - \frac{d}{dt} (F\dot{u} + G\dot{v}).
\end{align*}
\end{opomba}

\begin{izrek}
\label{izr_karakterizacija_geodetke_pri_naravni_parametrizaciji}
 Naj bo $\gamma(t)$ naravno parametrizirana krivulja. Potem je $\gamma(t)$ geodetka natanko tedaj, ko je \begin{equation*}
 \ddot{\gamma} \in (T_{\gamma(t)}X)^{\perp}.
 \end{equation*}  
Z drugimi besedami, odvod mora biti vedno pravokoten na tangentno ravnino.
\end{izrek}
\noindent
{\em Dokaz:\/}
Naj bo $\gamma = r(u,v)$. Potem imamo \begin{align*}
    \dot{\gamma} &= \dot{u} r_u + \dot{v} r_v ,\\
    \ddot{\gamma} &= \ddot{u} r_u + \dot{u}^2 r_{uu} + \ddot{v} r_v + \dot{v}^2.
\end{align*}
Veljati morata enačbi $\langle \ddot{\gamma}, r_u \rangle = 0$, $\langle \ddot{\gamma}, r_v \rangle  = 0$. Prva od teh enačb
je ekvivalentna \begin{align*}
    \left\langle \frac{d}{dt} (\dot{u} r_u + \dot{v} r_v), r_u \right\rangle  &= \frac{d}{dt}  \langle \dot{u} r_u + \dot{v} r_v, r_u \rangle - \langle \dot{u} r_u + \dot{v} r_v, \dot{r}_u \rangle \\
     &= \frac{d}{dt}  \langle \dot{u} r_u + \dot{v} r_v, r_u \rangle - \langle \dot{u} r_u + \dot{v} r_v, \dot{u} r_{uu} + \dot{v} r_{uv} \rangle \\
     &= \frac{d}{dt} (E \dot{u} + F\dot{v}) - (\dot{u}^2 \langle r_u, r_uu \rangle + \dot{u} \dot{v} \langle r_v, r_{uv} \rangle + \dot{u} \dot{v} \langle r_u, r_{uv} \rangle + \dot{v}^2 \langle r_v, r_{vv}  \rangle  ) \\
     &= \frac{d}{dt}   (E \dot{u} + F\dot{v}) - \frac{1}{2} (E_u \dot{u}^2 + 2 F_u \dot{u} \dot{v} + G_u \dot{v}^2).
    \end{align*}
To je natanko prva Euler-Lagrangeva enačba za geodetke. Simetrično iz $\langle \ddot{\gamma}, r_v \rangle  = 0$ dobimo drugo enačbo.
\qed

\begin{posledica}
\label{psl_geodetksa_ukrivljenost_naravno_parametrziranih_geodetk_je_0}
Naj bo $\gamma: [a,b] \to X$ naravno parametrizirana geodetka. Potem je njena geodetska ukrivljenost $\kappa_g$ identično enaka $0$.
\end{posledica}

\begin{trditev}
\label{trd_lokalni_obstoj_geodetk}
Geodetske krivulje lokalno obstajajo.
\end{trditev}

\noindent
{\em Dokaz:\/}
Geodetske krivulje $\gamma(t) = r(u(t), v(t))$ so rešitve sistema Euler-Lagrangevih enačb. To je sistem dveh enačb drugega reda za par $(u(t), v(t))$. Po 
eksistenčnem izreku za navadne diferencialne enačbe rešitev mora nujno obstajati na dovolj majhni okolici začetne točke $(u_0, v_0)$.
\qed

\begin{opomba}
Ta eksistenčni izrek je po naravi zelo podoben eksistenčnemu izreku za Cauchyjevo nalogo, ki smo ga obdelali pri analizi.
\end{opomba}

\subsection{Geodetska parametrizacija}

\begin{definicija}
\label{def_geodetska_parametrizacija}

Naj bo $X$ ploskev in $r: V \to  X$ parametrizacija, za katero je $r(0, v)$ geodetska krivulja, hkrati pa je za vsak
$v_0$ tudi $\gamma_{v_0}(u) = r(u, v_0)$ geodetska krivulja in velja, da je $\dot{\gamma}_{v_0} \perp r_v(0, v_0)$.
Potem parametrizaciji $r$ rečemo geodetska parametrizacija.  

\end{definicija}

\begin{trditev}
\label{trd_ekvivalentni_pogoj_za_geodetskost_parametrizacije}
Naravna parametrizacija $r(u,v) = \gamma_v(u)$ je geodetska natanko tedaj, ko je njena prva fundamentalna forma oblike \begin{equation*}
I = \begin{pmatrix}
    1 & 0\\
    0 & G(u,v)\\
\end{pmatrix}.
\end{equation*}  
\end{trditev}

\noindent
{\em Dokaz:\/}
Naj bo $r$ naravno parametrizirana in naj bo $E = 1$. Potem velja \begin{equation*}
E = \langle r_u, r_u \rangle = \langle \dot{\gamma}_v, \dot{\gamma}_v \rangle = 1. 
\end{equation*}
Dokazujemo \begin{equation*}
F(0, v) = 0.
\end{equation*}
Ker sta vektorja $\dot{\gamma}_v (0), r_v (0, v)$ pravokotna, lahko zapišemo  
\begin{equation*}
F(0, v ) = \langle r_u(0, v), r_u(0, v) \rangle = \langle \dot{\gamma}_v (0), r_v (0, v) \rangle =  0. 
\end{equation*}  
Ker je geodetka $u \mapsto r(u, v_0) = \gamma_{v_0}(u)$ naravno parametrizirana geodetka, velja \begin{equation*}
r_{uu}(u, v_0) = \ddot{\gamma}_{v_0}(u) \in (T_{r(u, v_0)}X)^{\perp}.
\end{equation*}  
V splošnem velja $r_v \in  T_{r(u, v_0)}X$, torej sledi \begin{equation*}
\langle r_{uu}, r_v \rangle = 0. 
\end{equation*}  
Po drugi strani pa vemo, da je $E = 1$, zato lahko poračunamo \begin{equation*}
\langle r_u, r_{uv} \rangle = \langle r_u, r_u \rangle_v = 2 \langle r_u, r_{uv} \rangle = 2 E_v = 0.   
\end{equation*}  
Zdaj iz pogojev $F(0, v_0) = 0$ ter $F_u(u, v_0) = 0$ sledi $F(u,v) = 0$ za vse $(u,v)$. 
\qed

\begin{primer}
Najenostavnejši primer je ploskev $X = S^2$. S pomočjo prejšnjega izreka je to očitno, saj za parametrizacijo \begin{equation*}
r(u, v) = (\cos u \cos v, \cos u \sin v, \sin u)
\end{equation*}dobimo prvo fundamentalno formo \begin{equation*}
I = \begin{pmatrix}
    1 & 0\\
    0 & \cos^2 u\\
\end{pmatrix}.
\end{equation*}  
\end{primer}

\begin{primer}
Naj bo $X$ rotacijska ploskev, ki jo dobimo, če okoli $x$-osi zavrtimo krivuljo $(f(u), \cosh (\alpha u))$, kjer je $f$ podan z \begin{equation*}
f(u) = \int_{0}^{u} \sqrt{1 - \alpha^2 \sinh^2(\alpha t)}  \, dt. 
\end{equation*}  
Po znanem postopku poračunamo prvo fundamentalno formo in dobimo rezultat \begin{equation*}
    I = \begin{pmatrix}
        1 & 0\\
        0 & \cosh^2 (\alpha u)\\
    \end{pmatrix}.
\end{equation*}
Ni težko opaziti, da je ta fundamentalna forma zelo podobna formi sfere. V resnici nam zgornji predpis podaja vložitev hiperbolične ravnine v prostor $\mathbb{R}^3$. 
\end{primer}

\subsection{Gaussov izrek (Theorema egregium)}

\begin{izrek}
\label{izr_theorema_egregium}
Gaussova ukrivljenost ploskve je izrazljiva s koeficienti prve fundamentalne forme in njihovimi odvodi.
\end{izrek}

\begin{opomba}
Ta izrek z drugimi besedami pravi, da je Gaussova ukrivljenost intrinzična lastnost ploskve. 
\end{opomba}

\begin{definicija}
\label{def_intrinzicna_kolicina}
Intrinzična količina je količina, ki ni odvisna od vložitve ploskve v prostor. Intuitivno to pomeni, da lahko takšno količino izmeri vsak prebivalec ploskve (če je količina seveda izmerljiva).
\end{definicija}

\begin{opomba}
Druga fundamentalna forma ploskve ni intrinzična, ampak ekstrinzična količina ploskve.
\end{opomba}

\noindent
{\em Dokaz Theorema egregium:\/}
Poiskali bomo način, kako izraziti $\kappa$ z $E, F, G$ in njihovimi odvodi. Naj bo $r: V \subseteq \mathbb{R}^2 \to  X \subseteq  \mathbb{R}^3$ parametrizacija ploskve $X$. Izberimo par ortonormiranih
vektorskih polj $e_1, e_2$ na $X$: \begin{equation*}
e_i(u,v): V \subseteq \mathbb{R}^2  \to \bigsqcup_{(u,v) \in  V} T_{r(u,v)}X.
\end{equation*}  
Skalarne produkte (glede na prvo fundamentalno formo) teh dveh vektorjev lahko izrazimo s Kroneckerjevo delto: \begin{equation*}
\langle e_i, e_j \rangle_{I} = \delta_{ij}. 
\end{equation*}
Naj bo $n(u,v) \in (T_{r(u,v)}X)^{\perp}$ enotska normala v točki $r(u,v)$ podana s predpisom $n(u,v) = e_1(u,v) \times  e_2(u,v)$. V vsaki točki je $r(u,v) = m \in  X$ je trojica vektorjev $\left\{ e_1(u,v), e_2(u,v), n(u,v)\right\}$ ortonormirana baza prostora $\mathbb{R}^3$. 
Trdimo, da velja sistem enačb \begin{align*}
    (e_1)_u &= \alpha_1 e_2 + \lambda_1 n, \\
    (e_1)_v &= \alpha_2 e_2 + \lambda_2 n, \\
    (e_2)_u &= - \alpha_1 e_1 + \mu_1 n, \\
    (e_2)_v &= - \alpha_2 e_1 + \mu_2 n.
\end{align*}
Pri tem so $\alpha_i, \lambda_i, \mu_i: V \to \mathbb{R}$ funkcije, odvisne od $(u,v)$. Vektor $(e_1)_u$ (in na enak način ostale) razvijemo po bazi: \begin{equation*}
(e_1)_u = \langle (e_1)_u, e_1 \rangle e_1 + \langle  (e_1)_u, e_2 \rangle e_2 + \langle (e_1)_u, n \rangle n. 
\end{equation*}
Ker velja $\langle e_i, e_i \rangle = 1 $, sledi \begin{align*}
    \langle e_i, e_i \rangle_u  &=  2 \langle (e_i)_u, e_i \rangle  = 0, \\
    \langle e_i, e_i \rangle_v  &=  2 \langle (e_i)_v, e_i \rangle  = 0.
\end{align*}
Velja pa tudi $\langle e_1, e_2 \rangle = 0$, iz česar po parcialnem odvajanju po $u$ sledi
\begin{equation*}
    \langle (e_1)_u, e_2 \rangle + \langle e_1, (e_2)_u \rangle = 0, 
\end{equation*}  
oziroma \begin{equation*}
\alpha_1 := \langle (e_1)_u, e_2 \rangle = - \langle e_1, (e_2)_u \rangle.    
\end{equation*}  
Simetrično pokažemo s parcialnim odvajanjem po $v$ \begin{equation*}
\alpha_2  = \langle (e_1)_v, e_2 \rangle = - \langle e_1, (e_2)_v \rangle.
\end{equation*}  

Bistveni korak pri dokazovanju Gaussovega Theorema egregium bo naslednja lema.
\begin{lema}
\begin{equation*}
(\alpha_1)_v - (\alpha_2)_u = \langle  (e_1)_u, (e_2)_v \rangle - \langle (e_1)_v , (e_2)_u \rangle  = \lambda_1 \mu_2 - \lambda_2 \mu_1 =  \frac{LN - M^2}{\sqrt{EG - F^2} }.
\end{equation*}  
\end{lema}

\noindent
{\em Dokaz leme:\/}
Sredinsko enakost smo že dokazali v zgornjem razmisleku. Po drugi strani pa iz istih enačb dobimo sklep \begin{align*}
    (\alpha_1)_v - (\alpha_2)_u &= \frac{ \partial  }{ \partial u }  \langle e_1, (e_2)_v \rangle - \frac{ \partial  }{ \partial v}  \langle e_1, (e_2)_u \rangle \\
     &= \langle  (e_1)_u, (e_2)_v \rangle + \langle  e_1, (e_2)_{uv} \rangle  - \langle  (e_1)_v, (e_2)_u \rangle - \langle  e_1, (e_2)_{uv} \rangle \\
     &= \langle  (e_1)_u, (e_2)_v \rangle - \langle  (e_1)_v, (e_2)_u \rangle.
\end{align*}  
V splošnem velja $\langle n, r_u \rangle = 0$. Ker je $n = e_1 \times  e_2$, imamo po Lagrangeevi enakosti za skalarni produkt vektorskih produktov \begin{equation*}
\langle n_u \times n_v, n \rangle = \langle n_u \times  n_v, e_1 \times e_2 \rangle = \langle n_u, e_1 \rangle \langle n_v, e_2 \rangle - \langle n_u, e_2 \rangle \langle n_v, e_1 \rangle = \lambda_1 \mu_2 - \lambda_2 \mu_1. 
\end{equation*}  
Člene iz zadnje enakosti dobimo s pomočjo sklepa \begin{align*}
     \langle n, e_1 \rangle &= 0, \,\,\, \left(\bigg/ \frac{ \partial  }{ \partial u } \right) \\
     \langle n_u, e_1 \rangle + \langle  n, (e_1)_u \rangle  &= 0, \\
     \langle n_u, e_1 \rangle &= - \langle n, (e_1)_u \rangle = - \lambda_1, \\
     (\text{podobno dobimo recimo}) \,\,\, \langle n_v, e_2 \rangle &= - \langle n, (e_2)_v \rangle = -\mu_2,   
\end{align*}
in upoštevanja definicij koeficientov pri razvoju vektorjev $(e_1)_u, (e_1)_v, (e_2)_u$ in $(e_2)_v$ po bazi.
\qed

Privzemimo sedaj, da smo dokazali \begin{equation*}
(\alpha_1)_v - (\alpha_2)_u = \frac{LN - M^2}{\sqrt{EG - F^2}}.  
\end{equation*}  
 Po drugi strani pa smo izpeljali zvezo za Gaussovo ukrivljenost \begin{equation*}
    \kappa = \frac{LN - M^2}{EG - F^2}.
 \end{equation*}  
Če nam uspe pokazati, da je izraz $(\alpha_1)_v - (\alpha_2)_u$ izrazljiv s koeficienti prve fundamentalne forme in njihovimi odvodi, bo Theorema egregium dokazan, saj lahko zapišemo \begin{equation*}
\kappa = \frac{(\alpha_1)_v - (\alpha_2)_u}{\sqrt{EG - F^2}}.
\end{equation*}  

Oglejmo si Gramm-Schmidtov postopek, ki nadomesti bazo $\left\{ r_u, r_v\right\}$ z bazo $\left\{ e_1, e_2 \right\} $. \begin{align*}
    e_1 &= a r_u   \\
    e_2 &= b r_u + c r_v.
\end{align*}
Pri tem takoj vidimo, da je $a = \frac{1}{\sqrt{E} }$. Iz enačbe $\langle e_1, e_2 \rangle$ dobimo \begin{align*}
    \langle a r_u , b r_u + c r_v \rangle  &= ab E + ac F  \\
     &= a(bE + cF) \\
     &= a \langle (b, c)^{T}, (E, F)^{T} \rangle = 0. 
\end{align*}
To pomeni, da je vektor $(E, F)^{T}$ vzporeden $(c, -b)^{T}$, oziroma \begin{align*}
    b &= -\lambda F, \\
    c &= \lambda E.
\end{align*}
Upoštevajmo še, da mora biti $e_2$ enotski vektor. \begin{align*}
    \langle e_2, e_2 \rangle  &= \langle - \lambda F r_u + \lambda E r_v, -\lambda F r_u + \lambda E r_v \rangle   \\
     &= \lambda^2 E F^2 - \lambda^2 E F^2 - \lambda^2 EF^2 + \lambda^2 E^2 G \\
     &= -\lambda^2 E F^2 + \lambda^2 E^2 G \\
     &= \lambda^2 (E^2 G - EF^2 ) = \lambda^2  E (EG - F^2) = 1.
\end{align*}
Torej je \begin{equation*}
\lambda = \frac{1}{\sqrt{E (EG - F^2)}}.
\end{equation*}  
To povzemimo: \begin{align*}
    a &= \frac{1}{\sqrt{E}}, \\
    b  &= -\frac{F}{\sqrt{E (EG - F^2)}}, \\
    c &= \frac{\sqrt{E}}{\sqrt{EG - F^2}}.
\end{align*}
Od tod dobimo \begin{align*}
    \alpha_1 &= \langle (e_1)_u, e_2 \rangle  \\
     &= \langle (a r_u)_u, e_2 \rangle  \\
     &= a \langle r_{uu}, e_2 \rangle + a_u \underbrace{\langle r_u, e_2 \rangle}_0  \\
    &= ab \langle r_{uu}, r_u \rangle + ac \langle r_{uu}, r_v \rangle \\
    &= \frac{1}{2} ab E_u + ac \left(F_u - \frac{1}{2} E_v \right). 
\end{align*}
Podobno dobimo \begin{align*}
    \alpha_2 &= \langle (e_1)_v, e_2 \rangle  \\
     &= \langle (a r_u)_v, e_2 \rangle  \\
     &= a \langle r_{uv}, e_2 \rangle + a_v \underbrace{\langle r_u, e_2 \rangle}_0  \\
    &= ab \langle r_{uv}, r_u \rangle + ac \langle r_{uv}, r_v \rangle \\
    &= \frac{1}{2} ab E_v + \frac{1}{2}ac G_u. 
\end{align*}
Ker smo koeficiente $a, b$ in $c$ že uspeli izraziti z $E, F, G$ in njihovimi odvodi, nam je isto uspelo tudi za izraz $(\alpha_1)_v - (\alpha_2)_u$.
S tem je Gaussov Theorema egregium dokaz.
\qed

\begin{opomba}
Splošen izraz za $\kappa$ je kar zakompliciran. Čim pa predpostavimo, da imamo parametrizacijo $F =0$, velja \begin{align*}
    a = \frac{1}{\sqrt{E} }, \,\,\,  b = 0, \,\,\,   c = \frac{1}{\sqrt{G}},
\end{align*}oziroma \begin{align*}
    \alpha_1 = -\frac{1}{2} \frac{E_v}{\sqrt{EG} }, \,\,\, \alpha_2 = \frac{1}{2} \frac{G_u}{\sqrt{EG} }.
\end{align*}
Od tod dobimo \begin{equation*}
\kappa  = \frac{(\alpha_1)_v - (\alpha_2)_u}{\sqrt{EG}}.
\end{equation*}
To povzamemo v izreku.
\end{opomba}

\begin{izrek}
\label{izr_gaussova_ukrivljenost_v_primeru_F_enako_0}
Če je $F = 0$, je Gaussova ukrivljenost podana s formulo \begin{equation*}
\kappa = - \frac{1}{2\sqrt{EG}} \left(\left(  \frac{G_u}{\sqrt{EG} } \right)_u + \left( \frac{E_v}{\sqrt{EG} } \right)_v \right).
\end{equation*}  
\end{izrek}

Formula za Gaussovo ukrivljenost se še bolj poenostavi, če je parametrizacija ploskve geodetska (torej da je $E = 1$, $F = 0$).

\begin{izrek}
\label{izr_gaussova_ukrivljenost_pri_geodetski_parametrizaciji}
 Če je ploskev $X$ parametrizirana z geodetsko parametrizacijo, se njena Gaussova ukrivljenost izraža s formulo \begin{equation*}
 \kappa = - \frac{1}{2\sqrt{G}} \left( \frac{G_u}{\sqrt{G}} \right)_u .
 \end{equation*}  
\end{izrek}

\subsection{Ploskve s konstantno ukrivljenostjo}
\begin{izrek}
\label{izr_klasifikacija_ploskev_z_ukrivljenostjo_0}\begin{enumerate}
    \item Ploskev $X$, katere ukrivljenost je identično enaka $0$, je izometrična kosu ravnine.
    \item Ploskev $X$, katere ukrivljenost je identično enaka $1 / a^2$ ali $- 1 / a^2$ je izometrična kosu sfere z radijem $a$ oziroma
    kosu traktroida (psevdosfere) s polmerom $a$.
 \end{enumerate}
\end{izrek}

\noindent
{\em Dokaz:\/}
\begin{enumerate}
    TODO: v tem dokazu je uporabljena napačna formula za Gaussovo ukrivljenost na ortogonalno parametrizirani ploskvi \ldots treba bo popraviti.
    \item Vpeljimo novo oznako za koeficient $g := \sqrt{G}$. Ta funkcija je dobro definirana, saj mora zaradi regularnosti parametrizacije biti $G > 0$. Dokazali smo že, da lahko vsako ploskev lokalno parametriziramo z geodetsko parametrizacijo. V tej parametrizaciji velja \begin{equation*}
        \kappa = - \frac{1}{\sqrt{G}} \left( \frac{ \partial^{2} \sqrt{G}  }{ \partial u^{2} }  \right) = \frac{1}{g} \frac{ \partial^{2} g }{ \partial u^{2} } .
        \end{equation*}  
        Za ploščato ploskev torej velja \begin{equation*}
        -\frac{1}{g} \frac{ \partial^{2} g }{ \partial u^{2} } = 0 \implies \frac{ \partial^{2} g }{ \partial u^{2} } = 0. 
        \end{equation*}
        To pomeni, da je funkcija $g = g(u,v)$ oblike \begin{equation*}
        g(u,v) = A(v) u + B(v).
        \end{equation*}  
        Poiskati moramo ustrezni funkciji $A(v)$ in $B(v)$, ki ju bomo dobili iz geometrijskih lastnosti geodetske parametrizacije. Spomnimo se, da za
        geodetsko krivuljo $v \mapsto r(0,v)$ velja \begin{equation*}
        G(0, v) = \langle r_v(0,v), r_v(0,v) \rangle = 1. 
        \end{equation*}  
        Enakost $\langle r_v(0,v), r_v(0,v) \rangle = 1$ dobimo iz naravnosti parametrizacije $r$. Poiščimo še $G_u(0, v)$. Ker je krivulja
        $v \mapsto  G(0, v)$ naravno parametrizirana, je rešitev prve Euler-Lagrangeeve enačbe za geodetke, ki jo v našem posebnem primeru lahko zapišemo v obliki \begin{equation*}
        \ddot{u} = \frac{1}{2} G_u \dot{v}^2. 
        \end{equation*}  
        Naša geodetka je podana z $\gamma(t) = r(0, v(t))$ in velja $u = 0, v = t$, od koder sledi \begin{align*}
            \ddot{u} &= 0, \\
            \dot{v} &= 1.
        \end{align*}
        Torej dobimo \begin{equation*}
        0 = \frac{1}{2} G_u \implies G_u(0,v) = 0.
        \end{equation*}  
        Začetna pogoja $G(0, v) = 1$ in $G_u(0,v) = 0$ implicirata \begin{align*}
            G(0,v)  = 1 &\implies g(0,v) = 1 \\
            G_u(0,v)  = 0 &\implies G_u = (g^2)_u = 2gg_u = 0 \implies g_u(0, v) = 0.
        \end{align*} 
        Od tod dobimo \begin{align*}
            g(0,v) &= B(v) = 1, \\
            g_u(0,v) &= A(v) = 0.
        \end{align*}
        To pomeni, da je \begin{equation*}
        g(u,v) = 1 \implies G(u, v) = 1,
        \end{equation*}  
        torej bo prva fundamentalna forma ploskve $X$ enaka \begin{equation*}
        I = \begin{pmatrix}
        1 & 0 \\
        0 & 1
        \end{pmatrix},
        \end{equation*}  
        in bo po izreku za prve fundamentalne forme izometrična kosu ravnine.
\item Najprej obravnavajmo primer $\kappa = \frac{1}{a^2}.$ Iz enačbe \begin{equation*}
\kappa = - \frac{g_{uu}}{g} = \frac{1}{a^2}
\end{equation*}  
dobimo \begin{equation*}
g_{uu} = -\frac{1}{a^2} g.
\end{equation*}
Splošna rešitev te enačbe je oblike \begin{equation*}
g(u,v) = A(v) \cos (\frac{1}{a} u) + B(v) \sin(\frac{1}{a} u)
\end{equation*}in zadošča začetnima pogojema \begin{equation*}
g(0, v) = 1, \,\,\, g_u(0, v) = 0.
\end{equation*}  
Tako dobimo $g(0, v) = A(v) = 1$, $g_u(0,v) = \frac{1}{a} B(v) = 0 \implies B(v)  = 0.\\$
Torej imamo \begin{align*}
    g(u,v) &= \cos (\frac{1}{a} u),  \\
    G(u,v) &= \cos^2 \left(\frac{1}{a} u\right),
\end{align*}
prva fundamentalna forma se glasi \begin{equation*}
I = \begin{pmatrix}
1 & 0 \\
0 & \cos^2 (\frac{1}{a} u)
\end{pmatrix},
\end{equation*}  
kar je natanko fundamentalna forma sfere. Torej je $X$ izometrična kosu sfere. 

V primeru $\kappa = - \frac{1}{a^2}$ dobimo enačbo 
\begin{equation*}
    \frac{g_{uu}}{g} &= \frac{1}{a^2} \implies  g_{uu} &= \frac{1}{a^2} g.
\end{equation*}
Nato na isti način kot za $\kappa = \frac{1}{a^2}$ dobimo splošno rešitev \begin{equation*}
g(u,v) = A(v) \cosh ( \frac{1}{a}  u) + B(v) \sinh (\frac{1}{a} u),
\end{equation*}  
od tod pa \begin{equation*}
G(u,v) = \begin{pmatrix}
1 & 0 \\
0 & \cosh^2 (\frac{1}{a} u)
\end{pmatrix},
\end{equation*}  
kar je natanko fundamentalna forma traktroida.
\end{enumerate}

\qed
