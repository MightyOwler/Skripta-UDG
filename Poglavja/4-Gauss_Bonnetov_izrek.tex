\section{Gauss-Bonnetov izrek}

Gauss-Bonnetov izrek ima dve verziji: lokalno in globalno. Prvi izrek obravnava (odsekoma) gladke sklenjene krivulje, drugi pa poveže Gaussovo ukrivljenost z Eulerjevo karakteristiko ploskve.

\begin{izrek}
\label{izr_lokalna_verzija_GB_izreka}
 Naj bo $X$ gladka ploskev in naj bo $\gamma$ enostavno sklenjena krivulja na $X$, ki omejuje območje $R \subseteq  X$, difeomorfno disku, z robom $\partial R = \gamma$. 
 Tedaj velja \begin{equation*}
 \int_{\gamma} \varkappa_g \, ds = 2 \pi - \int_{R} \kappa \, dA.
 \end{equation*}  
Pri tem je $\varkappa_g(\gamma(s))$ geodetska ukrivljenost krivlje $\gamma$, $ds$ pa ločni element na $\gamma$ glede na naravno parametrizacijo. 
\end{izrek}

Pred dokazom si poglejmo dva primera:
\begin{primer}
Recimo, da je $X$ ravnina. Zanjo velja $\kappa = 0$, torej se Gauss-Bonnetova formula glasi \begin{equation*}
    \int_{\gamma} \varkappa_g \, ds = 2 \pi = \int_{\gamma} \varkappa \, ds.
\end{equation*}  
To je smiselno zaradi naslednjega razmisleka. Naj bo $\psi(s)$ kot med $\dot{\gamma}(s)$ in abscisno osjo. Ker krivulja brez samopresečišč obkroži disk natanko enkrat, velja \begin{equation*}
\int_{\gamma} \frac{ d \psi }{ ds }   \, ds = \int_{\gamma}  \dot{\psi}(s) \, ds = \psi(l(s)) - \psi(0) = 2 \pi. 
\end{equation*}
Pokazati moramo, da je $\dot{\psi} = \lvert\lvert \ddot{\gamma} \rvert\rvert = \varkappa$. Najprej imamo po definiciji kota $\psi$ zvezo \begin{equation*}
    \dot{\gamma}(s) = (\cos(\psi(s)), \sin(\psi(s))).
\end{equation*} 
Sledi \begin{equation*}
\ddot{\gamma}(s) = (- \dot{\psi} \sin(\psi(s)), \dot{\psi} \cos(\psi(s))) \implies \lvert\lvert \ddot{\gamma} \rvert\rvert = \dot{\psi}.
\end{equation*}  
Torej res dobimo \begin{equation*}
    \int_{\gamma} \varkappa \, ds = 2 \pi.
\end{equation*}  

\begin{opomba}
Da lahko krivulja brez samopresečišč obroži točko največ enkrat, je dokazal Hopf v svojem \href{https://de.wikipedia.org/wiki/Umlaufsatz}{umlaufsatzu}.  
\end{opomba}
\end{primer}

\begin{primer}
    Naj bo $X$ sfera s polmerom $r$ in $R \subseteq  X$ zgornja polsfera. Potem je $\gamma = \partial R$ ekvator in zato geodetska krivulja. Velja torej $\varkappa_g = 0$ ter $\kappa = \frac{1}{r^2}$. Dobimo \begin{equation*}
    0 = \int_{\gamma} \varkappa_g  \, ds = 2 \pi - \int_{R} \kappa  \, dA \implies \int_{R}  \, dA =  2 \pi r^2.
    \end{equation*}  
    Torej ima zgornja polsfera površino $2 \pi r^2$, kar se seveda ujema z našim znanjem o površini sfere.
\end{primer}

\noindent
{\em Dokaz lokalnega Gauss-Bonnetovega izreka:\/}
Ključ do izreka bo Greenova formula. Naj bo $\beta$ sklenjena krivulja, ki omejuje območje $S$. Tedaj za poljubni gladki funkciji $P, Q$ na $S$ velja \begin{equation*}
\int_{\beta} P \, du  + Q \, dv = \int_{S} Q_u - P_v \, du dv.
\end{equation*}  
Naj bo $r: V \to  X$ parametrizacija naše ploskve in naj bo krivulja $\gamma(s) = r(\beta(s))$ naravno parametrizirana. Oglejmo si količino \begin{equation*}
I = \int_{\beta} \langle e_1, \dot{e}_2 \rangle   \, ds.
\end{equation*}  
Pri tem sta  $e_1$ in $e_2$ gladki, ortonormirani vektorski polji na $R$. Zapišimo najprej $e_2(s) = e_2(u(s), v(s))$. Potem imamo \begin{equation*}
\dot{e}_2 = (e_2)_u \dot{u} + (e_2)_v \dot{v}.
\end{equation*}  
Tako dobimo \begin{equation*}
\langle e_1, \dot{e}_2 \rangle \, ds  = \underbrace{\langle e_1, (e_2)_u \rangle}_P  \, du +  \underbrace{\langle e_1, (e_2)_v \rangle}_Q  \, dv.
\end{equation*}  
Zdaj uporabimo Greenovo formulo, da dobimo \begin{align*}
    I &= \int_{S} \langle e_1, (e_2)_v \rangle_u - \langle e_1, (e_2)_u \rangle_v   \, dS  \\
     &= \int_{S} \langle (e_1)_u, (e_2)_v \rangle + \langle e_1, (e_2)_{uv} \rangle - \langle (e_1)_v, (e_2)_u \rangle   - \langle e_1, (e_2)_{uv} \rangle \, dS \\
     &=  \int_{S} \langle (e_1)_u, (e_2)_v \rangle - \langle (e_1)_v, (e_2)_u \rangle  \, dS \\
     &= \int_{S} \frac{LN - M^2}{\sqrt{EG - F^2} }  \, dS. \,\,\,(\text{lema iz dokaza Theorema egregium})
\end{align*}
Imamo \begin{equation*}
I = \int_{S} \frac{LN - M^2}{\sqrt{EG - F^2} }  \, du dv =  \int_{S} \frac{LN - M^2}{EG - F^2} \sqrt{EG - F^2}  \, du dv = \int_{R} \frac{LN - M^2}{EG - F^2} \, dA = \int_{R} \kappa \, dA.
\end{equation*}  

Izrazimo $I$ še na drug način. Naj bo $\theta(s)$ kot med $\dot{\gamma}(s)$ in $e_1(s)$. Ker je $\left\{ e_1(\gamma(s)), e_2(\gamma(s))\right\}$ ortonormirana baza v tangentnem prostoru
$T_{\gamma(s)}X$, lahko pišemo \begin{equation*}
\dot{\gamma}(s) = \cos(\theta(s))e_1(\gamma(s)) + \sin(\theta(s))e_2(\gamma(s)).
\end{equation*}  
Označimo $\eta = n \times \dot{\gamma} = - \sin(\theta(s)) e_1(\gamma(s)) + \cos(\theta(s)) e_2(\gamma(s))$, ki je enotski vektor v  $T_{\gamma(s)}X$.
Imamo še \begin{equation*}
\ddot{\gamma} = -\dot{\theta} \sin(\theta) e_1 + \dot{\theta} \cos(\theta) e_2 +  \cos(\theta)\dot{e}_1 + \sin(\theta)\dot{e}_2 = \dot{\theta} \eta + \cos(\theta) \dot{e}_1 + \sin(\theta) \dot{e}_2.  
\end{equation*}  
Zdaj lahko izračunamo geodetsko ukrivljenost krivulje $\gamma$: \begin{align*}
    \varkappa_g &= \langle \ddot{\gamma}, \eta \rangle  \\
     &= \langle \dot{\theta} \eta + \cos(\theta) \dot{e}_1 + \sin(\theta) \dot{e}_2, \eta  \rangle \\
     &= \dot{\theta} + \langle \cos(\theta) \dot{e}_1 + \sin(\theta)\dot{e}_2, - \sin(\theta) e_1 + \cos(\theta) e_2 \rangle \\
     &= \dot{\theta} - (\cos^2(\theta) + \sin^2(\theta) ) \langle e_1, \dot{e}_2  \rangle \\
     &= \dot{\theta} - \langle e_1, \dot{e}_2  \rangle.  
\end{align*}
Pri prehodu na predzadnjo enakost smo uporabili dejstvi, da $\langle e_i, \dot{e}_i \rangle = 0$ ter $\langle e_1, \dot{e}_2 \rangle = - \langle e_2, \dot{e}_1 \rangle$.
Ugotovili smo, da je \begin{equation*}
\int_{\gamma} \varkappa_g  \, ds = \int_{\beta} \dot{\theta}  \, ds - \int_{\beta} \langle e_1, \dot{e}_2 \rangle   \, ds.  
\end{equation*}  
Od tod dobimo \begin{equation*}
    \int_{\gamma} \varkappa_g  \, ds  = 2 \pi  - \int_{R}  \kappa \, dA.  
\end{equation*}  
\qed
  
Zdaj dokažimo še nekoliko splošnejšo verzijo izreka, ki velja za odsekoma gladke kriuvlje.

\begin{izrek}
\label{izr_odsekoma_gladke_krivulje_GB}
Naj bo $\gamma$ naravno parametrizirana odsekoma gladka krivulja, ki omejuje $n$-kotnik $R$ z notranjimi koti $\alpha_1, \alpha_2, \ldots, \alpha_n$. Potem velja \begin{equation*}
\sum_{i=1}^{n} \alpha_i = (n-2) \pi + \int_{\gamma} \varkappa_g  \, ds + \int_{R} \kappa \, dA.  
\end{equation*}  
\end{izrek}

\noindent
{\em Dokaz:\/}
Naj bodo $\delta_i$ zunanji koti mnogokotnika, ki pripadajo kotom $\alpha_i$, torej $\delta_i = \pi - \alpha_i$. Opazimo, da bo vrednost izraza \begin{equation*}
    \int_{\gamma} \dot{\theta}  \, ds = 2 \pi - \sum_{i = 1}^{n} \delta_i.
\end{equation*}  
Torej bo \begin{align*}
    I = \int_{R} \kappa  \, dA &= 2 \pi  - \sum_{i = 1}^{n} \delta_i - \int_{\gamma} \varkappa_g  \, ds   \\
     &= 2 \pi- \sum_{i =1}^{n} (\pi - \alpha_i) - \int_{\gamma} \varkappa_g \, ds \\
    &= (2 - n)\pi + \sum_{i = 1}^{n} \alpha_i - \int_{\gamma} \varkappa_g \, ds \\
    &\implies \sum_{i = 1}^{n} \alpha_i = (n - 2)\pi + \int_{\gamma} \varkappa_g \, ds + \int_{R} \kappa \, dA.
\end{align*}  
\qed
\begin{posledica}
    \label{psl_posledica_globalnega_GB_izreka}
    Če je $R$ mnogokotnik, omejen z  geodetkami, velja \begin{equation*}
        \sum_{i=1}^{n} \alpha_i = (n-2) \pi  + \int_{R} \kappa \, dA. 
    \end{equation*}  
    \end{posledica}
S pomočjo izreka za odsekoma gladke krivulje lahko dokažemo globalno verzijo Gauss-Bonnetovega izreka, ki poveže Gaussovo ukrivljenost z Eulerjevo karakteristiko ploskve. Najprej uvedimo nekaj definicij.

\begin{definicija}
\label{def_sklenjena_mnogoterost}
Mnogoterost $X$ je sklenjena mnogoterost, če je kompaktna in brez roba.
\end{definicija}

\begin{opomba}
V resnici smo na začetku leta definirali samo mnogoterosti brez roba, torej takšne, za katere lahko za vsako točko najdemo neko okolico, homeomorfno odprti krogli.
\end{opomba}

\begin{primer}
Sfera, torus, povezane vsote torusov so sklenjene mnogoterosti. Plašč valja ni sklenjena mnogoterost, ker ima dva roba. 
\end{primer}

\begin{definicija}
\label{def_poligonalizacija_ploskve}
Poligonalizacija ploskve $X$ je družina krivočrtnih mnogokotnikov $\left\{ F_i\right\}_1^{n}$ na $X$, za katero velja \begin{enumerate}
    \item  $F_i^{\text{zap}} \subseteq  X$,
    \item $\bigcup_{i = 1}^{n} F_i = X$,
    \item za vsaka $i \neq j$ je presek $F_i \cap  F_j$ bodisi prazen, bodisi unija (ene ali več) stranic. 
\end{enumerate}
Družina $\left\{ F_i\right\}_1^{n}$ se imenuje množica lic na $X$. Poleg tega definiramo še množico vseh robov $\left\{ E_i\right\}$ ter množico vseh vozlišč $\left\{ V_i\right\}$ na $X$.
\end{definicija}

\begin{definicija}
\label{def_Eulerjeva_karakteristika}
Eulerjeva karakteristika poligonalizacije $\mathcal{P} = \left\{ F, E, V\right\}$ je podana s formulo \begin{equation*}
\chi(\mathcal{P}) =  \#V -  \#E + \#F.
\end{equation*}  
\end{definicija}

Brez dokaza bomo privzeli naslednji izrek, ki poskrbi za dobro definiranost Eulerjeve karakterisitike ploskve.
\begin{izrek}
\label{izr_dobra_definiranost_Eulerjeve_karkteristike}
 Naj bosta $\mathcal{P}_1, \mathcal{P}_2$ dve poligonalizaciji ploskve $X$. Potem velja \begin{equation*}
 \chi(\mathcal{P}_1) = \chi(\mathcal{P}_2).
\end{equation*}  
\end{izrek}

\begin{definicija}
\label{def_Eulerjeva_karkteristika}
Naj bo $\mathcal{P}$ poljubna poligonalizacija ploskve $X$. Eulerjeva karkteristika ploskve $X$ je podana s predpisom \begin{equation*}
\chi(X) = \chi(\mathcal{P}).
\end{equation*}  
   
\end{definicija}

Nazadnje povejmo globalno verzijo Gauss-Bonnetovega izreka.

\begin{izrek}
\label{izr_globalni_GB}
Naj bo $X$ orientabilna sklenjena ploskev. Potem velja \begin{equation*}
2 \pi \chi(X) = \int_{X} \kappa \, dA.
\end{equation*}  
\end{izrek}

\begin{izrek}
\label{izr_Eulerjeva_karakteristika_je_topoloska_invarianta}
Eulerjeva karakteristika ploskve je topološka invarianta. Natančneje, če sta ploskvi $X_1, X_2$ homeomorfni, potem 
je $\chi(X_1) = \chi(X_2)$. 
\end{izrek}

Če sta $X_1, X_2$ sklenjeni in orientabilni, velja tudi obrat, torej iz $\chi(X_1) = \chi(X_2)$ sledi $X_1 \approx X_2$. Še več, velja \begin{equation*}
\chi(X) = 2 - 2g(X),
\end{equation*}  
kjer je $g$ genus oziroma rod ploskve, torej na primer število lukenj v povezani vsoti torusov.

\noindent
{\em Dokaz globalne verzije Gauss-Bonnetovega izreka:\/}
Naj bo $\mathcal{P}$ poligonalizacija ploskve $X$. Za vsako lice $F_i$ iz $\mathcal{P}$ napišimo lokalno Gauss-Bonnetovo formulo:
\begin{equation*}
\sum_{i = 1}^{n_k} \alpha_{k_i} = \pi(n_k - 2) + \int_{F_k} \kappa \, dA + \int_{\partial F_k } \varkappa_g \, ds.  
\end{equation*}  
Vse te formule seštejemo: \begin{equation*}
\sum_{k = 1}^{\#F} \sum_{i = 1}^{n_k} \alpha_{k_i} = \pi \sum_{k = 1}^{\#F}  (n_k - 2) + \sum_{k = 1}^{\#F}  \int_{F_k} \kappa \, dA + \sum_{k = 1}^{\#F}  \int_{\partial F_k } \varkappa_g \, ds = 2 \pi \#E - 2\pi \# F + \int_{X} \kappa \, dA. 
\end{equation*}  
Po lemi o rokovanju je vsota vseh $n_k$ je enaka dvakratniku števila robov, oziroma \begin{equation*}
    \sum_{k = 1}^{\#F} n_k = 2 \# E \implies \sum_{k = 1}^{\#F}  (n_k - 2) = 2 \# E - 2 \# F.
\end{equation*}  
  Ker ima unija robov $\bigcup_{k = 1}^{n} \partial F_k$ na ploskvi mero $0$, imamo zvezo \begin{equation*}
    \sum_{k = 1}^{\#F}  \int_{F_k} \kappa \, dA = \int_{X} \kappa \, dA.  
  \end{equation*}  
Poleg tega je\begin{equation*}
    \sum_{k = 1}^{\#F} \int_{\partial F_k} \varkappa_g \, ds = 0, 
\end{equation*}  
ker ima zaradi skladne orientacije večkotnikov vsak prispevek stranice pri integiranju tudi svoj nasprotno enak prispevek. 
Ker je $X$ sklenjena ploskev, si lahko predstvljamo, da imamo okoli vsakega vozlišča vsoto kotov $2 \pi$, torej skupaj \begin{equation*}
    \sum_{k = 1}^{\#F} \sum_{i = 1}^{n_k} \alpha_{k_i} = 2 \pi \# V.
\end{equation*}  
Ko povežemo vse te enačbe, dobimo \begin{equation*}
\int_{X} \kappa \, dA = 2 \pi (\# V - \# E + \# F ) = 2 \pi \chi(X). 
\end{equation*} 
 \qed

\subsection{Vektorska polja na ploskvah} 

\begin{definicija}
\label{def_vektorsko_polje}
Naj bo $X$ sklenjena, orientabilna ploskev. Vektorsko polje $\zeta$ na $X$ je gladka preslikava $\zeta: X \to  \mathbb{R}^3$, da za vsak $m \in  X$ velja $\zeta(m) \in  T_mX$.
\end{definicija}

\begin{definicija}
\label{def_stacionarna_tocka_vektorskega_polja}
Stacionarna točka oziroma ničla vektorskega polja $\zeta$ je točka $m \in  X$, za katero velja $\zeta(m) = 0$. Ničla je izolirana, če obstaja okolica $m \in  U \subseteq X$, da je $\zeta(p) \neq 0$ za vsak $p \in  U \setminus \left\{ m\right\} $.
\end{definicija}

\begin{definicija}
\label{def_veckratnost_izolirane_nicle}
Večkratnost izolirane ničle $p \in  X$ polja $\zeta$ je podana s predpisom \begin{equation*}
m_p = m(p) = \frac{1}{2\pi} \int_{\gamma} \dot{\psi}(s) \, ds. 
\end{equation*}  
Pri tem je $\gamma$ dovolj majhna krivulja v $X$, ki obkroži točko $p$ in je rob nekega diska v $X$. Kot $\psi(s)$ je kot med
$\dot{\gamma}(s)$ in vektorjem $\eta(\gamma(s))$, kjer je $\eta$ neko neničelno vektorsko polje na disku $U$, ki ga obkroža $\gamma$, torej $\gamma = \partial U$.  
\end{definicija}

\begin{opomba}
    V splošnem imajo različne vrste ničel različne lastnosti. Ničla ima lahko negativno večkratnost.
    \end{opomba}

Pokažimo, da je definicija večkratnosti dobra. \begin{enumerate}
    \item Neodvisnost od izbire $\eta$: Naj bo $\zeta$ neničelno vektorsko polje na $U$. Potem bi imeli po definiciji \begin{equation*}
    m_{\eta}(p) = \frac{1}{2\pi} \int_{\gamma} \dot{\psi}(s) \, ds = \frac{1}{2 \pi} \int_{\gamma} \dot{(\psi + \varphi)}  \, ds,  
    \end{equation*}  
    kjer je $\varphi(s)$ kot med $\eta(s)$ in $\zeta(s)$.
    Dokazati moramo torej \begin{equation*}
        \int_{\gamma} \dot{\varphi}  \, ds = 0.
    \end{equation*}  
    Brez škode za splošnost lahko predpostavimo, da sta $\eta$ in $\zeta$ enotski vektorski polji (tu uporabimo predpostavko o neničelnost teh dveh polj).    
    Potem lahko zapišemo \begin{align*}
        \varphi(s) &= \arccos(\langle \eta(s), \zeta(s) \rangle     )  \\
        \dot{\varphi}(s) &= -\frac{\langle \eta_u \dot{u} + \eta_v \dot{v}, \zeta \rangle + \langle \eta, \zeta_u \dot{u} + \zeta_v \dot{v} \rangle}{\sqrt{1 - \langle \eta, \zeta \rangle^2} } \\
        &= -  \frac{(\langle \eta_u, \zeta \rangle + \langle \eta, \zeta_u \rangle)\dot{u} + (\langle \eta_v, \zeta \rangle + \langle \eta, \zeta \rangle ) \dot{v} }{\sqrt{1 - \langle \eta, \zeta \rangle^2} } \\
        &= - \frac{\langle \eta, \zeta \rangle_u \dot{u} + \langle \eta, \zeta \rangle_v \dot{v} }{\sqrt{1 - \langle \eta, \zeta \rangle^2} }.
    \end{align*}
Na tej točki uporabimo Greenovo formulo za polji $P = - \frac{\langle \eta, \zeta \rangle_u }{\sqrt{1 - \langle \eta, \zeta \rangle^2} }$ in $Q = - \frac{\langle \eta, \zeta \rangle_v }{\sqrt{1 - \langle \eta, \zeta \rangle^2} }$. Pred tem označimo $D = {\sqrt{1 - \langle \eta, \zeta \rangle^2}$. Najprej poračunamo \begin{align*}
    D_u &= -\frac{\langle \eta, \zeta \rangle \langle \eta, \zeta \rangle_u  }{D} \\
    D_v &= -\frac{\langle \eta, \zeta \rangle \langle \eta, \zeta \rangle_v  }{D}.
\end{align*} \begin{align*}
    Q_u - P_v &= -\frac{1}{D^2}\left(\langle \eta, \zeta \rangle_{uv} D - \langle \eta, \zeta \rangle_v D_u - \langle \eta, \zeta \rangle_{uv} D + \langle \eta, \zeta \rangle_u D_v \right) \\
     &= \underbrace{-\langle \eta, \zeta \rangle_{uv} D + \langle \eta, \zeta \rangle_{vu} D}_0  + \frac{1}{D} \underbrace{\left( \langle  \eta, \zeta \rangle_u \langle \eta, \zeta \rangle_v \langle \eta, \zeta \rangle - \langle  \eta, \zeta \rangle_v \langle \eta, \zeta \rangle_u \langle \eta, \zeta \rangle   \right)}_0  = 0
\end{align*}
\item Neodvisnost od poti, ki obkroži točko: Na območjih, ki jih omejujejo zanke iz  obhodov $\gamma$ in $\beta$ sta vektorski polji po predpostavki neničelni.
Po točki 1. je torej \begin{equation*}
\int_{\gamma} \dot{\psi}(s) \, ds - \int_{\beta} \dot{\psi}(s)  \, ds = 0. 
\end{equation*}   
\end{enumerate}

Dokazali bomo naslednji izrek.

\begin{izrek}
\label{izr_vsota_nicel_vektorskega_polja}
 Naj bo $\zeta$ zvezno vektorsko polje na gladki sklenjeni orientabilni ploskvi $X$, ki ima samo izolirane ničle $X_1, X_2, \ldots, X_n$. Potem velja \begin{equation*}
\sum_{i = 1}^{n} m_{\zeta}(X_i) = \chi(X).
 \end{equation*}  
\end{izrek}

\noindent
{\em Dokaz:\/}
Okoli vsake ničle $X_i$ našega polja vzemimo disk $R_i$. Definirajmo $Y = X \setminus \bigcup_{i = 1}^{n} R_i$. Opremimo $Y$ z ortonormiranimi vektorskimi polji $e_1, e_2$. Za vsak $m \in Y$ naj bo $e_1(m) = \frac{\zeta(m)}{\lvert\lvert \zeta(m) \rvert\rvert } \neq 0$. Hkrati naj bo
$e_2 = e_1(m) + \eta e_2(m)$. Po lokalni verziji Gauss-Bonnetovega izreka za gladek rob velja \begin{equation*}
\int_{Y} \kappa  \, dA = -\sum_{i = 1}^{n} \int_{\partial R_i} \langle e_1, \dot{e}_2 \rangle   \, ds.  
\end{equation*}
Predznak je v tej enačbi negativen, ker se po krivuljah $\gamma_i = \partial R_i$ sprehajamo v pozitivni smeri glede na $R_i$, torej v negativni smeri glede na zunanjost $\bigcup_{i = 1}^{n} R_i$.    
Poligonalizirajmo $Y$. Na vseh krivuljah, ki niso zunanje, se prispevki pri integraciji odštejejo. Za vsak $i = 1 ,\ldots, n$ izberemo med seboj ortonormirani polji $f_1^{i}, f_2^{i}$. Potem velja \begin{equation*}
\int_{R_i} \kappa   \, dA = \int_{\gamma_i} \langle f_1^{i}, \dot{f}_2^{i} \rangle   \, ds.  
\end{equation*}  
Iz dokaza Gauss-Bonnetove formule vemo \begin{align*}
    \langle e_1, \dot{e}_2  \rangle  &= \dot{\theta} - \varkappa_g, \\
    \langle f_1^{i}, \dot{f}_2^{i}  \rangle  &= \dot{\phi}_i - \varkappa_g. 
\end{align*}
Pri tem je $\theta(m)$ kot med $e_1(m)$ in $\dot{\gamma}(m)$, $\phi_i(m)$ pa kot med $f_1^{i}(m)$ in $\dot{\gamma}_i(m)$ za vsak $m \in  \gamma_i$. Če te formule povežemo, dobimo \begin{align*}
    \int_{Y} \kappa  \, dA  + \sum_{i = 1}^{n} \int_{R_i} \kappa \, dA  &= \sum_{i = 1}^{n}  \int_{\gamma_i} \langle f_1^{i}, \dot{f}_2^{i} \rangle -  \langle e_1, \dot{e}_2 \rangle    \, ds.    \\
    \int_{X} \kappa  \, dA  &= \sum_{i = 1}^{n} \int_{\gamma_i}  \dot{\phi}_i  - \dot{\theta} \, ds.  
\end{align*}
Zdaj upoštevamo, da je $\theta = \measuredangle{(e_1, \dot{\gamma})}$, $\phi_i = \measuredangle{(f_1^{i}, \dot{\gamma})}$, torej $\phi_i - \theta = \measuredangle{(f_1^{i}, e_1)} = \measuredangle{(f_1^{i}, \zeta)} = \measuredangle{(\zeta, -f_1^i)} =: \psi_i$.
Dobimo \begin{equation*}
  \chi(X) =  \frac{1}{2\pi} \int_{X} \kappa  \, dA = \frac{1}{2\pi}\sum_{i = 1}^{n} \int_{\gamma_i} \dot{\psi}_i  \, ds =  \sum_{i = 1}^{n} m_\zeta(X_i). 
\end{equation*}
\begin{equation*}
\implies \sum_{i = 1}^{n} m(X_i) = \chi(X).
\end{equation*}  
    
\qed

\begin{primer}
    Ker ima sfera $S^2$ Eulerjevo karakteristiko $\chi(S^2) = 2$, na njej ne obstaja (neničelno) vektorsko polje brez ničel. Prav tako ne obstaja vektorsko polje, ki bi imelo eno samo ničlo.
    \end{primer}


Kaj bi se zgodilo, če bi bilo naše polje gradientno polje neke funkcije?

\begin{definicija}
\label{def_gradientno_polje}
Naj bo $f: X \to  \mathbb{R}$ gladka funkcija. Naj bo $\gamma: (-\varepsilon, \varepsilon) \to  X$ gladka krivulja, $\gamma(0) = m$. Gradientno polje $\operatorname{grad}_X f = \nabla_X f$ je po Rieszovem reprezentacijskem izreku enolično določeno na naslednji način:
\[ \frac{d}{dt} \bigg|_{t = 0} f(\gamma(t)) = (D_mf)(\dot{\gamma}(0)) = \langle \nabla_X f, \dot{\gamma}(0) \rangle. \]  
Takšna definicija je legitimna, ker je $D_mf: T_mX \to  \mathbb{R}$ linearen funkcional, vsak linearen funkcional pa lahko predstavimo kot delovanje skalarnega produkta. Skalarni produkt $\langle \nabla_X f, \dot{\gamma}(0) \rangle$ je določen s prvo fundamentalno formo na $T_mX$.
\end{definicija}

Skupaj z izrekom \ref{izr_vsota_nicel_vektorskega_polja}[] smo dobili naslednji izrek.
\begin{izrek}
\label{izr_vsota_nicel_gradientnih_polj}
 Naj bo $X$ sklenjena in orientabilna gladka mnogoterost ter $f: X \to  \mathbb{R}$ taka funckija, da so vse ničle polja $\nabla_X f$ izolirane. Potem je \begin{equation*}
 \chi(X) = \sum_{i = 1}^{n} m_{\nabla_X f} (X_i).
 \end{equation*}  
\end{izrek}

\begin{izrek}
\label{izr_formula_stacionarnih_tock}
Naj bo $X$ sklenjena in orientabilna gladka mnogoterost ter $f: X \to  \mathbb{R}$ vsaj dvakrat zvezno parcialno odvedljiva funckija s končno mnogo izoliranimi stacionarnimi točkami in neizrojeno Hessejevo matriko.
Naj bo $\text{Max}$ število lokalnih maksimumov $f$, $\text{Min}$ število lokalnih minimumov $f$ ter $\text{Sed}$ število sedel $f$. Potem velja \begin{equation*}
\chi(X) = \text{Max} - \text{Sed} + \text{Min}.
\end{equation*}  
\end{izrek}
\noindent
{\em Skica dokaza:\/}
Okoli minumimov in maksimumov funkcije $f$ je večkratnost $m_{\nabla_X f} = 1$. Razmisliti moramo, kako izgleda $\nabla_X f$ v okolici sedla.
Naj bo $R \subseteq X$ tako majhen disk okoli sedla, da se geometrijsko ne razlikuje od situacije na ravnini. Razvijemo $f$ v Taylorjevo vrsto okoli stacionarne točke $\nabla_X f$. \begin{align*}
f(u_0 + h, v_0 + k) &\approx f(u_0 ,v_0) + f_u(u_0, v_0)h + f_v(u_0, v_0)k + \frac{1}{2} \begin{pmatrix}
  h & k 
\end{pmatrix}
\begin{pmatrix}
  f_{uu} & f_{uv} \\
  f_{uv} & f_{vv}
\end{pmatrix}_{(u_0, v_0)}  
\begin{pmatrix}
  h \\
  k 
\end{pmatrix} \\
&= f(u_0, v_0) + \underbrace{\langle \nabla_X f (u_0, v_0), (h, k) \rangle}_0 +  \frac{1}{2} \begin{pmatrix}
    h & k 
  \end{pmatrix}
  \text{Hess}(f)_{(u_0, v_0)}  
  \begin{pmatrix}
    h \\
    k 
  \end{pmatrix}.\end{align*}  
Označimo $\tilde{f} = f - f(u_0, v_0).$ V bižini $(u_0, v_0)$ imamo približno \begin{equation*}
\tilde{f}(u_0 + h, v_0 + k) = \begin{pmatrix}
    h & k 
  \end{pmatrix}
  \begin{pmatrix}
    f_{uu} & f_{uv} \\
    f_{uv} & f_{vv}
  \end{pmatrix}_{(u_0, v_0)}  
  \begin{pmatrix}
    h \\
    k 
  \end{pmatrix} = C.
\end{equation*}  
Ker je Hessejeva matrika simetrična, obstaja ortogonalna preslikava (rotacija) $Q$, da je $Q\text{Hess}(f)_{(u_0,v_0)}Q^{T}$ diagonalna. Tam imamo \begin{equation*}
\tilde{f}(u_0 +h, v_0 + k) = \begin{pmatrix}
    \tilde{h} & \tilde{k} 
  \end{pmatrix}
  \begin{pmatrix}
    \lambda_1 &  \\
     & \lambda_2
  \end{pmatrix}  
  \begin{pmatrix}
    \tilde{h} \\
    \tilde{k} 
  \end{pmatrix} = \lambda_1 \tilde{h}^2 + \lambda_2 \tilde{k}^2 = C. 
\end{equation*}  
V primeru pozitivno ali negativno definitne Hessejeve matrike ($\lambda_1 \lambda_2  > 0$) ta enačba predstavlja elipso. Pokažimo, da je gradient vedno pravokoten na nivojnice. Naj bo $\beta(t)$ nivojnica $f$, torej $f(\beta(t)) = \text{konst.}$\begin{equation*}
\frac{d}{dt} \bigg|_{t = 0} f(\beta(t)) = \langle \nabla_X f, \dot{\beta}(0) \rangle  = 0 \implies \nabla_X f \perp \dot{\beta}(0).
\end{equation*}  
Okoli negativno definitne točke ima torej gradientno polje izvor, okoli pozitino definitne točke pa ponor (prava izbira besede?). Če je Hessejeva matrika neizrojena in nedefinitna, je njena diagonalizacija oblike \begin{equation*}
\begin{pmatrix}
    \lambda_1 & \\
     & -\lambda_2\\
\end{pmatrix},
\end{equation*}kjer je $\lambda_1 \lambda_2 > 0$. V tem primeru za nivojnice velja enačba \begin{equation*}
    \begin{pmatrix}
        \tilde{h} & \tilde{k} 
      \end{pmatrix}
      \begin{pmatrix}
        \lambda_1 &  \\
         & -\lambda_2
      \end{pmatrix}
      \begin{pmatrix}
        \tilde{h} \\
        \tilde{k} 
      \end{pmatrix} = \lambda_1 \tilde{h}^2 - \lambda_2 \tilde{k}^2 = C,
\end{equation*}  
torej so nivojnice hiperbole. Takšnim stacionarnim točkam pravimo starcionarne točke sedla, zanje velja $m_{\nabla_X f} = -1$. Če uporabimo prejšnji izrek za $\zeta = \nabla_X f$ dobimo \begin{equation*}
\chi(X) = \sum_{i = 1}^{n} m_{\nabla_X f}(X_i) = \text{Max} - \text{Sed} + \text{Min}.
\end{equation*}  
\qed